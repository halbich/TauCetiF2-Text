%!TEX root = ../prace.tex

\chapter{Detailní analýza}

\section{použitý herní engine}
- máme několik možností:
- napsat si vlastní (ne, moc práce)
- použít low level (XNA) - opět ne, moc práce
- Unity nebo Unreal Engine\\
- Unity mělo alespoň v době analýzy této práce problémy s dynamickým navmeshem, oproti tomu mělo editovatelný terén. Další nevýhoda je absance editorů materiálu tk jak je tomu v UE
- Unreal je prostě nej

\section{Herní svět}

- Herní svět je složen z bloků\\
- velikost bloků je omezena, minimální velikost bloku je 20 na 3 cm\\
- maximální velikost bloku je 20ti násobek

- mám různé tvary - krychle, stranově seříznutá krychle, tělně říznutá krychle (obrázky)
- pak jsou zde i speciální tvary - ty bylo nutné vymodelovat v Cinemě4D
- speciální tvary definují svoji pevnou velikost, od této definice se pak odvíjí další vlastnosti (výpočet zdraví, energie bloku)

\section{Vlastnosti bloků}
- bloky mohou mít několik vlastností:\\
- mít možnost elektriky a zapojení do elektrické sítě\\
- mít možnost uchování kyslíku, v případě použitíí elektirky pak i generování
- bloky mohou být použitelné, tj. hráč s nimi může nějakým zůsobem interagovat
- bloky mohou být sebratelné, tedy hráč si je může dát do svého inventáře. vlastnosti jako třeba uchovaná hodnota kyslíku, pak zůstávají zachované
- bloy mohou být zákadem pro rozpoznávání tvarů

\section{Uchování informace o blocích ve světě}
- je více mmožností. Uchování pole 50000 x 50000 x 25000 // todo ověřit
je nesmysl. \\
- nepotřebujeme otevřený svět bez mřížky (pozdější aktualizace ME, jinak SE), takže budeme hledat nějakou variantu stromové struktury
- nabízí se možnost clustorování budov a shlukování do skupin, s následnou optimalizací počtu hladin
- my jsme zvolili K-D strom kombinovatný s AABB.
- náš strom  má optimalizaci jedinného potomka, v případě potřeby se dogeneruje do úrovně níže, případně rozpadne na podčásti a rekurzivně se přidá.
- díky této variantě se můžeme snadno dotazovat na sousedy, což je hlavní cíl

\section{Počasí}
- počasí chceme proměnlivé ale s tím, že gamedesignéři mohou snadno ovlňovat výsledné počasí, případně aby šlo snadno rozšířit varianty pro různé herní módy\\
budeme mít ve světě umístenou entitu (Pawn) ovládaný AI Controllerem  - to z toho důvodu, že pro AI Controller můžeme použít BehaviorTree
\\
- popsat BT\\
další možnosti by byly, že bychom prostě použili update smyčku nějakého Actora - není potřeba, tohle se vyřeší updatem na komponentě


Zde bych měl realisticky vybrané cíle rozebrat do podrobna.

\begin{itemize}
	\item Svět + jak vypadá + jak je reprezentován (K-D tree zde, nebo v programátorské?)
	\item Popsat bloky, velikosti
	\item Popsat komponenty bloků (že je něco jako komponenta elektriky, komponenta vzduchu
	\item Popsat hratelnou postavičku (že má taky možnost elektriky a kyslíku
	\item Popsat počasí - že má taky svoji blokovou reprezentaci a že je na pozadí Behavior Tree, který to celé řídí (ovlivňování konfigurace v programátorské části)
	\item Popsat implementaci elektriky // TODO dodělat ve hře
	\item Popsat implementaci rozpoznávání tvarů // TODO dodělat ve hře
	\item Popsat způsob ukládání a načítání hry (to je možná až do Progr. sekce?)
	\item Popsat, že bychom chtěli nějaké UI + nabídky menu
	\item Popsat, že bychom chtěli základní hudbu
	\item Popsat, že máme něco jako inventář s možností nějaké správy bloků
	\item Stejně tak pro builder + herní terminály  //TODO doimplementovat
\end{itemize}
