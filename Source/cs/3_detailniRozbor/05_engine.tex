%!TEX root = ../prace.tex

\section{Herní engine}
V první řadě bychom se měli zamyslet nad tím, jaký nástroj pro vývoj hry použijeme. Díky tomu budeme moct počítat s možnostmi a omezeními danými touto volbou. Shrňme si, co budeme ve hře potřebovat:

\begin{itemize}
	\item Renderování 3D objektů, pokročilé možnosti texturování
	\item Podpora I/O pro práci se savy
	\item Podpora UI
	\item Podpora zvuků
	\item Snadná implementace lokalizace
	\item Správa assetů
	\item Správa scény
	
\end{itemize}

Pro další případný rozvoj bychom potřebovali:

\begin{itemize}
	\item Podpora pathfindingu
	\item Podpora síťové hry
	\item Podpora AI
\end{itemize}

Cílové platformy pro nás bude PC s OS Windows. Pokud se rozhodneme pro již existující herní engine, který bude navíc podporovat multiplatformní vývoj, bude to pro nás, i s ohledem na další vývoj, plus.

Dalším kritériem je volba programovacího jazyka. Ta vychází z autorových znalostí. Budeme tedy preferovat primárně jazyk \CS{}, který známe nejlépe. Pokud to bude nezbytně nutné, nebudeme se bránit ani jazyku \CPP{}, který je v herní branži dlouho zavedený a je stále hojně využívaný. Ačkoliv zkušenost s tímto programovacím jazykem máme minimální, můžeme se tímto způsobem naučit novým dovednostem.


Možných použitých enginů a frameworků je opravdu mnoho. Podívat do databáze herních enginů na stránce Devmaster. Jen zde je možné nalézt 236 možných řešení našeho problému volby herního enginu \citep{engines_list}. Všechny záznamy jsme omezili na \textit{vývojově aktivní}, v jazycích \CS{}, \CPP{} a vybrali jsme námi požadované vlastnosti.

Mezi čím tedy můžeme volit?
\begin{itemize}
	\item Implementace proprietárního enginu
	\item Použití existující grafické knihovny
	\item Použití existující herní engine
\end{itemize}

Je zřejmé, že možností na výběr máme opravdu hodně. V následujících podkapitolách si jednotlivé možnosti podrobně rozebereme.

\subsection{Vlastní engine}
Hlavní výhodou, ale zároveň nevýhodou této volby je to, že bychom si všechny potřebné součásti enginu (třeba renderování) museli napsat sami. Tím bychom měli naprostou kontrolu nad celým produktem, ale zabralo by nám to netriviální množství času.
Vzhledem k rozsahu plánované funkcionality by tato volba byla nepraktická a tedy touto cestou se nevydáme.

\subsection{Nízkoúrovňový framework}
Máme na výběr z více druhů frameworků postavených na různých platformách. Mezi známějšími bychom mohli uvést například XNA (\texttt{C\#}) nebo Ogre (\texttt{C++}). Oba frameworky jsou k dispozici zdarma, nicméně jejich podpora stagnuje. Implementace hry s použitím některého z těchto frameworků by byla rychlejší než v předchozím případě, ale stále bychom museli spoustu funkcionality implementovat sami. 

\subsection{Existující herní engine}
V této kategorii máme nejvíce možností jak rychle implementovat celou práci. Zástupců je opět mnoho, nicméně mezi nejoblíbenější se řadí Unity (\texttt{C\#}) a \UE{}(\texttt{C++}), které jsou oba zdarma. Díky práci komunity pro oba enginy existuje i kvalitní dokumentace. Navíc jsou enginy obvykle multiplatformní a tedy existuje zde snadný postup distribuce na různé typy herních zařízení. Při rešerši jsme zjistili následující klady a zápory:

\subsubsection{Unity}
Výhoda (ale i nevýhoda) Unity je celkově v jednoduchosti. Engine nabízí dostatečné možnosti i pro tvorbu AAA herních titulů, ale za cenu toho, že si toho autoři musejí dost napsat sami (oproti \UE{}). 
\begin{itemize}
	\item Klady
		\subitem -- Pokud bychom chtěli modifikovatelný či nějakým způsobem dynamický terén, Unity implementuje podporu modifikace terénu.
	\item Zápory
		\subitem -- Při rešerši jsme řešili podporu dynamického navigačního meshe, která v Unity nebyla příliš dobrá. Při změně prostředí docházelo k lagům během přepočtu navmeshe. To by byl při umisťování bloků zásadní problém.
		\subitem -- Další nevýhodou je podpora materiálů, kdy bychom snadným způsobem nedosáhli vizuálně přitažlivého prostředí.

\end{itemize}

\subsubsection{Unreal Engine}
Oproti Unity je \UE{} podstatně komplexnější a pochopení všech vztahů a závislostí může být pro začínajícího herního programátora obtížné. Dalším zdrojem problémů může být i programování v \texttt{C++}, které je navíc díky technologii \UBT{} trochu jiné než klasické \texttt{C++} a je potřeba dodržovat standardy definované \UE{}.
\begin{itemize}
	\item Klady
		\subitem -- Komplexnější engine (\UE{} je primárně určen pro vývoj AAA titulů)
		\subitem -- Oproti Unity lepší podpora materiálů --- i negrafik může snadno vytvořit vizuálně přitažlivé povrchy objektů a nemusí se přitom zabývat psaním vlastních shaderů 
		\subitem -- Rychlý dynamický navmesh
	\item Zápory
		\subitem -- Komplexnější engine

\end{itemize}

Nakonec jsme zvolili poslední možnost - Unreal Engine. Autorovy znalosti především z oblasti \texttt{C\#} sice hovořily pro použití Unity, nicméně výhody použití \UE{} převážily nad nevýhodami i všemi výhodami Unity.	// TODO tohle chce vyladit