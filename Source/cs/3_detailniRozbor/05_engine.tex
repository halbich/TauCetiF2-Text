%!TEX root = ../prace.tex

\section{Herní engine}
Nyní už víme, čeho bychom chtěli dosáhnout a je na čase vyřešit, jak toho dosáhneme. V první řadě bychom se měli zamyslet nad tím, jaký herní engine použijeme. Díky tomu budeme moct počítat s možnostmi a omezeními danými touto volbou. V zásadě máme několik možností:

\begin{itemize}
	\item Implementace vlastního enginu
	\item Použití existující nízkoúrovňový framework
	\item Použití existující herní engine
\end{itemize}

Každá volba má své pro a proti, což podrobně rozebereme v následujících odstavcích.

\subsection{Vlastní engine}
Hlavní výhodou, ale zároveň nevýhodou této volby je to, že bychom si všechny potřebné součásti enginu (třeba renderování) museli napsat sami. Tím bychom měli naprostou kontrolu nad celým produktem, ale zabralo by nám to netriviální množství času.
Vzhledem k rozsahu plánované funkcionality by tato volba byla nepraktická a tedy touto cestou se nevydáme.

\subsection{Nízkoúrovňový framework}
Máme na výběr z více druhů frameworků postavených na různých platformách. Mezi známějšími bychom mohli uvést například XNA (\texttt{C\#}) nebo Ogre (\texttt{C++}). Oba frameworky jsou k dispozici zdarma, nicméně jejich podpora stagnuje. Implementace hry s použitím některého z těchto frameworků by byla rychlejší než v předchozím případě, ale stále bychom museli spoustu funkcionality implementovat sami. 

\subsection{Existující herní engine}
V této kategorii máme nejvíce možností jak rychle implementovat celou práci. Zástupců je opět mnoho, nicméně mezi nejoblíbenější se řadí Unity (\texttt{C\#}) a \UE{}(\texttt{C++}), které jsou oba zdarma. Díky práci komunity pro oba enginy existuje i kvalitní dokumentace. Navíc jsou enginy obvykle multiplatformní a tedy existuje zde snadný postup distribuce na různé typy herních zařízení. Při rešerši jsme zjistili následující klady a zápory:

\subsubsection{Unity}
Výhoda (ale i nevýhoda) Unity je celkově v jednoduchosti. Engine nabízí dostatečné možnosti i pro tvorbu AAA herních titulů, ale za cenu toho, že si toho autoři musejí dost napsat sami (oproti \UE{}). 
\begin{itemize}
	\item Klady
		\subitem -- Pokud bychom chtěli modifikovatelný či nějakým způsobem dynamický terén, Unity implementuje podporu modifikace terénu.
	\item Zápory
		\subitem -- Při rešerši jsme řešili podporu dynamického navigačního meshe, která v Unity nebyla příliš dobrá. Při změně prostředí docházelo k lagům během přepočtu navmeshe. To by byl při umisťování bloků zásadní problém.
		\subitem -- Další nevýhodou je podpora materiálů, kdy bychom snadným způsobem nedosáhli vizuálně přitažlivého prostředí.

\end{itemize}

\subsubsection{Unreal Engine}
Oproti Unity je \UE{} podstatně komplexnější a pochopení všech vztahů a závislostí může být pro začínajícího herního programátora obtížné. Dalším zdrojem problémů může být i programování v \texttt{C++}, které je navíc díky technologii \UBT{} trochu jiné než klasické \texttt{C++} a je potřeba dodržovat standardy definované \UE{}.
\begin{itemize}
	\item Klady
		\subitem -- Komplexnější engine (\UE{} je primárně určen pro vývoj AAA titulů)
		\subitem -- Oproti Unity lepší podpora materiálů --- i negrafik může snadno vytvořit vizuálně přitažlivé povrchy objektů a nemusí se přitom zabývat psaním vlastních shaderů 
		\subitem -- Rychlý dynamický navmesh
	\item Zápory
		\subitem -- Komplexnější engine

\end{itemize}

Nakonec jsme zvolili poslední možnost - Unreal Engine. Autorovy znalosti především z oblasti \texttt{C\#} sice hovořily pro použití Unity, nicméně výhody použití \UE{} převážily nad nevýhodami i všemi výhodami Unity.	// TODO tohle chce vyladit