%!TEX root = ../prace.tex

\chapter{Detailní analýza}


%!TEX root = ../prace.tex

\section{Herní engine}
Nyní už víme, čeho bychom chtěli dosáhnout a je na čase vyřešit, jak toho dosáhneme. V první řadě bychom se měli zamyslet nad tím, jaký herní engine použijeme. Díky tomu budeme moct počítat s možnostmi a omezeními danými touto volbou. V zásadě máme několik možností:

\begin{itemize}
	\item Implementace vlastního enginu
	\item Použití existující nízkoúrovňový framework
	\item Použití existující herní engine
\end{itemize}

Každá volba má své pro a proti, což podrobně rozebereme v následujících odstavcích.

\subsection{Vlastní engine}
Hlavní výhodou, ale zároveň nevýhodou této volby je to, že bychom si všechny potřebné součásti enginu (třeba renderování) museli napsat sami. Tím bychom měli naprostou kontrolu nad celým produktem, ale zabralo by nám to netriviální množství času.
Vzhledem k rozsahu plánované funkcionality by tato volba byla nepraktická a tedy touto cestou se nevydáme.

\subsection{Nízkoúrovňový framework}
Máme na výběr z více druhů frameworků postavených na různých platformách. Mezi známějšími bychom mohli uvést například XNA (\texttt{C\#}) nebo Ogre (\texttt{C++}). Oba frameworky jsou k dispozici zdarma, nicméně jejich podpora stagnuje. Implementace hry s použitím některého z těchto frameworků by byla rychlejší než v předchozím případě, ale stále bychom museli spoustu funkcionality implementovat sami. 

\subsection{Existující herní engine}
V této kategorii máme nejvíce možností jak rychle implementovat celou práci. Zástupců je opět mnoho, nicméně mezi nejoblíbenější se řadí Unity (\texttt{C\#}) a \UE{}(\texttt{C++}), které jsou oba zdarma. Díky práci komunity pro oba enginy existuje i kvalitní dokumentace. Navíc jsou enginy obvykle multiplatformní a tedy existuje zde snadný postup distribuce na různé typy herních zařízení. Při rešerši jsme zjistili následující klady a zápory:

\subsubsection{Unity}
Výhoda (ale i nevýhoda) Unity je celkově v jednoduchosti. Engine nabízí dostatečné možnosti i pro tvorbu AAA herních titulů, ale za cenu toho, že si toho autoři musejí dost napsat sami (oproti \UE{}). 
\begin{itemize}
	\item Klady
		\subitem -- Pokud bychom chtěli modifikovatelný či nějakým způsobem dynamický terén, Unity implementuje podporu modifikace terénu.
	\item Zápory
		\subitem -- Při rešerši jsme řešili podporu dynamického navigačního meshe, která v Unity nebyla příliš dobrá. Při změně prostředí docházelo k lagům během přepočtu navmeshe. To by byl při umisťování bloků zásadní problém.
		\subitem -- Další nevýhodou je podpora materiálů, kdy bychom snadným způsobem nedosáhli vizuálně přitažlivého prostředí.

\end{itemize}

\subsubsection{Unreal Engine}
Oproti Unity je \UE{} podstatně komplexnější a pochopení všech vztahů a závislostí může být pro začínajícího herního programátora obtížné. Dalším zdrojem problémů může být i programování v \texttt{C++}, které je navíc díky technologii \UBT{} trochu jiné než klasické \texttt{C++} a je potřeba dodržovat standardy definované \UE{}.
\begin{itemize}
	\item Klady
		\subitem -- Komplexnější engine (\UE{} je primárně určen pro vývoj AAA titulů)
		\subitem -- Oproti Unity lepší podpora materiálů --- i negrafik může snadno vytvořit vizuálně přitažlivé povrchy objektů a nemusí se přitom zabývat psaním vlastních shaderů 
		\subitem -- Rychlý dynamický navmesh
	\item Zápory
		\subitem -- Komplexnější engine

\end{itemize}

Nakonec jsme zvolili poslední možnost - Unreal Engine. Autorovy znalosti především z oblasti \texttt{C\#} sice hovořily pro použití Unity, nicméně výhody použití \UE{} převážily nad nevýhodami i všemi výhodami Unity.	// TODO tohle chce vyladit

%!TEX root = ../../prace.tex

\section{Bloky}

Zde by měl být popis možností jak definovat a~následně implementovat bloky. jaké jsou výhody a~nevýhody jednotlivých implementací

- externě editovatelné formáty (+ - modding, - těžší implementace, parsing, validace)
- binární formát

- xml


- interní formát
- specifické subclassy pro bloky včetně specifických vlastností přímo na 
- definiční struktura



%!TEX root = ../prace.tex

\section{Vlastnosti bloků}

- bloky mohou mít několik vlastností:

- mít možnost elektriky a zapojení do elektrické sítě

- mít možnost uchování kyslíku, v případě použitíí elektirky pak i generování

- bloky mohou být použitelné, tj. hráč s nimi může nějakým zůsobem interagovat

- bloky mohou být sebratelné, tedy hráč si je může dát do svého inventáře. vlastnosti jako třeba uchovaná hodnota kyslíku, pak zůstávají zachované

- bloy mohou být zákadem pro rozpoznávání tvarů (TODO )

\subsection{Elektrika}


\subsection{Kyslík}


\subsection{Označovatelnost}


\subsection{Možnost vzít do inventáře}

\subsection{Interakce}


%!TEX root = ../../prace.tex



\section{Komponenty bloků}

Abychom mohli snadno rozšiřovat vlastnosti a~chování bloků, použijeme systém \textit{komponent}, který nám \UE{} nabízí. Komponenta je programová část, která ovlivňuje chování vlastníka dané komponenty. Cílem je pak dosáhnout toho, že je možné za běhu hry jednu komponentu transparentně vyměnit za jinou (komponentu s~jinou implementací), a~vlastník komponenty se nemusí zajímat o~detaily implementace. V naší hře toto chování nejspíše nevyužijeme, ale použití komponent není na škodu a~v~případě dalšího vývoje budeme mít snazší práci. (TODO učesat)

Z předchozí analýzy vyplývá, že budeme potřebovat řešit následující problémy:

\begin{itemize}
	\item Práce s~kyslíkem
	\item Práce s~elektrickou sítí a~energií
	\item Interakce s~uživatelem
	\item Umístění bloku v~herním světě
\end{itemize}


První dva body jsou ideální kandidáti na použití komponent. Pokud bychom se někdy v~budoucnu rozhodli upravit chování této funkcionality či jej z~libovolného důvodu měnit, komponentový systém pro nás bude výhodou. Navíc ne všechny herní bloky umí (z hlediska herního designu) s~danými prvky pracovat. Jak jsme již zmínili dříve, \UE{} nepovoluje vícenásobnou dědičnost a~bylo by velmi těžké vymyslet hiearchii dědičnosti bloků tak, abychom splnili požadavky pro všechny bloky a~zároveň si \uv{nesvázali ruce} pro nové bloky. S použitím komponent to bude snadné -- bloky, které danou funkcionalitu mají umět, budou mít danou komponentu a~budou s~ní moci pracovat.

Dalším problémem je interakce s~uživatelem. Abychom věděli, že hráč s~daným blokem chce interagovat, musíme vědět, že:
\begin{itemize}
	\item Je dostatečně blízko bloku
	\item Z pohledu hráče se dívá na daný blok 
	\item Vyjadřuje fakt, že chce interagovat (např. stiskem klávesy)
\end{itemize}


\subsection{Interakce a~označování}

Nejsnazší způsob, jak zjistit, na jaký herní objekt se hráč dívá, je použití RayTracingu (TODO link?, formát textu?). Díky němu můžeme \uv{z kamery} vyslat virtuální paprsek, který má stejný směr, jako je směr pohledu kamery. Pokud bude hráčův \HUD{} zobrazovat zaměřovací kříž či nějaký obdobný mechanismus a~náš paprsek bude z~pohledu kamery tímto zaměřovačem procházet, hráč může cíleně mířit na herní objekty a~my zároveň budeme mít správnou informaci o~objektu, na který hráč zaměřovačem míří. Tento způsob získávání informace o~objektech v~hráčově zaměřovači je ve hrách běžný a~použití RayTrace je (pokud je vhodně použito) i~dostatečně rychlé.

Nyní, když už víme, jak můžeme získávat informace o~tom, na který objekt hráč míří, tak tento mechanismus ještě rozšíříme o~další vlastnost. Je zapotřebí si uvědomit, že interakce s~blokem a~umisťování nového herního bloku (případně mazání) jsou prakticky jedna a~ta samá akce. Liší se pouze výsledkem -- reakcí na stisk nějaké klávesy či tlačítka myši. Ale ve všech případech musíme vědět, na jaký blok hráč míří zaměřovačem, u~umisťování navíc potřebujeme znát i~přesný polygon, na který hráč míří. Konkrétní polygon potřebujeme znát z~toho důvodu, že chceme, aby se přidávaný blok \uv{přilepil} k~bloku, na který míříme. Tedy chceme zachovat herní mechaniku, která je v~hrách z~kapitoly \ref{chap:uvod} běžná a~je natolik intuitivní a~rozšířená, že změna této mechaniky by nejspíše nedopadla dobře a~hráči by nebyla kladně přijata.

Všechny tyto požadavky lze splnit použitím metody \TT{LineTraceSingleByObjectType} (TODO ref?), které předáme správné parametry (především počátek a~konec paprsku a~typy objektů , které paprsek zaznamená) a~ta nám vrátí strukturu, popisující výsledek RT. Z něj se můžeme dozvědět, jestli byl nějaký blok v~cestě paprsku. A pokud ano, můžeme se ptát, zda měl komponentu interakce (potenciálně bychom mohli chtít bloky bez možnosti zaměření a~interakce, jakožto nesmazatelné objekty). Pokud bude i~tato podmínka splněna, můžeme se zajímat o~další vlastnosti kolize paprsku s~blokem a~na základě toho se nějak chovat.


\subsection{Umístění ve světě}

Smyslem této komponenty je oddělení implementace bloku jako takového a~implementace herního světa.


Ve výsledku tedy budeme chtít komponentu, která 

popis jednotlivých komponent dle předchozího, co všechno umí (např. přidání / odebrání hodnoty energie za použité zámku (není transakce))





%!TEX root = ../prace.tex

\section{Vlastnosti bloků}

- bloky mohou mít několik vlastností:

- mít možnost elektriky a zapojení do elektrické sítě

- mít možnost uchování kyslíku, v případě použitíí elektirky pak i generování

- bloky mohou být použitelné, tj. hráč s nimi může nějakým zůsobem interagovat

- bloky mohou být sebratelné, tedy hráč si je může dát do svého inventáře. vlastnosti jako třeba uchovaná hodnota kyslíku, pak zůstávají zachované

- bloy mohou být zákadem pro rozpoznávání tvarů (TODO )

\subsection{Elektrika}


\subsection{Kyslík}


\subsection{Označovatelnost}


\subsection{Možnost vzít do inventáře}

\subsection{Interakce}


%!TEX root = ../../prace.tex

\section{Bloky v herním světě}


- je více mmožností. Uchování pole 50000 x 50000 x 25000 // todo ověřit
je nesmysl. 

- nepotřebujeme otevřený svět bez mřížky (pozdější aktualizace ME, jinak SE), takže budeme hledat nějakou variantu stromové struktury

- nabízí se možnost clustorování budov a shlukování do skupin, s následnou optimalizací počtu hladin

- my jsme zvolili K-D strom kombinovatný s AABB. (proč? )

- náš strom má optimalizaci jedinného potomka, v případě potřeby se dogeneruje do úrovně níže, případně rozpadne na podčásti a rekurzivně se přidá.

- díky této variantě se můžeme snadno dotazovat na sousedy, což je hlavní cíl (proto)



%!TEX root = ../../prace.tex

\section{Počasí}

Abychom mohli popsat koncept počasí ve hře, musíme mít definovaný \textit{typ} počasí. Typ popisuje vlastnosti počasí (například \uv{je částečně zataženo}, \uv{hustě prší}) na základě nějakých parametrů. Základní koncept počasí ve hře pak můžeme popsat dvěma stavy -- typ počasí zůstává \textit{stejný}, nebo se \textit{mění}. Ve vlastnostech typu pak můžeme mít definováno, jak dlouho zůstává počasí v~daném typu a~jak dlouho probíhá změna na jiný typ (tyto konstanty mohou být dány intervalem, ze kterého se zvolí náhodné číslo a~tím dostaneme jistý stupeň \textit{proměnlivosti} počasí.

Přístupů, jak implementovat systém počasí, je opět více. Jednou možností je vytvořit herní objekt, který bude dědit z~třídy \TT{UActor}. V~aktualizační smyčce tohoto herního objektu pak můžeme mít implementovanou logiku, která bude aktualizovat počasí ve hře. Další možností je použít jednoduchý \textit{Behavior tree}, tedy strom chování. Použitím \textit{Behavior tree} získáváme možnost, jak měnit chování počasí v~Editoru bez nutnosti zásahů do zdrojových kódů hry. 

My jsme se rozhodli, že systém počasí implementujeme jako herní entitu s~umělou inteligencí, kterou bude tvořit právě \textit{Behavior tree}. Chceme, abychom mohli snadno spravovat chování počasí a~navíc tato funkcionalita není natolik výpočetně náročná, abychom ji nutně museli implementovat v~\CPP{}.

\textit{Typy} počasí pak budeme definovat v~Editoru, přičemž stejně jako u~definic bloků vytvoříme definiční třídu s~odpovídající strukturou v~\CPP{}. Tento přístup vyžadujeme z~toho důvodu, že budeme chtít ukládat stav počasí do souboru uložené hry a~pak tento stav při nahrávání opět obnovit. 

Dále požadujeme následující vlastnosti typu počasí: zda je daný typ \textit{bouřkový}, \textit{délku trvání}, \textit{čas změny} na jiný typ, koeficient \textit{rychlosti mraků} a~\textit{zataženosti oblohy}. Nechceme se zabývat grafickými detaily, takže jediným způsobem, jak můžeme hráči počasí vizualizovat, je zobrazením mraků. Podle koeficientu zataženosti pak budeme chtít měnit i~intenzitu osvětlení od Slunce. Protože je naše počasí velice jednoduché, budeme chtít, všechny vlastnosti byly zadány intervalem (vyjma informaci o~tom, zda jde typ bouřkový).

Bouřkový typ počasí budeme chápat speciální způsobem. V~průběhu tohoto typu počasí budeme poškozovat bloky a~tím tak budeme simulovat bouři kyselých dešťů. 

%!TEX root = ../../prace.tex

\section{Hráčova postava}
Hráčovu postavu nemusíme nijak dlouze rozebírat. Budeme nám stačit se držet základního konceptu ovládatelných postav v \UEu{}. Této ovládatelné postavě přidáme komponenty kyslíku, energie a inventáře. V kapitole (TODO ref) jsme zjistili, že budeme chtít umět označovat bloky, proto herní postavě navíc přidáme komponentu, která se bude starat o výpočty sledovacího paprsku. Tato komponenta bude mít k dispozici referenci na aktuální kameru, přiřazenou k ovládané postavě, což jí umožní tyto výpočty provádět.


%!TEX root = ../../prace.tex

\section{Inventář}

Inventář bude vhodné implementovat jako komponentu, kterou posléze můžeme přiřadit herní postavě. Opět se zde opíráme o~fakt, že bychom v~budoucnu mohli mít bloky či \NPC{}, kteří také mají svůj inventář. Z~kapitoly \ref{subsec:inventory} vyplývá, že inventář má několik základních vlastností:

\begin{itemize}
	\item Seznam postavitelných bloků
	\item Seznam umístitelných bloků
	\item Seznam inventárních skupin pro filtrování
\end{itemize}

V momentě, kdy budeme chtít hráči nabídnout bloky k~postavení či umístění, oba seznamy dofiltrujeme dle aktuálně zvolené inventární skupiny. Musíme se tedy zaměřit na to, jak budeme řešit inventární skupiny.

\subsection{Inventární skupiny}

Protože má každý blok své definované \textit{tagy}, musíme vymyslet takový systém, abychom snadno filtrovali podle těchto tagů. Nejsnazší způsob je takový, kdy každá inventární skupina definuje, jaké tagy musí blok mít, aby byl přiřazen. Takže pro blok, který má být zobrazen v~rámci inventární skupiny, platí, že blok definuje všechny tagy v~rámci dané inventární skupiny. Nicméně se může stát, že hráč bude chtít mít v~dané skupině takové bloky, které nemusí splňovat všechny tagy, ale splňují alespoň jeden z~definované skupiny. Z~této myšlenky se dostáváme k~jednoznačnému zápisu algoritmu filtrování tagů -- \CNF{}. \CNF{} je označení pro \textit{konjunktivně normální formu}, užívané ve výrokové logice.

Formálně bychom tuto myšlenku zapsali jako:
\begin{equation}\label{eq:cnf}
	( A_1 \lor A_2 \lor ... \lor A_n ) \land ( B_1 \lor B_2 \lor ... \lor B_n ) \land ... \land ( X_1 \lor X_2 \lor ... \lor X_n )
\end{equation}


Při pohledu na formuli \ref{eq:cnf} vidíme, že můžeme implementovat takový systém, kdy \textit{inventární skupina} definuje \textit{inventární skupiny tagů} (konjunktivní vyhodnocení), přičemž \textit{inventární skupiny tagů} definují \textit{skupiny tagů} (disjunktivní vyhodnocení). Při vyhodnocování, zda blok patří do dané \textit{inventární skupiny}, tedy budeme sledovat to, zda daný blok splňuje všechny tyto \textit{inventární skupiny tagů} a~pro každou z~nich budeme zjišťovat, zda blok definuje alespoň jeden tag ze \textit{skupiny tagů}. 

Abychom hráči ještě více zpříjemnili práci s~inventářem, nebudeme požadovat přesnou shodu tagů definovaných blokem a~\textit{skupinou tagů}. Bude nám totiž stačit pouze podmínka podřetězce, tedy podmínka, aby tag bloku obsahoval jako podřetězec tag ze \textit{skupiny tagů}. Tím dosáhneme toho, že hráč nebude muset v~inventáři vypisovat celé řetězce z~tagů, které si nadefinoval u~svých bloků. 





%!TEX root = ../../prace.tex

\section{Ukládání hry}

Možností, jak implementovat systém ukládání hry, je opět více. Stejně jako u~definic bloků v~kapitole \ref{sec:db} můžeme volit mezi textovým a~binárním formátem. Očekáváme, že soubor uložené hry bude obsahovat velké množství dat, a~proto bychom chtěli minimalizovat výslednou velikost uloženého souboru. Z~tohoto požadavku vyplývá, že nebudeme chtít ukládat hry v~textovém formátu.

Při rešerši možného řešení jsme objevili zajímavý tutoriál (wiki stránka tutoriálu~\citep{ue_save_system}) na ukládání hry v~binárním formátu. Pojďme se podívat, co nám dané řešení nabízí.

Výše zmíněný systém ukládání je postaven na faktu, že v~C++ je možné přetěžovat operátory, mimo jiné i~streamové operátory $<<$. Této vlastnosti je využito tak šikovně, že v~závislosti na volání funkce buď zapisuje do archivu (souboru uložené hry), nebo z~něj čte. Pořád se však jedná o~jediný zápis jedné funkce. To je výhodné, protože to předchází chybám, které by mohly vzniknout při použití 2 metod (jedné čtecí, jedné zapisovací). Také tím předcházíme chybám typu \textit{přehození} pořadí datových \textit{typů} (což by v~případě typů různých velikostí znamenalo následné špatné pochopení binárních dat, nebo rovnou pád aplikace), nebo kupříkladu prohození dvou vlastností stejného typu, což by vytvářelo těžko odhalitelné situace změn hodnot ve hře. Dále nám tento systém nabízí možnost \textit{komprese} dat. Pokud bychom měli ukládané soubory příliš velké, tuto vlastnost bychom mohli využít a~hráči tak šetřit místo na disku.

Shledali jsme, že je pro nás přínosné využít tento způsob ukládání dat.


%!TEX root = ../prace.tex

\section{Doplňující vlastnosti}



\subsection{Lokalizace}

- použití lokalizace 

\subsection{Hudba}

- atmosférický hudební doprovod




\chapter{Backlog}

\begin{itemize}
	\item Svět + jak vypadá + jak je reprezentován (K-D tree zde, nebo v programátorské?)
	\item Popsat bloky, velikosti
	\item Popsat komponenty bloků (že je něco jako komponenta elektriky, komponenta vzduchu
	\item Popsat hratelnou postavičku (že má taky možnost elektriky a kyslíku
	\item Popsat počasí - že má taky svoji blokovou reprezentaci a že je na pozadí Behavior Tree, který to celé řídí (ovlivňování konfigurace v programátorské části)
	\item Popsat implementaci elektriky // TODO dodělat ve hře
	\item Popsat implementaci rozpoznávání tvarů // TODO dodělat ve hře
	\item Popsat způsob ukládání a načítání hry (to je možná až do Progr. sekce?)
	\item Popsat, že bychom chtěli nějaké UI + nabídky menu
	\item Popsat, že bychom chtěli základní hudbu
	\item Popsat, že máme něco jako inventář s možností nějaké správy bloků
	\item Stejně tak pro builder + herní terminály  //TODO doimplementovat
\end{itemize}
