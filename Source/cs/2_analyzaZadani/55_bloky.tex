%!TEX root = ../prace.tex

\subsection{Bloky}

různé druhy, velikosti, jejich vizuální reprezentace, rozšiřovatelnost, obecně co všechno by měly umět.

Název - Min - Max - Pitch - Roll - Type (kostka, zkosený, roh, vlastní)

Tam kde Min == Max -> Vlastní škálování

Typ ovlivňuje další chování

Třeba u Světla by typ mohl být i K a hra by se chovala stejně, K = 1, Z = 0.5, R = 1/6, V = 1 (není v potaz objem)

komponenty bloků a nějaké další ptákoviny

\begin{tabular}{|rll*{5}{c}|}
	\hline
	\tableColumnTitles{Název}								{	&	Min		&	Max			&	P			&	R			&T	}		\hline
	\currentCategory{\textbf{Základní bloky}} 																					\\		\hline
		\mytablerow 				& Blok základny				& 1--1--4	& 20--20--4		& 				& 				&K	\\		\hline
		\mytablerow 				& Blok stavby				& 1--1--1	& 20--20--20	& \checkmark	& \checkmark	&K	\\		\hline
		\mytablerow 				& Blok polykarbonátu		& 1--1--1	& 20--20--20	& \checkmark	& \checkmark	&K	\\		\hline
		\mytablerow 				& Zkosený blok základny		& 1--1--4	& 20--20--4		& 				& 				&Z	\\		\hline
		\mytablerow 				& Zkosený blok stavby		& 1--1--1	& 20--20--20	& \checkmark	& \checkmark	&Z	\\		\hline
		\mytablerow 				& Roh bloku stavby			& 1--1--1	& 20--20--20	& \checkmark	& \checkmark	&R	\\		\hline
	\currentCategory{\textbf{Speciální bloky}} 									 												\\		\hline
		\mytablerow 				& Terminál			 		& 1--8--5 	& 1--8--5		& 				& 				&V	\\		\hline
		\mytablerow 				& Napájené okno				& 2--1--2	& 20--1--20		& \checkmark	& \checkmark	&K	\\		\hline
		\mytablerow 				& Dveře 					& 7--7--11	& 7--7--11		& 				& 				&V	\\		\hline
		\mytablerow 				& Světlo					& 1--1--1	& 1--1--1		& \checkmark	& \checkmark	&V	\\		\hline
		\mytablerow 				& Přepínač 					& 1--1--1	& 1--1--1		& \checkmark	& \checkmark	&V	\\		\hline
		\mytablerow 				& Generátor energie			& 3--3--2	& 20--20--2		& 				& 				&K	\\		\hline
		\mytablerow 				& Generátor objektů 		& 3--3--2	& 20--20--2		& 				& 				&K	\\		\hline
		\mytablerow 				& Akumulátor				& 3--3--3	& 3--3--3		& 				& 				&V	\\		\hline
		\mytablerow 				& Plnička kyslíkových bomb 	& 4--3--4	& 4--3--4		& 				& 				&V	\\		\hline
		\mytablerow 				& Kyslíková bomba			& 2--2--2	& 2--2--2		& 				& 				&V	\\		\hline
		
\end{tabular}


\subsection{Podrobný popis bloků}

Popis některých vlastností - má energetickou komponentu - > implikuje definici bindovacích bodů
má kyslíkovou komponentu - implikuje TotalObjectOxygen

Producer nebo Consumer implikuje Total object energy

Controllable implikuje IsController nebo IsControllable



\subsubsection{A1 - Blok základny}
- velikost v ose Z omezena na 4 základní bloky

- má elektriku

Pokud bychom měli nerovný terén, tento blok by mohl zahrnovat podstavce pro vyrovnání terénu.

\subsubsection{A2 - Blok stavby}
- všechny velikosti

- má elektriku

Tento blok je základním stavebním blokem ve hře.

\subsubsection{A3 - Blok polykarbonátu}
- všechny velikosti
Tento blok je nejlevnější, není připojen do elektrické sítě. Ideou bloku je podpora průhledných stěn a také možné pomocné stavební konstrukce pro výstavbu do výšky. Inspiraci můžeme vidět v používání třeba bloku hlíny ve hře \MC{}, kdy hráč vyskočí a pod sebe umístí nový blok a tím se ve světě posune o 1 metr výš.

\subsubsection{A4 - Zkosený blok základny}
- velikost v ose Z omezena na 4 základní bloky

- má elektriku

Stejné jako blok \textit{A1}, jen je zkosený. Může sloužit jako přístupová rampa.

\subsubsection{A5 - Zkosený blok stavby}
- všechny velikosti

- má elektriku
\subsubsection{A6 - Roh bloku stavby}
-všechny velikosti

- má elektriku

\subsubsection{B1 - Terminál}
- speciální, pevná velikost 1 x 8 x 5 bloků

- má elektriku, konzument, rychlé doplnění energie, ovládání rozhraní, komplexní přehled připojené elektrické sítě.
\subsubsection{B2 - Napájené okno}
- minimální velikost 2 x 1 x 2, maximální velikost 20 x 1 x 20 základních bloků

- má elektriku, konzument
\subsubsection{B3 - Dveře}
- speciální, pevná velikost 7 x 7 x 11 bloků

- má elektriku, otevírání
\subsubsection{B4 - Světlo}
-velikost omezena na 1 x 1 x 1 blok

- má elektriku, konzument, ovládání bez přepínače
\subsubsection{B5 - Přepínač}
-velikost omezena na 1 x 1 x 1 blok

- má elektriku, náhled stavu
\subsubsection{B6 - Generátor energie}
- omezená velikost v ose Z na 2 bloky, jinak 3 x 3 až 20 x 20 v ostatních osách

- má elektriku, producent
\subsubsection{B7 - Generátor objektů}
- omezená velikost v ose Z na 2 bloky, jinak 3 x 3 až 20 x 20 v ostatních osách

- má elektriku, konzument
\subsubsection{B8 - Akumulátor}
- speciální, pevná velikost 3 x 3 x 3 bloků

- má elektriku, producent, konzument, rychlý náhled naplnění
\subsubsection{B9 - Plnička kyslíkových bomb}
- speciální, pevná velikost 4 x 3 x 4 bloků

- má elektriku, kyslíkovou komponentu, konzument, UI, rychlé doplnění kyslíku

- využijeme ideu náhledu inventáře a plnička bude zobrazovat blok B10, pokud bude nějaký takový blok plnit.

\subsubsection{B10 - Kyslíková bomba}
- speciální, pevná velikost 2 x 2 x 2 bloků

- má kyslíkovou komponentu, možnost sebrat, rychlý náhled naplnění, rychlé doplnění kyslíku


