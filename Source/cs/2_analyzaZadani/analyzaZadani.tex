%!TEX root = ../prace.tex

\chapter{Analýza zadání}

\section{Stávající implementace mechanismů}
- v dalším textu budeme vycházet z her: \\

Minecraft \\
Space Engineers / Medieval Engineers \\

\begin{itemize}
	\item popsat, jak je to v jednotlivých zmíněných hrách (musel jsem je nutně hrát všechny?)
	\item popsat velikosti bloků, nějaké zákonitosti, fyziku
	\item popsat strategické mechaniky
\end{itemize}

\section{Rozbor zadání}

Zde bychom měli popsat, co by se nám ve hře líbilo a stanovit reálnost implementace
\\

Nejspíše se text bude prolínat s kapitolou - dalším vývojem?\\
Měl bych si tu vysnit celou hru, nebo to spíše seškrtat?


\begin{itemize}
	\item tedy že bychom chtěli panďuláka, jaké pohledy
	\item a že bychom s ním chtěli chodit a stavět a bourat
	\item ale že nám taky může umřít - počasí, kyslík
	\item popsat svět, bloky, co by asi měly umět

\end{itemize}


\section{Cíle práce}