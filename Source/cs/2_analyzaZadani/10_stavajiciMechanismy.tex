%!TEX root = ../prace.tex


\section{Stávající implementace mechanismů}

V následujících podkapitolách si rozebereme jednotlivé části her a jak je implementují ostatní.

\subsection{Bloky}

Bloky, respektive jejich vizuální reprezentace, nemusí být vždy ve tvaru krychle.\\(TODO reference na MC, SE) Stále však budeme blok chápat jako objekt, který je umístěn v ortogonální mřížce 3D prostoru a beze zbytku tento prostor vyplňuje. Nebudeme se tedy zabývat multibloky (reference na ME)

různé druhy, velikosti, jejich vizuální reprezentace, rozšiřovatelnost, obecně co všechno by měly umět

\subsubsection{Základní vlastnosti}
Chceme navrhnout systém, ve kterém bude nejmenší blok o hraně 20 cm, tedy objemu odpovídající 0,008 m$^3$. Tento blok nazveme jako jednotkový. Největší blok pak omezíme na 20-ti násobek jednotkové krychle ve všech 3 rozměrech. Největší blok tedy bude mít objem 64 m$^3$. Může se stát, že dolní limit bude příliš malý, ale v tuto chvíli považujeme tuto konstantu za dostatečnou. Naopak horní limit bude nejspíše dostatečný - práce s příliš velkými bloky by mohla být neefektivní a stavba nepřehledná.

\subsubsection{Součásti bloků}
TODO zmínit MC má readstone, popsat základy chování.



-- TODO komponenty elektřiny, inventáře

\subsection{Komunikace bloků}

Chceme, aby bloky v elektrické síti spolu uměly komunikovat a bylo třeba možné vzdáleně tyto bloky ovládat. Obdobný systém je možné nalézt i ve hře \SE{}, kde jsou tlačítka pro ovládání různých dveří, pístů a dalších interaktivních bloků.

\subsection{Skládání bloků do struktur}
Chceme hráči umožnit postavení komplexní struktury bloků, která bude dohromady dávat nějaký speciální význam.V našem případě to bude konstruktor objektů, díky kterému za pomoci bloku \textit{B1} -- terminálu -- může hráč vytváře nové bloky, které pak bude moci umístit do světa. V našem pojetí to bude spíše objekt, který bude imaginárně vymýšlet optimální rozvržení bloku (de facto takový automatizovaný návrhář Blueprintů). Bloky jednou vymyšlené pak hráč bude moci stavět libovolně mnohokrát, jen musí mít dostatečnou zásobu energie pro jejich postavení.

\subsection{Zdraví bloků}
Chceme, aby bloky měly zdraví a aby bylo možné je zničit. Bloky v elektrické síti ale necháme se uzdravovat, což bude spotřebovávat energii. Protože očekáváme, že pouze bloky exponované na vnější straně budov budou předmětem uzdravování, dává nám smysl požadovat nějaký způsob přednostního uzdravování bloků, které budou s největší pravděpodobností nejdříve zničeny. Cílem je větší podpora exponovaných a tedy kriticky důležitých bloků. Oproti tomu pokud bude blok z větší části zastíněn nějakými jinými bloky, nebude jeho expozice vůči celkovému zdraví tak velká, že by hrozilo okamžité zničení.

\subsubsection{Speciality}
Multiblocks, náhled inventáře (\ME{} - stůl a jídlo), conveyor system SE, MC složité, ale dá se vyřešit módy, propagace kyslíku ME

\subsection{Herní svět}

jaký je herní svět

\subsubsection{Reprezentace}

MC - bloky, chuncks, SE + ME planety

\subsubsection{Bloky v herním světě}

do gridu, start-free grid


\subsubsection{Denní / noční cyklus}

obvykle tam je, MC 20minut. My zkusíme 30 minut (zkusili jsme 60 minut, ale ukázalo se to jako příliš dlouhá doba - brzy nebylo co dělat kvůli malé nabídce bloků).

\subsubsection{Herní překážky}

počasí, \NPC{}, atributy avataru

\subsubsection{(Ne)fyzikální chování}

MC - bloky stojí ve vzduchu, ale třeba písek při updatu začne padat

\subsection{Inventář}

mc pevné sloty, SE skupiny slotů.

neomezíme váhově ani jinak 

\subsection{Avatar hráče}
avatar má nějaké vlastnosti, \HUD{}, 1St / 3rd person view, zdraví, stamina, hlad, O2



