%!TEX root = ../prace.tex


\section{Stávající implementace mechanismů}

V následujících podkapitolách si rozebereme jednotlivé části her a jak je implementují ostatní.

\subsection{Bloky}

různé druhy, velikosti, jejich vizuální reprezentace, rozšiřovatelnost, obecně co všechno by měly umět

\subsubsection{Základní vlastnosti}
rozměry

\subsubsection{Součásti bloků}
komponenty elektřiny, inventáře

\subsubsection{Speciality}
Multiblocks, náhled inventáře (\ME{})

\subsection{Herní svět}

jaký je herní svě

\subsubsection{Reprezentace}

MC - bloky, chuncks

\subsubsection{Bloky v herním světě}

do gridu, start-free grid

\subsubsection{Denní / noční cyklus}

obvykle tam je

\subsubsection{Herní překážky}

počasí, \NPC{}, atributy avataru

\subsubsection{(Ne)fyzikální chování}

MC - bloky stojí ve vzduchu, ale třeba písek při updatu začne padat

\subsection{Inventář}

mc pevné sloty

\subsection{Avatar hráče}
avatar má nějaké vlastnosti, \HUD{}, 1St / 3rd person view, zdraví, stamina, hlad, O2



