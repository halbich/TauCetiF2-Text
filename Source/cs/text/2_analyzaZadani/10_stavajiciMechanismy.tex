%!TEX root = ../../prace.tex


\section{Stávající implementace mechanismů}

Rozebereme si, jakým způsobem je v současné době přistupováno k blokům a jak jsou tyto bloky umisťovány do herního světa. Dále nás budou zajímat ostatní herní mechaniky, jako například denní cyklus, nebo jakým způsobem je řešena herní postava, její inventář a možnosti hráče interakce se světem. Poté se rozhodneme, jakým způsobem budeme uvedené mechaniky řešit my.

\subsection{Bloky}

Pod pojmem \uv{blok} budeme chápat objekt, který je umístěn v~nějaké ortogonální mřížce 3D prostoru a~beze zbytku tento prostor vyplňuje. Bloky spolu sdílí společnou množinu vlastností a dále pak samy definují své vlastní vlastnosti, které vychází z povahy bloku, tedy toho, co daný blok reprezentuje. V této kapitole rozebereme tyto vlastnosti a následně definujeme, jaké problémy budeme muset v této práci řešit.


\subsubsection{Základní vlastnosti}

Mezi základní vlastnosti bloku bychom měli zařadit \textit{vizuální reprezentaci}, \textit{pozici ve světě}, \textit{rotaci} a \textit{velikost}. Tento výčet není kompletní, ale můžeme říct, že jsou pro hráče nejvíce viditelné. Podle typu hry, jejího stylu a celkové koncepce bychom pak mohli do této množiny přidat třeba \textit{zdraví bloku}. V této části se zaměříme právě na základní vlastnosti a rozšiřujícím vlastnostem se budeme věnovat v následujících částech této kapitoly.

Vizuální reprezentace zřejmě vychází z povahy bloku. Vždy ale platí, že se vždy musí vejít do nějakých hranic bloku, obvykle daných v rámci pravidelné mřížky, ve které se blok nachází. Může v sobě zahrnovat i nějaké informace, které jsou pak hráčem vnímány pohledem. Kupříkladu mohou různými způsoby indikovat vnitřní stav objektu, podobně jako třeba svítivé diody indikují různé stavy elektronických zařízení v reálném světě. Podrobněji se budeme vzhledem zabývat až budeme řešit konkrétní bloky v naší hře.

Pozice bloku v herním světě je u her z kapitoly \ref{chap:uvod} může být buď na jednoznačně definovaném místě v herním světě, přičemž všechny bloky mají vůči sobě stejnou rotaci (\MC{}), nebo je blok součástí nějaké skupiny\footnote{Za skupinu budeme považovat shluk alespoň jednoho bloku a všechny bloky ve skupině musí být ve stejné mřížce.} bloků s jednoznačnou pozicí v herním světě. Tato skupina pak může mít libovolnou rotaci vůči nějakému globálnímu souřadnicovému systému a můžeme říct, že tato skupina tvoří lokální ortogonální systém. Různé skupiny mohou být vůči sobě různě natočeny. Toto chování je možné pozorovat třeba ve hře \ME{}, kde postavením prvního bloku určíme pozici a rotaci této skupiny. Pokud budeme chtít přidat do světa další blok, do nějaké vzdálenosti od prvního bloku je tento nově stavěný blok stále přichytáván do mřížky definované prvním blokem a rotace bloku jsou vždy o 90~stupňů. Obvykle platí, že první blok je umístěn vodorovně (tedy krychle bude i na šikmém terénu umístěna tak, že její horní a dolní strana je ve vodorovné poloze) a je možné ho libovolně rotovat kolem vertikální osy.

Rotace bloků jsou u naprosté většiny her řešeny v zásadě podobně. Výjimkou je pouze \MC{}, který rotaci bloků neumožňuje. U většiny bloků to nevadí, protože jsou bloky umístěny po směru, ze kterého je blok umisťován. Můžeme však nalézt některé speciality, třeba blok \textit{kolejí} (oficiální popis bloku \citep{mc_rail}). Koleje v \MC{u} mohou být rovné (ve směru sever -- jih nebo východ -- západ), mohou být nakloněné a překonávat výškový rozdíl mezi dvěma bloky, nebo mohou tvořit zatáčku. Díky absenci rotací tak hra používá několik pravidel, které určují výslednou podobu bloku. Podrobněji se jimi však zabývat nemusíme. U zbývajících her je situace, kterou jsme popsali v úvodu této části -- první blok je natočen vodorovně, má libovolnou rotaci dle vertikální osy a tímto určuje přichytávací mřížku pro další stavěné bloky.

Hry mají velikost bloků vždy stejně velkou. \MC{} má hranu bloku o~délce $1\,\rm m$ (popis bloku na oficiální Wiki stránce Minecraftu \citep{mc_block}, oficiální popis jednotek použitých v Minecraftu \citep{mc_units}). \SE{} sice bloky hranově omezuje dle kategorií od $0,5\,\rm m$ do $2,5\,\rm m$ (oficiální popis bloků ve hře \citep{se_blocks_wiki}), ale tyto kategorie mezi sebou nelze kombinovat a je možné k sobě vázat pouze bloky z jedné a té samé kategorie. U ostatních her je situace podobná, byť některé jsou v raných fázích vývoje a tudíž pro ně neexistuje žádný oficiální zdroj informací, takže velikosti bloků bychom mohli pouze odhadovat. 

Jako další základní vlastnosti bloku bychom mohli brát třeba \textit{zdraví}, \textit{cenu za postavení}, \textit{čas doby stavby}, \textit{délku trvání destrukce} (pokud vychází z hráčovy akce), co hráč získá za zničení bloku apod. U doby konstrukce a destrukce bloku můžeme říct, že doba trvání dané akce závisí na parametrech bloku a použitém nástroji. Kupříkladu v \MC{u} trvá kutání bloku kamene \textit{diamantovým} krumpáčem podstatně rychleji, než při použití krumpáče \textit{dřevěného}. Taktéž rychlost opotřebení nástrojů je různá. Nicméně když se ve hrách přepneme do tzv.~\textit{kreativního} módu, pak máme stavbu bloků \uv{zadarmo} a ihned. Tedy nemusíme mít žádné komponenty ke stavbě, ani specializovaný nástroj (případ \SE{}, \ME{}), nebo nám při postavení bloku tyto bloky neubývají z inventáře (\MC{}). 

\subsubsection{Součásti bloků}
Jako součásti bloků můžeme brát cokoliv, co rozšiřuje základní vlastnosti. Zde bychom mohli zmínit třeba \textit{Redstone} v \MC{u}. Redstone je speciální typ bloku, chová se jako elektrický vodič a v kombinaci s jinými bloky je možné vytvářet logická hradla. Je zřejmé, že pokud vhodným způsobem zkombinujeme určitá hradla, je možné v \MC{u} vytvořit třeba bitovou sčítačku. Ačkoliv vytvoření jednoho hradla je snadné, složitější logické obvody (například kódový zámek) jsou pak velmi náročné na prostor. Nejen z tohoto důvodu pro \MC{} vnikly různé módy, rozšiřující základní funkcionalitu hry o nové elektrické komponenty. Pro zajímavost, mód \textit{RedPower} (blog autora \citep{eloraam}) má jednotlivá hradla jako samostatné bloky (což šetří místo) a navíc má možnost skládat různě barevné vodiče (což jsou pouze ekvivalenty Redstonových bloků) do sebe (až 16~linek signálu). Bez tohoto módu by hráč potřeboval takový prostor, aby položil 16~vedení Redstone tak, aby se nedotýkaly a vzájemně neovlivňovaly (na rovině by tyto vodiče zabíraly šířku 31~bloků). 

Další možné součásti bloků, kromě vedení elektřiny, může být například práce s kyslíkem či práce s inventářem. Některé bloky hráči nabízí nějaký úložný prostor, kam může přesunovat objekty ze svého inventáře. Později pak může k bloku přijít a objekty si opět navrátit do svého inventáře. 
Dále můžeme zmínit interakci s uživatelem, ať už přímou, nebo nepřímou. Jako přímou interakci budeme chápat takové použití bloku, kdy rovnou vidíme nějakou změnu. To může být například stisknutí tlačítka, změna polohy nějaké páky. Výsledek této přímé interakce pak hráč vidí okamžitě a vizuálně se blok nějakým způsobem změní. Jako nepřímou interakci bychom mohli uvést například otevření nějakého ovládacího rozhraní bloku, což je obvykle nějaká UI obrazovka. Obě interakce se mohou prolínat, takže výsledkem nepřímé interakce může být třeba změna barvy bloku.


\subsection{Komunikace bloků}
Bloky spolu mnohdy umí komunikovat. Dříve zmíněný Redstone z \MC{u} by se dal taktéž považovat za jistou metodu interakce mezi bloky. Například stiskem tlačítka lze změnit výstupní Redstonový signál z tohoto tlačítka (třeba z neaktivního na aktivní), který způsobí změnu polohy nějakého pístu. Ovšem komunikace může být i méně viditelná -- například z terminálu v \SE{} je možné ovládat písty, otevírat a zavírat dveře hangáru apod. bez explicitních vodičů. Ve hrách \MC{} a \SE{} můžeme nalézt příklad dopravníkových systémů. Ty slouží k přemístění bloků či objektů mezi dvěma či více body. Svým způsobem je pak přeprava v rámci těchto systémů také druhem meziblokové komunikace - bloky si mezi sebou předávají konkrétní instance objektů.

\subsection{Skládání bloků do struktur}
Asi jediné příklady, který můžeme zmínit, jsou ve hře \MC{} -- postavení portálu, sněhuláka či golema. (TODO obrázek tvarů?) V momentě, kdy nějaká skupina bloků splňuje přesně definovaný tvar, tak se vykoná nějaká pevně definovaná událost. Například je možné otevřít portál, nebo se bloky zničí a na místo nich se spawne NPCčko (takže jako by se to zrodilo z těch bloků) (TODO uhladit)

Skládání do nějakých komplexnějších tvarů vidíme jako potenciálně zajímavou herní vlastnost, obzvláště v kombinaci s různě velkými bloky. Budeme tedy toto téma chtít rozvinout a implementovat do naší práce. 

\subsubsection{Speciality}
Ve hrách \ME{} a  \NMS{} můžeme nalézt zajímavou funkcionalitu \textit{multibloků}. Tato funkcionalita nabízí například nahrazení nějaké stěny nějakého bloku oknem či nějakým dalším vizuálním či funkčním elementem. Tato funkcionalita se nám sice líbí, ale v tuto chvíli to bereme spíše jako druhořadou záležitost. Implementace této vlastnosti chápeme spíše jako \uv{\textit{Nice To Have}}, tedy pouze v případě, že na to budeme mít prostor a čas.

Další zajímavou specialitou je třeba náhled inventáře. Kupříkladu u hry \ME{} se jedná stůl a~jídlo na stole. Stůl má svůj inventář, do kterého je možné umisťovat jídlo. Stůl pak obsah tohoto inventáře zobrazuje tak, že jídlo má svůj náhled na talířích na stole, takže to vypadá, jako by bylo jídlo připraveno ke konzumaci. Opět to chápeme jako hezkou, ale spíše vizuální záležitost.

Určitě bychom také mohli zmínit přepravníkový systém. Tento systém umožňuje \uv{poslat} nějaké bloky na jiné místo, kupříkladu z jednoho křídla budovy do druhého. V \MC{u} bez módu se této funkcionality dá dosáhnout, nicméně je to velice nepraktické a pomalé. Ale opět se můžeme obrátit na módy, třeba na \textit{BuildCraft}, který umožňuje dopravovat bloky i na velké vzdálenosti. Hra \SE{} má tento systém už ve svém základu a slouží například pro dopravu natěženého materiálu od bloku Vrtáku do bloku Skladiště (který má nějaký svůj inventář). Tento dopravníkový systém v současné chvíli nevyužijeme a proto ho nebudeme řešit.

-- propagace kyslíku ME, TOM

\subsection{Herní svět}

Herní svět nám bude stačit jednoduchý, bez terénních nerovností. Stále se držíme premisy, že nás zajímá, zda je celkový koncept hry použitelný. (dle použitého přístupu můžeme ale nemusíme mít editovatelný terén za běhu hry)

\subsubsection{Reprezentace}

MC -- bloky, chuncks, SE + ME planety

\subsubsection{Bloky v~herním světě}

do gridu, start-free grid


\subsubsection{Denní / noční cyklus}

obvykle tam je, MC 20minut. My zkusíme 30 minut (zkusili jsme 60 minut, ale ukázalo se to jako příliš dlouhá doba -- brzy nebylo co dělat kvůli malé nabídce bloků).

\subsubsection{Herní překážky}

počasí, \NPC{}, atributy avataru

\subsubsection{(Ne)fyzikální chování}

MC -- bloky stojí ve vzduchu, ale třeba písek při updatu začne padat

\subsection{Inventář}

mc pevné sloty, SE skupiny slotů.

neomezíme váhově ani jinak 

\subsection{Avatar hráče}
avatar má nějaké vlastnosti, \HUD{}, 1St / 3rd person view, zdraví, stamina, hlad, O2



