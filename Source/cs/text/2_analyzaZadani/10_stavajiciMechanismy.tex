%!TEX root = ../../prace.tex


\section{Stávající implementace mechanismů}

V následujících podkapitolách si rozebereme jednotlivé části her a~jak je implementují ostatní. Pokusíme se rozebrat to, z čeho bychom mohli čerpat a co rozvinout dále.

\subsection{Bloky}

Hry mají obvykle velikost bloků stejně velkou. \MC{} má hranu bloku o~délce 1 metru (popis bloku na oficiální Wiki stránce Minecraftu \citep{mc_block}, oficiální popis jednotek použitých v Minecraftu \citep{mc_units}). \SE{} bloky hranově omezuje dle kategorií od 0.5\,\rm~m do 2.5\,\rm~m (oficiální popis bloků ve hře \citep{se_blocks_wiki}). U ostatních her je situace podobná, byť některé jsou v raných fázích vývoje a tudíž pro ně neexistuje žádný oficiální zdroj informací, takže velikosti bloků bychom mohli pouze odhadovat.

Vizuální reprezentace bloků nemusí být vždy jen tvaru krychle. Stále však budeme pod pojmem blok chápat objekt, který je umístěn v~ortogonální mřížce 3D prostoru a~beze zbytku tento prostor vyplňuje. Některé z námi zmíněných her umožňují volnou počáteční rotaci (kolem vertikální osy) a také nabízí nějaký způsob \uv{ukotvení} bloku do terénu, tedy prvotní umístění bloku. Z tohoto pohledu je pak možné chápat tento první umístěný blok jako pivot, od jehož umístění a rotace se odvíjí následné vlastnosti přichycovací mřížky. Obvykle však platí, že bloky jsou umístěny vodorovně (tedy krychle bude na šikmém terénu umístěna tak, že její horní a dolní strana je ve vodorovné poloze). Pro naše potřeby bude stačit, když budeme mít bloky zarovnané do jednotné mřížky v rámci celého světa (tedy tak, jak to má \MC{}).

\subsubsection{Základní vlastnosti}

Mezi základní vlastnosti bloku bychom měli zařadit vizuální reprezentaci, pozici ve světě, rotaci a velikost. Rotace krychle u jiných her je de facto zbytečná, protože tento konkrétní tvar bude vypadat vždy stejně. Ostatně třeba \MC{} žádné rotace bloků nenabízí. Ovšem v našem případě se budeme bavit obecně o kvádru a tam už rotace bloku dostávají svůj význam. Ne vždy bychom měli podporovat rotace ve všech osách -- když si představíme blok dveří, kdy v jednom bloku jsou futra a zároveň samotné křídlo dveří, tak u takového bloku dává smysl pouze rotace kolem vertikální osy. 

Jako další základní vlastnosti bychom mohli brát třeba zdraví, cenu za postavení, čas doby stavby, délku trvání destrukce (pokud vychází z hráčovy akce), co hráč získá za zničení bloku apod. Všechny námi zmiňované hry tyto vlastnosti nějakým způsobem implementují, takže se tohoto schématu budeme také nějakým základním způsobem držet. Například konkrétně u doby konstrukce a destrukce bloku můžeme říct, že doba trvání dané akce závisí na parametrech bloku a použitém nástroji. Například v \MC{u} trvá kutání bloku kamene \textit{diamantovým} krumpáčem podstatně rychleji, než při použití krumpáče \textit{dřevěného}. Taktéž rychlost opotřebení nástrojů je různá. Nicméně když se ve hrách přepneme do tzv.~\textit{kreativního} módu, pak máme stavbu bloků \uv{zadarmo} a ihned. Tedy nemusíme mít žádné komponenty ke stavbě, ani specializovaný nástroj (případ \SE{}, \ME{}), nebo nám při postavení bloku tyto bloky neubývají z inventáře (\MC{}). Tuto vlastnost bychom chtěli také nějakým způsobem implementovat, protože nám samotným tento mód usnadní vývoj hry -- nebudeme muset řešit speciální vývojové nastavení a zároveň nebudeme muset řešit herní problémy (třeba nedostatek nějakých surovin). 

\subsubsection{Součásti bloků}
Jako součásti bloků můžeme brát cokoliv, co rozšiřuje základní vlastnosti. Zde bychom mohli zmínit třeba \textit{Redstone} v \MC{u}. Redstone je speciální typ bloku, chová se jako elektrický vodič a v kombinaci s jinými bloky je možné vytvářet logická hradla. Je zřejmé, že pokud vhodným způsobem zkombinujeme určitá hradla, je možné v \MC{u} vytvořit třeba bitovou sčítačku. Ačkoliv vytvoření jednoho hradla je snadné, složitější logické obvody (například kódový zámek) jsou pak velmi náročné na prostor. Z toho důvodu pro \MC{} vnikly různé elektrické módy, rozšiřující základní funkcionalitu hry. Vytváření takového módu zde nebudeme řešit, ale můžeme čtenáři prozradit, že díky nim je možné přidávat nové bloky do hry a zároveň jim implementovat i poměrně složitou funkcionalitu. Pro zajímavost, mód \textit{RedPower} má jednotlivá hradla jako samostatné bloky (což šetří místo) a navíc má možnost skládat různě barevné vodiče (což jsou pouze ekvivalenty Redstonových bloků) do sebe (až 16 linek signálu). Bez tohoto módu by hráč potřeboval takový prostor, aby položil 16 vedení Redstone tak, aby se nedotýkaly a vzájemně neovlivňovaly. Celkově vidíme koncept elektrického vedení jako zajímavý i pro naši hru, takže obdobnou funkcionalitu bychom také chtěli mít.

Další možné součásti bloků, kromě vedení elektřiny, může být například práce s kyslíkem či práce s inventářem. Blok třeba může nabízet nějaký úložný prostor, kam může hráč přesunovat objekty ze svého inventáře. Později pak může k bloku přijít a objekty si opět navrátit do svého inventáře. 
Dále můžeme zmínit interakci s uživatelem, ať už přímou, nebo nepřímou. Jako přímou interakci budeme chápat takové použití bloku, kdy rovnou vidíme nějakou změnu. To může být například stisknutí tlačítka, změna polohy nějaké páky. Výsledek této přímé interakce pak hráč vidí okamžitě a vizuálně se blok nějakým způsobem změní. Jako nepřímou interakci bychom mohli uvést například otevření nějakého ovládacího rozhraní bloku, což je obvykle nějaká UI obrazovka. Obě interakce se mohou prolínat, takže výsledkem nepřímé interakce může být třeba změna barvy bloku.

V naší hře bychom byli rádi, aby měl uživatel dobrý pocit z toho, že se tam alespoň něco děje a není to pouze statický svět složený z různě velkých kostiček. Proto budeme chtít, abychom mohli s bloky maximálně interagovat ať už přímo či nepřímo.

\subsection{Komunikace bloků}
Bloky spolu mnohdy umí komunikovat. Dříve zmíněný Redstone z \MC{u} by se dal taktéž považovat za jistou metodu interakce mezi bloky. Například stiskem tlačítka lze změnit výstupní Redstonový signál z tohoto tlačítka (třeba z neaktivního na aktivní), který způsobí změnu polohy nějakého pístu. Ovšem komunikace může být i méně viditelná -- například z terminálu v \SE{} je možné ovládat písty, otevírat a zavírat dveře hangáru apod. Nějaký základ takovéto meziblokové komunikace bychom také chtěli implementovat.


\subsection{Skládání bloků do struktur}
Asi jediný příklad, který můžeme zmínit, je postavení portálu v \MC{u}, nebo postavení sněhuláka či golema. (TODO obrázek tvarů?) V momentě, kdy nějaká skupina bloků splňuje přesně definovaný tvar, tak se vykoná nějaká pevně definovaná událost. Například je možné otevřít portál, nebo se bloky zničí a na místo nich se spawne NPCčko (takže jako by se to zrodilo z těch bloků) (TODO uhladit)

Skládání do nějakých komplexnějších tvarů vidíme jako potenciálně zajímavou herní vlastnost, obzvláště v kombinaci s různě velkými bloky. Budeme tedy toto téma chtít rozvinout a implementovat do naší práce. 

\subsubsection{Speciality}
Ve hrách \ME{} a  \NMS{} můžeme nalézt zajímavou funkcionalitu \textit{multibloků}. Tato funkcionalita nabízí například nahrazení nějaké stěny nějakého bloku oknem či nějakým dalším vizuálním či funkčním elementem. Tato funkcionalita se nám sice líbí, ale v tuto chvíli to bereme spíše jako druhořadou záležitost. Implementace této vlastnosti chápeme spíše jako \uv{\textit{Nice To Have}}, tedy pouze v případě, že na to budeme mít prostor a čas.

Další zajímavou specialitou je třeba náhled inventáře. Kupříkladu u hry \ME{} se jedná stůl a~jídlo na stole. Stůl má svůj inventář, do kterého je možné umisťovat jídlo. Stůl pak obsah tohoto inventáře zobrazuje tak, že jídlo má svůj náhled na talířích na stole, takže to vypadá, jako by bylo jídlo připraveno ke konzumaci. Opět to chápeme jako hezkou, ale spíše vizuální záležitost.

Určitě bychom také mohli zmínit přepravníkový systém. Tento systém umožňuje \uv{poslat} nějaké bloky na jiné místo, kupříkladu z jednoho křídla budovy do druhého. V \MC{u} bez módu se této funkcionality dá dosáhnout, nicméně je to velice nepraktické a pomalé. Ale opět se můžeme obrátit na módy, třeba na \textit{BuildCraft}, který umožňuje dopravovat bloky i na velké vzdálenosti. Hra \SE{} má tento systém už ve svém základu a slouží například pro dopravu natěženého materiálu od bloku Vrtáku do bloku Skladiště (který má nějaký svůj inventář). Tento dopravníkový systém v současné chvíli nevyužijeme a proto ho nebudeme řešit.

-- propagace kyslíku ME, TOM

\subsection{Herní svět}

jaký je herní svět

\subsubsection{Reprezentace}

MC -- bloky, chuncks, SE + ME planety

\subsubsection{Bloky v~herním světě}

do gridu, start-free grid


\subsubsection{Denní / noční cyklus}

obvykle tam je, MC 20minut. My zkusíme 30 minut (zkusili jsme 60 minut, ale ukázalo se to jako příliš dlouhá doba -- brzy nebylo co dělat kvůli malé nabídce bloků).

\subsubsection{Herní překážky}

počasí, \NPC{}, atributy avataru

\subsubsection{(Ne)fyzikální chování}

MC -- bloky stojí ve vzduchu, ale třeba písek při updatu začne padat

\subsection{Inventář}

mc pevné sloty, SE skupiny slotů.

neomezíme váhově ani jinak 

\subsection{Avatar hráče}
avatar má nějaké vlastnosti, \HUD{}, 1St / 3rd person view, zdraví, stamina, hlad, O2



