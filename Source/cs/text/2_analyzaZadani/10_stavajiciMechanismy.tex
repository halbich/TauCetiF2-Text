%!TEX root = ../../prace.tex


\section{Stávající implementace mechanismů}

V následujících podkapitolách si rozebereme jednotlivé části her a~jak je implementují ostatní. Pokusíme se rozebrat to, z čeho bychom mohli čerpat a co rozvinout dále.

\subsection{Bloky}

Hry mají obvykle velikost bloků stejně velkou. \MC{} má hranu bloku o~délce 1 metru (popis bloku na oficiální Wiki stránce Minecraftu \citep{mc_block}, oficiální popis jednotek použitých v Minecraftu \citep{mc_units}). \SE{} bloky hranově omezuje dle kategorií od 0.5\,\rm m do 2.5\,\rm m (oficiální popis bloků ve hře \citep{se_blocks_wiki}). U ostatních her je situace podobná, byť některé jsou v raných fázích vývoje a tudíž pro ně neexistuje žádný oficiální zdroj informací, takže velikosti bychom mohli pouze odhadovat.

Vizuální reprezentace bloků nemusí být vždy jen tvaru krychle. Stále však budeme pod pojmem blok chápat objekt, který je umístěn v~ortogonální mřížce 3D prostoru a~beze zbytku tento prostor vyplňuje. Některé z námi zmíněných her umožňují volnou počáteční rotaci (kolem vertikální osy) a také nabízí nějaký způsob \uv{ukotvení} bloku do terénu, tedy prvotní umístění bloku. Z tohoto pohledu je pak možné chápat tento první umístěný blok jako pivot, od jehož umístění a rotace se odvíjí následné vlastnosti přichycovací mřížky. Obvykle však platí, že bloky jsou umístěny vodorovně (tedy krychle bude na šikmém terénu umístěna tak, že její horní a dolní strana je ve vodorovné poloze).

\subsubsection{Základní vlastnosti}

Mezi základní vlastnosti bloku bychom měli zařadit vizuální reprezentaci, pozici ve světě, rotaci a velikost. Rotace u krychle u jiných her je de facto zbytečná, protože tvar bude vypadat vždy stejně. Ovšem v našem případě se budeme bavit obecně o kvádru a tam už rotace bloku dostávají hlubší význam. Naopak když si představíme blok dveří, kdy v jednom bloku jsou futra a samotné křídlo dveří, tak u takového bloku dává smysl pouze rotace kolem vertikální osy. 

Jako další základní vlastnosti bychom mohli brát třeba zdraví, cenu za postavení, čas doby stavby, délka trvání destrukce (pokud je od hráče), co hráč získá od zničení bloku apod.

\subsubsection{Součásti bloků}
Jako součásti bloků můžeme brát cokoliv, co rozšiřuje základní vlastnosti. Zde bychom mohli zmínit třeba Redstone v \MC{u}. Redstone je speciální typ bloku, chová se jako elektrický obvod a v kombinaci s bloky a redstonovou loučí je možné vytvářet logická hradla. Je zřejmé, že v kombinaci s dalšími hradly je možné v \MC{u} vytvořit třeba bitovou sčítačku. Ačkoliv vytvoření jednoho hradla je snadné, složitější logické obvody (například kódový zámek) jsou pak velmi náročné na prostor. Z toho důvodu pro \MC{} vnikly různé elektrické módy, rozšiřující základní funkcionalitu hry. Vytváření takového módu zde nebudeme řešit, ale můžeme čtenáři prozradit, že díky nim je možné Přidávat nové bloky do hry a zároveň jim implementovat i poměrně složitou funkcionalitu. Pro zajímavost, mód RedPower má jednotlivá hradla jako samostatné bloky (což šetří místo) a navíc má možnost skládat různě barevné vodiče do sebe (až 16 linek signálu). Bez tohoto módu by hráč potřeboval takový prostor, aby položil 16 vedení redstone tak, aby se nedotýkaly a neovlivňovaly.  

Další možné součásti bloků, kromě vedení elektřiny, může být například práce s kyslíkem či práce s inventářem. Blok třeba může nabízet nějaký úložný prostor, kam může hráč přesunovat objekty ze svého inventáře. Dále můžeme zmínit interakci s uživatelem, ať už přímou, nebo nepřímou. Jako přímou interakci uveďme takové použití bloku, kdy rovnou vidíme nějakou změnu. To může být například stisknutí tlačítka, změna polohy nějaké páky. Výsledek této přímé interakce pak hráč vidí okamžitě a vizuálně se blok nějakým způsobem změní. Jako nepřímou interakci bychom mohli uvést například otevření nějakého ovládacího rozhraní bloku, což je obvykle nějaká UI obrazovka. Obě interakce se mohou prolínat, takže výsledkem nepřímé interakce může být třeba změna barvy bloku.

\subsection{Komunikace bloků}
Bloky spolu mnohdy umí komunikovat. Dříve zmíněný redstone z \MC{u} by se dal taktéž považovat za metodu interakce. Například stiskem tlačítka lze vyslat redstonový signál, který změní polohu nějakého pístu. Ovšem komunikace může být i méně viditelná -- například z terminálu v \SE{} je možné ovládat nějaké písty, je možné otevírat a zavírat dveře hangáru apod.


\subsection{Skládání bloků do struktur}
Asi jediný příklad, který můžeme zmínit, je postavení portálu v \MC{u}, nebo postavení sněhuláka či golema. (TODO obrázek tvarů?) V momentě, kdy nějaká skupina bloků splňuje přesně definovaný tvar, tak se vykoná nějaká událost. Například je možné otevřít portál, nebo se bloky zničí a na místo nich se spawne NPCčko (takže jako by se to zrodilo z těch bloků) 



\subsubsection{Speciality}
Multiblocks, náhled inventáře (\ME{} -- stůl a~jídlo), conveyor system SE, MC složité, ale dá se vyřešit módy, propagace kyslíku ME

\subsection{Herní svět}

jaký je herní svět

\subsubsection{Reprezentace}

MC -- bloky, chuncks, SE + ME planety

\subsubsection{Bloky v~herním světě}

do gridu, start-free grid


\subsubsection{Denní / noční cyklus}

obvykle tam je, MC 20minut. My zkusíme 30 minut (zkusili jsme 60 minut, ale ukázalo se to jako příliš dlouhá doba -- brzy nebylo co dělat kvůli malé nabídce bloků).

\subsubsection{Herní překážky}

počasí, \NPC{}, atributy avataru

\subsubsection{(Ne)fyzikální chování}

MC -- bloky stojí ve vzduchu, ale třeba písek při updatu začne padat

\subsection{Inventář}

mc pevné sloty, SE skupiny slotů.

neomezíme váhově ani jinak 

\subsection{Avatar hráče}
avatar má nějaké vlastnosti, \HUD{}, 1St / 3rd person view, zdraví, stamina, hlad, O2



