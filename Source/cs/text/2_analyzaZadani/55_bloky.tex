%!TEX root = ../../prace.tex

\subsection{Bloky}

Pro naše potřeby bude stačit, když budeme mít bloky zarovnané do jednotné mřížky v rámci celého světa (tedy tak, jak to má \MC{}).  Z toho vyplývá, že nebudeme požadovat implementaci volné počáteční rotace bloku (ve smyslu rotace kolem vertikální osy). Opět zdůrazňujeme, že budeme implementovat prototyp stavění různě velkých bloků a proto bude jednotná mřížka dostatečná. S tím souvisí další zjednodušení -- nechceme řešit terén ani jeho modifikace a s tím související řešení umisťování bloků. V této fázi si postačíme s pouhou rovinou.


Budeme chtít mít bloky, které nebudou pouze statické objekty ve hře. Bloky by měly mít svůj význam ve světě a mělo by jít s nimi interagovat. Navíc budeme chtít to, aby bloky byly vizuálně přitažlivé, což bude u některých bloků znamenat, že je budeme muset vymodelovat v nějakém 3D modelovacím programu. Autor však nějaké minimální znalosti v tomto oboru má, takže by to neměl být problém. \textit{Multiblokový} systém není potřeba implementovat. Vzhledem k dynamickému škálování bloků bude potřeba vyřešit škálování vzhledem k případným 3D modelům. 


Dále budeme chtít, aby bylo možné definovat jednotlivé vlastnosti bloků jednoduchým a přímočarým způsobem, nejlépe mimo samotný zdrojový kód hry a v nějakém editoru. Tento požadavek je zde z toho důvodu, že je zcela běžné, že herní designéři ladí různé konstanty a nastavení během vývoje tak, aby hra co nejvíce odpovídala jejich představám a byla pro hráče zábavná. Mít tedy tyto konstanty pevně zakompilované v kódu by znamenalo opětovnou kompilaci celého projektu, což v pozdějších fázích může znamenat výrazné zpomalení vývoje. Protože očekáváme další vývoj této hry, měli bychom se držet nějakých rozumných postupů na udržovatelnost kódu a celého vývoje hry.

\subsubsection{Vlastnosti}

Chceme navrhnout systém, ve kterém bude nejmenší blok o~hraně 20 cm, tedy objemu odpovídající 0,008 m$^3$. Tento blok nazveme jako jednotkový, identisky s tímto budeme používat pojem \textit{jednotková krychle}. Největší blok pak omezíme na 20-ti násobek jednotkové krychle ve všech 3 rozměrech. Největší blok tedy bude mít objem 64 m$^3$. Může se stát, že dolní limit bude příliš malý, ale v~tuto chvíli považujeme tuto konstantu za dostatečnou. Naopak horní limit bude nejspíše dostatečný -- práce s~příliš velkými bloky by mohla být neefektivní a~stavba nepřehledná.

V následující tabulce si popíšeme, jaké bloky budeme ve hře chtít mít, jak by měly vypadat a co by měly umět. Bloky jsme rozdělili do dvou kategorií -- první jsou \textit{konstrukční} prvky, druhá kategorie obsahuje \textit{funkční} prvky. \textit{Název} je zřejmý. Sloupce \textit{Min} a \textit{Max} popisují velikost bloku jako násobky jednotkové krychle. Dále jsou v tabulce sloupce \textit{P} a \textit{R}\footnote{Písmenka vychází z anglického \textit{Pitch} a \textit{Roll}. }. Ty popisují to, zda je blok možné rotovat podle daných os. Správně bychom tu měli mít i sloupec \textit{Y} (z anglického \textit{Yawn}, ale protože je pro všechny bloky povolen, tak jsme tento sloupec vynechali. 


Jako poslední je sloupec \textit{T}, tedy typ bloku. Typ bloku může být: \textit{Kostka}, \textit{Zkosený}, \textit{Rohový}, \textit{Vlastní}. Typ pak ovlivňuje další vlastnosti bloku, jako například cenu za jeho postavení, ale i jeho zdraví. (TODO podrobný popis algoritmu; , K = 1, Z = 0.5, R = 1/6, V = 1 (není v~potaz objem) )


\begin{tabular}{|rll*{5}{c}|}
	\hline
	\tableColumnTitles{Název}								{	&	Min		&	Max			&	P			&	R			&T	}		\hline
	\currentCategory{\textbf{Základní bloky}} 																					\\		\hline
		\mytablerow 				& Blok základny				& 1--1--4	& 20--20--4		& 				& 				&K	\\		\hline
		\mytablerow 				& Blok stavby				& 1--1--1	& 20--20--20	& \checkmark	& \checkmark	&K	\\		\hline
		\mytablerow 				& Blok polykarbonátu		& 1--1--1	& 20--20--20	& \checkmark	& \checkmark	&K	\\		\hline
		\mytablerow 				& Zkosený blok základny		& 1--1--4	& 20--20--4		& 				& 				&Z	\\		\hline
		\mytablerow 				& Zkosený blok stavby		& 1--1--1	& 20--20--20	& \checkmark	& \checkmark	&Z	\\		\hline
		\mytablerow 				& Roh bloku stavby			& 1--1--1	& 20--20--20	& \checkmark	& \checkmark	&R	\\		\hline
	\currentCategory{\textbf{Speciální bloky}} 									 												\\		\hline
		\mytablerow 				& Terminál			 		& 1--8--5 	& 1--8--5		& 				& 				&V	\\		\hline
		\mytablerow 				& Napájené okno				& 2--1--2	& 20--1--20		& \checkmark	& \checkmark	&K	\\		\hline
		\mytablerow 				& Dveře 					& 7--7--11	& 7--7--11		& 				& 				&V	\\		\hline
		\mytablerow 				& Světlo					& 1--1--1	& 1--1--1		& \checkmark	& \checkmark	&V	\\		\hline
		\mytablerow 				& Přepínač 					& 1--1--1	& 1--1--1		& \checkmark	& \checkmark	&V	\\		\hline
		\mytablerow 				& Generátor energie			& 3--3--2	& 20--20--2		& 				& 				&K	\\		\hline
		\mytablerow 				& Generátor objektů 		& 3--3--2	& 20--20--2		& 				& 				&K	\\		\hline
		\mytablerow 				& Akumulátor				& 3--3--3	& 3--3--3		& 				& 				&V	\\		\hline
		\mytablerow 				& Plnička kyslíkových bomb 	& 4--3--4	& 4--3--4		& 				& 				&V	\\		\hline
		\mytablerow 				& Kyslíková bomba			& 2--2--2	& 2--2--2		& 				& 				&V	\\		\hline
		
\end{tabular}


\subsection{Podrobný popis bloků}

V následujících odstavcích si podrobně popíšeme všechny bloky  a nastíníme jejich význam. Některé bloky mohou obsahovat energetické a kyslíkové \textit{komponenty}, jejichž význam bude zmíněn o pár odstavců dále (TODO link!).



\subsubsection{A1 -- Blok základny}
\label{blocks:A1}
Tento blok je ve hře z toho důvodu, protože má sloužit jako základový blok staveb. Pokud bychom měli nerovný terén, tento blok by mohl zahrnovat podstavce pro vyrovnání terénu. Velikost v~ose Z omezena na 4 základní bloky. Obsahuje \textit{energetickou komponentu}.


\subsubsection{A2 -- Blok stavby}
\label{blocks:A2}
Tento blok je základním stavebním blokem ve hře a proto může existovat ve všech možných velikostech. Obsahuje \textit{energetickou komponentu}.


\subsubsection{A3 -- Blok polykarbonátu}
\label{blocks:A3}
Tento blok je nejlevnější, neobsahuje žádné komponenty. Ideou bloku je podpora průhledných stěn a~také možné pomocné stavební konstrukce pro výstavbu do výšky. Inspiraci můžeme vidět v~používání třeba bloku hlíny ve hře \MC{}, kdy hráč vyskočí a~pod sebe umístí nový blok a~tím se ve světě posune o~1 metr výš.


\subsubsection{A4 -- Zkosený blok základny}
\label{blocks:A4}
Tento blok je zde ze stejného důvodu jako blok \nameref{blocks:A1}. Má stejné vlastnosti, pouze má zkosený tvar. Může sloužit jako přístupová rampa.


\subsubsection{A5 -- Zkosený blok stavby}
\label{blocks:A5}
Obdobně jako u základny, tento blok má stejné vlastnosti jako \nameref{blocks:A2}, pouze má jiná tvar.


\subsubsection{A6 -- Roh bloku stavby}
\label{blocks:A6}
Tento blok má stejné vlastnosti jako \nameref{blocks:A2} a opět definuje pouze jiný tvar.


\subsubsection{B1 -- Terminál}
\label{blocks:B1}
Terminál má pevnou velikost (1 x 8 x 5 bloků) a měl by vypadat jako nějaká obrazovka budoucnosti.  Obsahuje \textit{energetickou komponentu}. Terminál by pro hráče měl být použitelný přímo i nepřímo. Přímé použití by mělo dobít herní postavě energii, nepřímé by mělo otevřít uživatelské rozhraní. Co přesně by mělo být obsahem tohoto rozhraní bude řešeno v podrobné analýze (TODO link).

\subsubsection{B2 -- Napájené okno}
\label{blocks:B2}
Napájené okno je omezeno pouze v jednom rozměru, jinak povolíme velikost od 2 do 20 násobku jednotkové krychle. Dolní omezení je zde z toho důvodu, že okno o velikosti 20 x 20 cm je zbytečnš malé
- minimální velikost 2 x 1 x 2, maximální velikost 20 x 1 x 20 základních bloků

- má elektriku, konzument


\subsubsection{B3 -- Dveře}
\label{blocks:B3}
- speciální, pevná velikost 7 x 7 x 11 bloků

- má elektriku, otevírání


\subsubsection{B4 -- Světlo}
\label{blocks:B4}
-velikost omezena na 1 x 1 x 1 blok

- má elektriku, konzument, ovládání bez přepínače


\subsubsection{B5 -- Přepínač}
\label{blocks:B5}
-velikost omezena na 1 x 1 x 1 blok

- má elektriku, náhled stavu


\subsubsection{B6 -- Generátor energie}
\label{blocks:B6}
- omezená velikost v~ose Z na 2 bloky, jinak 3 x 3 až 20 x 20 v~ostatních osách

- má elektriku, producent


\subsubsection{B7 -- Generátor objektů}
\label{blocks:B7}
- omezená velikost v~ose Z na 2 bloky, jinak 3 x 3 až 20 x 20 v~ostatních osách

- má elektriku, konzument


\subsubsection{B8 -- Akumulátor}
\label{blocks:B8}
- speciální, pevná velikost 3 x 3 x 3 bloků

- má elektriku, producent, konzument, rychlý náhled naplnění


\subsubsection{B9 -- Plnička kyslíkových bomb}
\label{blocks:B9}
- speciální, pevná velikost 4 x 3 x 4 bloků

- má elektriku, kyslíkovou komponentu, konzument, UI, rychlé doplnění kyslíku

- využijeme ideu náhledu inventáře a~plnička bude zobrazovat blok B10, pokud bude nějaký takový blok plnit.


\subsubsection{B10 -- Kyslíková bomba}
\label{blocks:B10}
- speciální, pevná velikost 2 x 2 x 2 bloků

- má kyslíkovou komponentu, možnost sebrat, rychlý náhled naplnění, rychlé doplnění kyslíku


