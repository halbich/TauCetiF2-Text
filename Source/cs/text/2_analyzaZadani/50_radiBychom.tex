%!TEX root = ../../prace.tex



\section{Co bychom chtěli implementovat}

V následujících podkapitolách si rozebereme naše požadavky na hru



creative mode -- Tuto vlastnost bychom chtěli také nějakým způsobem implementovat, protože nám samotným tento mód usnadní vývoj hry -- nebudeme muset řešit speciální vývojové nastavení a zároveň nebudeme muset řešit herní problémy (třeba nedostatek nějakých surovin). 



V naší hře bychom byli rádi, aby měl uživatel dobrý pocit z toho, že se tam alespoň něco děje a není to pouze statický svět složený z různě velkých kostiček. Proto budeme chtít, abychom mohli s bloky maximálně interagovat ať už přímo či nepřímo. (TODO!! 05:55)


Nějaký základ takovéto meziblokové komunikace bychom také chtěli implementovat. (vodiče, ovládání )



Skládání do nějakých komplexnějších tvarů vidíme jako potenciálně zajímavou herní vlastnost, obzvláště v kombinaci s různě velkými bloky. Budeme tedy toto téma chtít rozvinout a implementovat do naší práce. 

(multibloky Tato funkcionalita se nám sice líbí, ale v tuto chvíli to bereme spíše jako druhořadou záležitost. Implementace této vlastnosti chápeme spíše jako \uv{\textit{Nice To Have}}, tedy pouze v případě, že na to budeme mít prostor a čas.)


Herní svět nám bude stačit jednoduchý, bez terénních nerovností. Stále se držíme premisy, že nás zajímá, zda je celkový koncept hry použitelný. (dle použitého přístupu můžeme ale nemusíme mít editovatelný terén za běhu hry)

%!TEX root = ../../prace.tex

\subsection{Bloky}

TODO  Většina bloků je stejně velká a má hranu o délce 1 metru \citep{mc_block}, \citep{mc_units}. (TODO přesunout do detailní analýzy. TODO popisek odkazu)

V současné době jsou velikosti bloků omezeny na konstantné velikost. Ve hře Minecraft je blok hranově omezen na 1m, hra Space Engineers bloky omezuje dle kategorií od 0.5\,\rm m do 2.5\,\rm m \citep{se_blocks_wiki}.


různé druhy, velikosti, jejich vizuální reprezentace, rozšiřovatelnost, obecně co všechno by měly umět.

Název - Min - Max - Pitch - Roll - Type (kostka, zkosený, roh, vlastní)

Tam kde Min == Max -> Vlastní škálování

Typ ovlivňuje další chování

Třeba u Světla by typ mohl být i K a hra by se chovala stejně, K = 1, Z = 0.5, R = 1/6, V = 1 (není v potaz objem)

komponenty bloků a nějaké další ptákoviny

\begin{tabular}{|rll*{5}{c}|}
	\hline
	\tableColumnTitles{Název}								{	&	Min		&	Max			&	P			&	R			&T	}		\hline
	\currentCategory{\textbf{Základní bloky}} 																					\\		\hline
		\mytablerow 				& Blok základny				& 1--1--4	& 20--20--4		& 				& 				&K	\\		\hline
		\mytablerow 				& Blok stavby				& 1--1--1	& 20--20--20	& \checkmark	& \checkmark	&K	\\		\hline
		\mytablerow 				& Blok polykarbonátu		& 1--1--1	& 20--20--20	& \checkmark	& \checkmark	&K	\\		\hline
		\mytablerow 				& Zkosený blok základny		& 1--1--4	& 20--20--4		& 				& 				&Z	\\		\hline
		\mytablerow 				& Zkosený blok stavby		& 1--1--1	& 20--20--20	& \checkmark	& \checkmark	&Z	\\		\hline
		\mytablerow 				& Roh bloku stavby			& 1--1--1	& 20--20--20	& \checkmark	& \checkmark	&R	\\		\hline
	\currentCategory{\textbf{Speciální bloky}} 									 												\\		\hline
		\mytablerow 				& Terminál			 		& 1--8--5 	& 1--8--5		& 				& 				&V	\\		\hline
		\mytablerow 				& Napájené okno				& 2--1--2	& 20--1--20		& \checkmark	& \checkmark	&K	\\		\hline
		\mytablerow 				& Dveře 					& 7--7--11	& 7--7--11		& 				& 				&V	\\		\hline
		\mytablerow 				& Světlo					& 1--1--1	& 1--1--1		& \checkmark	& \checkmark	&V	\\		\hline
		\mytablerow 				& Přepínač 					& 1--1--1	& 1--1--1		& \checkmark	& \checkmark	&V	\\		\hline
		\mytablerow 				& Generátor energie			& 3--3--2	& 20--20--2		& 				& 				&K	\\		\hline
		\mytablerow 				& Generátor objektů 		& 3--3--2	& 20--20--2		& 				& 				&K	\\		\hline
		\mytablerow 				& Akumulátor				& 3--3--3	& 3--3--3		& 				& 				&V	\\		\hline
		\mytablerow 				& Plnička kyslíkových bomb 	& 4--3--4	& 4--3--4		& 				& 				&V	\\		\hline
		\mytablerow 				& Kyslíková bomba			& 2--2--2	& 2--2--2		& 				& 				&V	\\		\hline
		
\end{tabular}


\subsection{Podrobný popis bloků}

Popis některých vlastností - má energetickou komponentu - > implikuje definici bindovacích bodů
má kyslíkovou komponentu - implikuje TotalObjectOxygen

Producer nebo Consumer implikuje Total object energy

Controllable implikuje IsController nebo IsControllable



\subsubsection{A1 - Blok základny}
- velikost v ose Z omezena na 4 základní bloky

- má elektriku

Pokud bychom měli nerovný terén, tento blok by mohl zahrnovat podstavce pro vyrovnání terénu.

\subsubsection{A2 - Blok stavby}
- všechny velikosti

- má elektriku

Tento blok je základním stavebním blokem ve hře.

\subsubsection{A3 - Blok polykarbonátu}
- všechny velikosti
Tento blok je nejlevnější, není připojen do elektrické sítě. Ideou bloku je podpora průhledných stěn a také možné pomocné stavební konstrukce pro výstavbu do výšky. Inspiraci můžeme vidět v používání třeba bloku hlíny ve hře \MC{}, kdy hráč vyskočí a pod sebe umístí nový blok a tím se ve světě posune o 1 metr výš.

\subsubsection{A4 - Zkosený blok základny}
- velikost v ose Z omezena na 4 základní bloky

- má elektriku

Stejné jako blok \textit{A1}, jen je zkosený. Může sloužit jako přístupová rampa.

\subsubsection{A5 - Zkosený blok stavby}
- všechny velikosti

- má elektriku
\subsubsection{A6 - Roh bloku stavby}
-všechny velikosti

- má elektriku

\subsubsection{B1 - Terminál}
- speciální, pevná velikost 1 x 8 x 5 bloků

- má elektriku, konzument, rychlé doplnění energie, ovládání rozhraní, komplexní přehled připojené elektrické sítě.
\subsubsection{B2 - Napájené okno}
- minimální velikost 2 x 1 x 2, maximální velikost 20 x 1 x 20 základních bloků

- má elektriku, konzument
\subsubsection{B3 - Dveře}
- speciální, pevná velikost 7 x 7 x 11 bloků

- má elektriku, otevírání
\subsubsection{B4 - Světlo}
-velikost omezena na 1 x 1 x 1 blok

- má elektriku, konzument, ovládání bez přepínače
\subsubsection{B5 - Přepínač}
-velikost omezena na 1 x 1 x 1 blok

- má elektriku, náhled stavu
\subsubsection{B6 - Generátor energie}
- omezená velikost v ose Z na 2 bloky, jinak 3 x 3 až 20 x 20 v ostatních osách

- má elektriku, producent
\subsubsection{B7 - Generátor objektů}
- omezená velikost v ose Z na 2 bloky, jinak 3 x 3 až 20 x 20 v ostatních osách

- má elektriku, konzument
\subsubsection{B8 - Akumulátor}
- speciální, pevná velikost 3 x 3 x 3 bloků

- má elektriku, producent, konzument, rychlý náhled naplnění
\subsubsection{B9 - Plnička kyslíkových bomb}
- speciální, pevná velikost 4 x 3 x 4 bloků

- má elektriku, kyslíkovou komponentu, konzument, UI, rychlé doplnění kyslíku

- využijeme ideu náhledu inventáře a plnička bude zobrazovat blok B10, pokud bude nějaký takový blok plnit.

\subsubsection{B10 - Kyslíková bomba}
- speciální, pevná velikost 2 x 2 x 2 bloků

- má kyslíkovou komponentu, možnost sebrat, rychlý náhled naplnění, rychlé doplnění kyslíku





\subsection{Komunikace bloků}

Chceme, aby bloky v~elektrické síti spolu uměly komunikovat a~bylo třeba možné vzdáleně tyto bloky ovládat. Obdobný systém je možné nalézt i~ve hře \SE{}, kde jsou tlačítka pro ovládání různých dveří, pístů a~dalších interaktivních bloků.


Celkově vidíme koncept elektrického vedení jako zajímavý i pro naši hru, takže obdobnou funkcionalitu bychom také chtěli mít.

\subsection{Skládání bloků do struktur}

Chceme hráči umožnit postavení komplexní struktury bloků, která bude dohromady dávat nějaký speciální význam.V našem případě to bude konstruktor objektů, díky kterému za pomoci bloku \textit{B1} -- terminálu -- může hráč vytváře nové bloky, které pak bude moci umístit do světa. V našem pojetí to bude spíše objekt, který bude imaginárně vymýšlet optimální rozvržení bloku (de facto takový automatizovaný návrhář Blueprintů). Bloky jednou vymyšlené pak hráč bude moci stavět libovolně mnohokrát, jen musí mít dostatečnou zásobu energie pro jejich postavení.

\subsection{Zdraví bloků}
Chceme, aby bloky měly zdraví a~aby bylo možné je zničit. Bloky v~elektrické síti ale necháme se uzdravovat, což bude spotřebovávat energii. Protože očekáváme, že pouze bloky exponované na vnější straně budov budou předmětem uzdravování, dává nám smysl požadovat nějaký způsob přednostního uzdravování bloků, které budou s~největší pravděpodobností nejdříve zničeny. Cílem je větší podpora exponovaných a~tedy kriticky důležitých bloků. Oproti tomu pokud bude blok z~větší části zastíněn nějakými jinými bloky, nebude jeho expozice vůči celkovému zdraví tak velká, že by hrozilo okamžité zničení.

\subsection{Herní svět}

jaký chceme herní svět

\subsubsection{Reprezentace}

bude nám stačit nějaký tree, definovat rozměry, na chuncky kašlem

\subsubsection{Bloky v~herním světě}

do gridu

\subsubsection{Denní / noční cyklus}
dáme ho

obvykle tam je, MC 20minut. My zkusíme 30 minut (zkusili jsme 60 minut, ale ukázalo se to jako příliš dlouhá doba -- brzy nebylo co dělat kvůli malé nabídce bloků).



\subsubsection{Herní překážky}

počasí, , atributy avataru

\subsubsection{(Ne)fyzikální chování}

nebudeme hrotit

\subsection{Inventář}

chceme volné sloty, rozšiřitelnost

\subsection{Avatar hráče}
avatar má nějaké vlastnosti, \HUD{}, 1St / 3rd person view, zdraví, O2, energie



