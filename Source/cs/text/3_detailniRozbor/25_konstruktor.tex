%!TEX root = ../../prace.tex

\section{Konstruktor objektů}
\label{sec:konstruktor}


V části \ref{subsec:adrt} jsme zjistili, že rozpoznávání tvarů je v~našem prostředí velice komplexní problém. Rádi bychom však využili některých myšlenek a~převedli je do realizovatelnější podoby.

Jediný tvar, který budeme chtít umět rozpoznat, bude tvar objektu \textit{Konstruktoru objektů}. Abychom měli snazší začátek ověřování splňujícího tvaru, řekneme, že konstruktor musí obsahovat speciální blok \RB{B7}. Chceme, aby konstruktor objektů vypadal následovně:

\begin{enumerate}
	\item Bloky \RB{B7} tvoří homogenní kvádr. Tento kvádr má čtvercovou podstavu, jejíž hrana je délky nejvýše $20$-ti násobku jednotkové krychle. Pro účely dalšího textu si označme délku hrany jako \textit{A}.
	\item Nad bloky \RB{B7} je volný prostor až do výše \textit{A}. Tento volný prostor bude reprezentovat místo pro vytváření bloků až do velikosti \textit{A}, a~proto musí být celý volný. Tento prostor označme jako \textbf{VP}.
	\item Okolo \textbf{VP} jsou těsně umístěny bloky \RB{A3}. Tyto bloky musí mít navíc netriviální dotyk s~boky bloků \RB{B7}. Vzhledem k~výšce bloku \textit{B7} můžeme ekvivalentně říct, že horní polovina boční stěny musí být \uv{překryta} bloky \RB{A3}. Cílem tohoto pravidla je snaha o~\uv{uzavření} konstruktoru průhlednými bloky.
\end{enumerate}

Celkové rozměry \textit{Konstruktoru objektů} jsou minimálně\linebreak $[(A+2), (A+2), (A+3)]$. Dále budeme požadovat, aby bylo možné používat \textit{Konstruktor objektů} skrze UI rozhraní bloku \RB{B1}.

Všechny tyto požadavky můžeme splnit tak, že budeme blok \RB{B7} v~rámci herního světa sdružovat do skupin, přičemž ve skupině spolu musí bloky Generátoru přímo sousedit. Ověřování validity bodů 2 a~3 může začít až v~momentě, kdy skupina bude splňovat bod 1 našich požadavků. Pokud jsou všechny body splněny, blok \RB{B1} může začít tento objekt zobrazovat v~rámci UI rozhraní.

