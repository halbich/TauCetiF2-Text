%!TEX root = ../../prace.tex

\section{Doplňující vlastnosti}



\subsection{Lokalizace}

Pro lokalizaci použijeme již implementovaný systém lokalizace, který nám \UE{} nabízí. Z~toho vyplývá fakt, že všechny textové řetězce, které budou předmětem lokalizace, budou datového typu \TT{FText}. To nám práci nijak nekomplikuje, ale je dobré s~touto vlastností počítat při návrhu konkrétních datových struktur.

\subsection{Hudba}

Pro atmosférický hudební doprovod nám bude stačit základní funkcionalita. Chceme, abychom mohli definovat sadu skladeb, které budeme náhodně přehrávat. Mezi skladbami pak budeme chtít mít nějaký náhodný interval, kdy nic nehraje. Stejným způsobem je hudební doprovod řešen třeba i~ve hře \MC{} a~nevidíme důvod, proč bychom se tohoto řešení nemohli držet. Z~předchozího tedy vyplývá, že nám bude stačit použít třídu \TT{AmbientSoundActor}, jejíž nabízená funkcionalita je pro nás dostačující.
