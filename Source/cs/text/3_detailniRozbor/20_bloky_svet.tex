%!TEX root = ../../prace.tex

\section{Bloky v~herním světě}
\label{sec:blocksWorld}


- je více mmožností. Uchování pole 50000 x 50000 x 25000 // todo ověřit
je nesmysl. 

- nepotřebujeme otevřený svět bez mřížky (pozdější aktualizace ME, jinak SE), takže budeme hledat nějakou variantu stromové struktury

- nabízí se možnost clustorování budov a~shlukování do skupin, s~následnou optimalizací počtu hladin

- my jsme zvolili K-D strom kombinovatný s~AABB. (proč? )

- náš strom má optimalizaci jedinného potomka, v~případě potřeby se dogeneruje do úrovně níže, případně rozpadne na podčásti a~rekurzivně se přidá.

- díky této variantě se můžeme snadno dotazovat na sousedy, což je hlavní cíl (proto)



\subsection{Zapojení do rozpoznávání tvarů}

- generátor bloků

jaké byly možnosti -- abecné rozpoznávání (původní implementace, rozvést nutnost rozpadu tvarů na menší (slope) + doplnění kvádry

