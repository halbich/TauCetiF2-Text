%!TEX root = ../../prace.tex

\section{Herní engine}
V první řadě bychom se měli zamyslet nad tím, jaký nástroj pro vývoj hry použijeme. Díky tomu budeme moct počítat s~možnostmi a~omezeními danými touto volbou. Shrňme si, co budeme ve hře potřebovat:

\begin{itemize}
	\item Renderování 3D objektů, pokročilé možnosti texturování
	\item Podpora I/O pro práci se savy
	\item Podpora UI
	\item Správa assetů
	\item Správa scény
\end{itemize}

Plusem pak bude:

\begin{itemize}
	\item Podpora zvuků
	\item Snadná implementace lokalizace
\end{itemize}

Pro další případný rozvoj bychom potřebovali:

\begin{itemize}
	\item Podpora pathfindingu
	\item Podpora síťové hry
	\item Podpora AI
\end{itemize}

Cílová platformy pro nás bude PC s~OS Windows. Pokud se rozhodneme pro již existující herní engine, který bude navíc podporovat multiplatformní vývoj, bude to pro nás, i~s~ohledem na další vývoj, plus. Dále bychom chtěli, abychom minimalizovali finanční náklady na vývoj. Z toho vyplývá, že placené řešení bychom byli ochotni použít pouze pokud by to pro nás znamenalo naprostou výhodu oproti jiným, neplaceným řešením.

Dalším kritériem je volba programovacího jazyka. Ta vychází z~autorových znalostí. Budeme tedy preferovat primárně jazyk \CS{}, který známe nejlépe. Pokud to bude nezbytně nutné, nebudeme se bránit ani jazyku \CPP{}, který je v~herní branži dlouho zavedený a~je stále hojně využívaný. Ačkoliv zkušenost s~tímto programovacím jazykem máme minimální, můžeme se tímto způsobem naučit novým dovednostem.

\pagebreak
Jak můžeme naši hru implementovat?
\begin{itemize}
	\item Implementace kompletního vlastního enginu
	\item Použit existující grafické knihovny a~nad tím implementovat vlastní engine
	\item Použit existující herní engine
\end{itemize}

Možných použitých enginů a~frameworků je opravdu mnoho. Pro jejich seznam se můžeme podívat do databáze herních enginů na stránce Devmaster. Jen zde je možné nalézt 236 možných řešení našeho problému volby herního enginu~\citep{engines_list}. Všechny záznamy jsme omezili na \textit{vývojově aktivní}, v~jazycích \CS{}, \CPP{} a~vybrali jsme námi požadované vlastnosti.
Je zřejmé, že možností na výběr máme opravdu hodně. V následujících podkapitolách si jednotlivé možnosti podrobně rozebereme.

\subsection{Vlastní engine}
Tuto možnost rovnou zavrhneme. Vzhledem k~tomu, kolik funkcionality budeme implementovat, nevidíme přínos v~další práci s~implementací vlastního enginu. Naším cílem je prototyp hry a~tudíž nechceme ztrácet drahocenný čas vývojem nutných nástrojů a~systému pro naši hru.


\subsection{Vlastní engine s~použitím již existujících grafických knihoven}
Máme na výběr z~více druhů grafických frameworků postavených na různých platformách. Mezi známějšími bychom mohli uvést například \XNA{} (\CS{}) či jeho klon \MG{} (\CS{}). Oba frameworky jsou k~dispozici zdarma, podpora \XNA{} je v~současné době už ukončena, vývoj \MG{} je stále aktivní. Implementace hry s~použitím některého z~těchto frameworků by byla rychlejší než v~předchozím případě, ale stále bychom museli spoustu funkcionality implementovat sami. 

\subsection{Existující herní engine}
Jak jsme již předeslali výše, v~této kategorii máme nejvíce možností. Buď můžeme využít enginy jako třeba \OG{} (\CPP{}), nebo použít více robustnější řešení v~podobě enginů typu \UN{} (\texttt{C\#}) či \UE{} (\texttt{C++})\footnote{V době volby herního engine byl další známý zástupce \CRY{} placený a proto jsme tento herní engine nebrali v potaz.}. Zde opět použijeme předchozí argument -- budeme hledat engine, který nám nabídne pokud možno co nejvíce uživatelské a~vývojářské přívětivosti a~bude poskytovat dostatek nástrojů pro vývoj naší hry v~uvažovaném rozsahu.

Výhodou zmíněných robustních enginů je to, že jsou k~dispozici zdarma. Taktéž zde, díky práci komunity, existuje pro oba enginy kvalitní vývojová dokumentace. Dalším kladem je fakt, že jsou oba multiplatformní a~tedy zde existuje relativně snadný postup v~případě distribuce na různé typy herních zařízení. Pojďme si je tedy rozebrat podrobněji.

\subsubsection{Unity}
Výhodu \UN{} vidíme v~tom, že i~programátor bez rozsáhlých zkušeností s~herním vývojem může začít velmi brzy prototypovat a~vyvíjet hry v~tomto enginu. Dalším pozitivem je programování v~\CS{} a~možnost editovatelného terénu.

Použití \UN{} s~sebou přináší i~několik problémů, které bychom museli během vývoje řešit. Během rešerše jsme zaznamenali problémy s~aktualizací dynamického navigačního meshe, kdy aktualizace tohoto meshe způsobovala krátkodobé zaseknutí hry (tzv. lagy). Můžeme očekávat, že tato funkcionalita bude v~budoucnu vylepšena a~zrychlena, nicméně na konkrétní datum se nemůžeme spoléhat. Vzhledem k~povaze naší hry ale můžeme očekávat časté modifikace herního světa a~tudíž toto chování pro nás představuje významný problém. Další nevýhodu vidíme v~materiálovém editoru, který nabízí oproti \UE{} limitované možnosti a~pro implementaci náročnějších materiálových funkcí bychom museli přistoupit k~implementaci vlastních shaderů.

Co se lokalizace hry týče, museli bychom si napsat vlastní správu lokalizace (oficiální tutorial na implementaci lokalizace~\citep{unity_loc}). \UE{} má tuto funkcionalitu implementovanou ve svém editoru (oficiální dokumentace k lokalizaci~\citep{ue_loc}).


\subsubsection{Unreal Engine}
Oproti \UN{} je \UE{} podstatně komplexnější a~pochopení všech vztahů a~závislostí může být pro začínajícího herního programátora obtížné. Přes tuto zjevnou nevýhodu jsme však během rešerše zjistili, že \UE{} nám poskytuje podstatně příjemnější prostředí pro vývoj s~komplexnějšími nástroji. Grafické možnosti máme díky materiálovému editoru k~dispozici od začátku a~nemusíme k~tomu umět psát shadery třeba v~jazyce HLSL. Je nám jasné, že výsledný grafický výkon nemusí být nutně optimální, nicméně vzhledem k~povaze této práce stejně nebudeme cílit na grafickou a~výkonovou optimalizaci.

Testy s~navigačním meshem a~jeho dynamickou aktualizací byly uspokojivé -- nenarazili jsme na žádný zádrhel nebo pokles výkonu během aktualizace meshe. 

Co musíme zmínit jako nevýhodu je absence editovatelného terénu. V editoru je možné vytvořit krásný terén se rozličnými možnostmi detailů, nicméně tento terén není možné jednoduchým způsobem editovat. Chápeme to spíše jako nepříjemnost, než zásadní nevýhodu. 

Další komplikaci vidíme v~použití \CPP{}, se kterým jsme v~době rešerše měli malé zkušenosti.




\subsection{Volba engine -verdikt}


Nakonec jsme zvolili poslední možnost -- \UE{}. Autorovy znalosti především z~oblasti \texttt{C\#} sice hovořily pro použití \UN{}, nicméně výhody použití \UE{} byly dostatečnými argumenty pro to, abychom navzdory malé znalosti \CPP{} a obtížnosti pro začínajícího vývojáře zvolili právě toto řešení.

\section{Struktura projektu}

Celý projekt v \UE{} je rozdělen do dvou částí -- \CPP{} část a~Blueprintová část. \UE{} umožňuje herním vývojářům implementovat celou hru kompletně za pomocí Blueprintů, tedy vizuálního rozhraní. Toho se však obvykle nevyužívá, protože vykonávání programu v~Blueprintu je přibližně 10 krát pomalejší, než vykonávání nativního \CPP{} kódu (odpověď jednoho z vývojářů \UE{} na dotaz ohledně rychlosti Blueprintů: \uv{\textit{Blueprints are approximately the speed of UnrealScript in UE3, and our rule of thumb there was about 10x slower than C++.}},~\citep{ue_performance}). V praxi tak dochází k~tomu, že vývojář (i neprogramátor) může za pomocí Blueprintů rychle prototypovat funkcionalitu, kterou pak kodér přepíše do metod v~\CPP{}, správně tyto metody označí makry tak, aby \UE{} tyto metody v~kódu našel (což se provádí za pomocí reflexe v~\UBT{} a~použitím příslušných \CPP{} maker) a~upraví Blueprint tak, aby původní kód tyto metody volal.

 V Blueprintu je kód vizualizován jako graf uzlů a~jsou zde zaznamenány vztahy mezi těmito uzly. Uzly tedy odpovídají funkcím v~\CPP{} a~těm je pak možné předávat parametry ať už v~podobě proměnných definovaných v~\CPP{} kódu nějaké třídy, nebo proměnných definovaných přímo v~Blueprintu. Vztahy mezi uzly pak označují následující kód určený k~vykonávání. Můžeme také říct, že vykonávání kódu v~Blueprintu je \textit{interpretované}, z~čehož vyplývá absence případných kompilačních optimalizací. Ergo slabý výkon. 

Některé části programu jsou implementovány na úrovni \CPP{}, jiné bylo nutné implementovat v~Blueprintech. Už víme, že komunikace z~Blueprintu do \CPP{} je možná prostým voláním metod, ale určitě budeme potřebovat i~možnost opačného směru. Toho je možné dosáhnout více způsoby -- kupříkladu použitím delegátů a~událostí (Blueprint se pak na tuto událost naváže a~v~případě vyvolání dané události vyvolá v~Blueprintu definovanou obsluhu), nebo BlueprintImplementable či BlueprintNative metod. (TODO formátování!, link na Specifikaci?) Poslední dvě nám nabízí možnost, jak volat virtuální metody, které je možné přepisovat jak na straně \CPP{}, tak na straně Blueprintu, což se nám bude hodit.
