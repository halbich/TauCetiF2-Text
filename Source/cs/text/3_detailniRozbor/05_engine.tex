%!TEX root = ../../prace.tex

\section{Herní engine}
V první řadě bychom se měli zamyslet nad tím, jaký nástroj pro vývoj hry použijeme. Díky tomu budeme moct počítat s možnostmi a omezeními danými touto volbou. Shrňme si, co budeme ve hře potřebovat:

\begin{itemize}
	\item Renderování 3D objektů, pokročilé možnosti texturování
	\item Podpora I/O pro práci se savy
	\item Podpora UI
	\item Podpora zvuků
	\item Snadná implementace lokalizace
	\item Správa assetů
	\item Správa scény
	
\end{itemize}

Pro další případný rozvoj bychom potřebovali:

\begin{itemize}
	\item Podpora pathfindingu
	\item Podpora síťové hry
	\item Podpora AI
\end{itemize}

Cílové platformy pro nás bude PC s OS Windows. Pokud se rozhodneme pro již existující herní engine, který bude navíc podporovat multiplatformní vývoj, bude to pro nás, i s ohledem na další vývoj, plus.

Dalším kritériem je volba programovacího jazyka. Ta vychází z autorových znalostí. Budeme tedy preferovat primárně jazyk \CS{}, který známe nejlépe. Pokud to bude nezbytně nutné, nebudeme se bránit ani jazyku \CPP{}, který je v herní branži dlouho zavedený a je stále hojně využívaný. Ačkoliv zkušenost s tímto programovacím jazykem máme minimální, můžeme se tímto způsobem naučit novým dovednostem.


Možných použitých enginů a frameworků je opravdu mnoho. Podívat do databáze herních enginů na stránce Devmaster. Jen zde je možné nalézt 236 možných řešení našeho problému volby herního enginu \citep{engines_list}. Všechny záznamy jsme omezili na \textit{vývojově aktivní}, v jazycích \CS{}, \CPP{} a vybrali jsme námi požadované vlastnosti.

Mezi čím tedy můžeme volit?
\begin{itemize}
	\item Implementace kompletního vlastního enginu
	\item Použit existující grafické knihovny a nad tím implementovat vlastní engine
	\item Použit existující herní engine
\end{itemize}

Je zřejmé, že možností na výběr máme opravdu hodně. V následujících podkapitolách si jednotlivé možnosti podrobně rozebereme.

\subsection{Vlastní engine}
Tuto možnost rovnou zavrhneme. Vzhledem k tomu, kolik funkcionality budeme implementovat, nevidíme přínos v další práci s implementací vlastního enginu. Naším cílem je prototyp hry a tudíž nechceme ztrácet drahocenný čas vývojem nutných nástrojů a systému pro naši hru.


\subsection{Vlastní engine s použitím již existujících grafických knihoven}
Máme na výběr z více druhů grafických frameworků postavených na různých platformách. Mezi známějšími bychom mohli uvést například \XNA{} (\CS{}) či jeho klon \MG{} (\CS{}). Oba frameworky jsou k dispozici zdarma, podpora \XNA{} je v současné době už ukončena, vývoj \MG{} je stále aktivní. Implementace hry s použitím některého z těchto frameworků by byla rychlejší než v předchozím případě, ale stále bychom museli spoustu funkcionality implementovat sami. 

\subsection{Existující herní engine}
Jak jsme již předeslali výše, v této kategorii máme nejvíce možností. Buď můžeme využít enginy jako třeba\OG{} (\CPP{}), nebo použít více robustnější řešení v podobě enginů typu \UN{} (\texttt{C\#}) či \UE{} (\texttt{C++}). Zde opět použijeme předchozí argument --- budeme hledat engine, který nám nabídne pokud možno co nejvíce uživatelské a vývojářské přívětivosti a bude poskytovat dostatek nástrojů pro vývoj naší hry v uvažovaném rozsahu. Tudíž enginy jako třeba \OG{} nebudou naší volbou.

Výhodou zmíněných robustních enginů je to, že jsou k dispozici zdarma (oproti třeba \CRY{}). Taktéž zde, díky práci komunity, existuje pro oba enginy kvalitní vývojová dokumentace. Dalším kladem je fakt, že jsou oba multiplatformní a tedy zde existuje relativně snadný postup v případě distribuce na různé typy herních zařízení. Pojďme si je tedy rozebrat podrobněji.

\subsubsection{Unity}
Výhodu \UN{} vidíme v tom, že i programátor bez rozsáhlých zkušeností s herním vývojem může začít velmi brzy prototypovat a vyvíjet hry v tomto enginu. Dalším pozitivem je programování v \CS{} a možnost editovatelného terénu.

Použití \UN{} s sebou přináší i několik problémů, které bychom museli během vývoje řešit. Během rešerše jsme zaznamenali problémy s aktualizací dynamického navigačního meshe, kdy aktualizace tohoto meshe způsobovala krátkodobé zaseknutí hry (tzv. lagy). Můžeme očekávat, že tato funkcionalita bude v budoucnu vylepšena a zrychlena, nicméně na konkrétní datum se nemůžeme spoléhat. Vzhledem k povaze naší hry ale můžeme očekávat časté modifikace herního světa a tudíž toto chování pro nás představuje významný problém. Další nevýhodu vidíme v materiálovém editoru, který nabízí oproti \UE{} limitované možnosti a pro implementaci náročnějších materiálových funkcí bychom museli přistoupit k implementaci vlastních shaderů.

Co se lokalizace hry týče, museli bychom si napsat vlastní správu lokalizace\citep{unity_loc}. \UE{} má tuto funkcionalitu implementovanou ve svém editoru\citep{ue_loc}.


\subsubsection{Unreal Engine}
Oproti \UN{} je \UE{} podstatně komplexnější a pochopení všech vztahů a závislostí může být pro začínajícího herního programátora obtížné. Přes tuto zjevnou nevýhodu jsme však běhe, rešerš zjistili, že \UE{} nám poskytuje podstatně příjemnější prostředí pro vývoj s komplexnějšími nástroji. Grafické možnosti máme díky materiálovému editoru k dispozici od začátku a nemusíme k tomu umět psát shadery třeba v jazyce HLSL. Je nám jasné, že výsledný grafický výkon nemusí být nutně optimální, nicméně vzhledem k povaze této práce stejně nebudeme cílit na grafickou a výkonovou optimalizaci.

Testy s navigačním meshem a jeho dynamickou aktualizací byly uspokojivé - nenarazili jsme na žádný zádrhel nebo pokles výkonu během aktualizace meshe. 

Co musíme zmínit jako nevýhodu je absence editovatelného terénu. (TODO link). V editoru je možné vytvořit krásný terén se rozličnými možnostmi detailů, nicméně tento terén není možné jednoduchým způsobem editovat. Chápeme to spíše jako nepříjemnost, než zásadní nevýhodu. 

Další komplikaci vidíme v použití \CPP{}, se kterým jsme v době rešerše měli malé zkušenosti.




\subsection{Volba engine -verdikt}


Nakonec jsme zvolili poslední možnost - \UE{}. Autorovy znalosti především z oblasti \texttt{C\#} sice hovořily pro použití \UN{}, nicméně výhody použití \UE{} převážily nad nevýhodami i všemi výhodami \UN{}.	// TODO tohle chce vyladit