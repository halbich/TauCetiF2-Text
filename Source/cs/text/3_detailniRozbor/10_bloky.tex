%!TEX root = ../../prace.tex

\section{Bloky}

V této části rozebereme, jak můžeme definovat a~následně implementovat bloky a~popíšeme, jaké jsou výhody a~nevýhody jednotlivých implementací. V~prvé řadě se změříme na celkovou strukturu bloků a~následně budeme řešit, jak budeme spravovat konstanty ovlivňující chování bloků.

\subsection{Celková struktura}

Aby se nám s~bloky dobře pracovalo, jistě bude vhodné využít jednoho ze základních principů \textit{OOP} (objektově orientovaného programování) -- dědičnosti. Takže v~naší hře bude existovat základní třída, která bude vycházet ze třídy \TT{UActor}\footnote{\TT{UActor} je základní třída \UEu{}, ze které dědí všechny herní objekty, které chceme v~hlavní herní smyčce aktualizovat a~renderovat.} a~bude předkem všech našich herních bloků.

Tento prapředek bude obsahovat dvě podstatné informace -- referenci na \textit{definici} daného bloku a~referenci na třídu s~vlastnostmi dané \textit{instance} bloku. Díky tomu, že oddělíme definiční třídu a~instanční třídu, tak získáme možnost získat definiční třídu pro daný typ bloku za běhu hry pouze jednou a~posléze tuto referenci předávat všem instancím bloku daného typu. Instanční třída bude mít pro každý blok jiné hodnoty, takže je zřejmé, že by měla být samostatná. Smyslem definičního souboru je popis omezujících podmínek kladených na daný typ bloku (například povolené minimální a~maximální rozměry), přičemž hodnoty v~instanční třídě by měly být v~mezích dané definicí. Vlastnosti, které nejsou omezující (například cena za postavení bloku), nebudeme uchovávat v~\textit{instanční} třídě, tyto údaje budeme získávat přímo z~\textit{definiční} třídy.

\subsection{Instanční vlastnosti}
\label{subsec:instVlast}
Mezi instanční vlastnosti zařadíme vlastnosti definované v~části \ref{subsec:blocks} --  \textit{vizuální reprezentaci}, \textit{pozici ve světě}, \textit{rotaci}, \textit{velikost}, \textit{zdraví} apod. Protože máme dynamicky škálovatelné bloky, budeme chtít, abychom měli kupříkladu výpočet výsledného zdraví či ceny za postavení bloku bloku co nejjednodušší. K~tomu využijeme vlastnost \textit{typ bloku}. (TODO když má blok nějakou vlastnost, tak podle typu a~rozměěrů se přepočítá výsledná síla vlastnost, abstraktní cena, bloky různě velké, pak přepočet)

Zadefinujme si následující konstanty pro daný \textit{typ}, vycházející z~objemu základní krychle: (jednotlivé hodnoty odpovídají sloupci \textbf{T} v~tabulce \ref{table:requiredBlocks})

\begin{enumerate}
	\item Krychle \textbf{K}: $1$
	\item Zkosená krychle \textbf{Z}: $\frac{1}{2}$
	\item Rohová krychle \textbf{R}: $\frac{1}{6}$
	\item Vlastní \textbf{V}: $1$
\end{enumerate}

Tyto konstanty využijeme v~následujícím algoritmu výpočtu

\begin{equation}\label{eq:alg}
	\bm H = \bm T * h  * x * y * z
\end{equation}

kde $x, y, z$ jsou rozměry bloku v~daných osách, $\bm T$ je konstanta dle typu bloku, $h$ je nějaká základní hodnota (například zdraví) a~$\bm H$ je výsledná hodnota vlastnosti. Výpočet \ref{eq:alg} se opírá o~následující fakta:

\begin{enumerate}
	\item Blok, který je typu \textbf{V}, má vždy pevně definované rozměry a~nelze jej škálovat.
	\item Blok, který je typu \textbf{V}, do rovnice vždy dosazuje $ x = y = z~= 1$.
	\subitem To je čistě designová záležitost, abychom mohli během zadávání konstant pro daný blok tohoto typu vždy zadat pouze výslednou hodnotu. Vyhneme se tím přepočítávání a~zadání konstant bude přehlednější.
\end{enumerate}




%!TEX root = ../../prace.tex



\section{Komponenty bloků}

Abychom mohli snadno rozšiřovat vlastnosti a~chování bloků, použijeme systém \textit{komponent}, který nám \UE{} nabízí. Komponenta je programová část, která ovlivňuje chování vlastníka dané komponenty. Cílem je pak dosáhnout toho, že je možné za běhu hry jednu komponentu transparentně vyměnit za jinou (komponentu s~jinou implementací), a~vlastník komponenty se nemusí zajímat o~detaily implementace. V naší hře toto chování nejspíše nevyužijeme, ale použití komponent není na škodu a~v~případě dalšího vývoje budeme mít snazší práci. (TODO učesat)

Z předchozí analýzy vyplývá, že budeme potřebovat řešit následující problémy:

\begin{itemize}
	\item Práce s~kyslíkem
	\item Práce s~elektrickou sítí a~energií
	\item Interakce s~uživatelem
	\item Umístění bloku v~herním světě
\end{itemize}


První dva body jsou ideální kandidáti na použití komponent. Pokud bychom se někdy v~budoucnu rozhodli upravit chování této funkcionality či jej z~libovolného důvodu měnit, komponentový systém pro nás bude výhodou. Navíc ne všechny herní bloky umí (z hlediska herního designu) s~danými prvky pracovat. Jak jsme již zmínili dříve, \UE{} nepovoluje vícenásobnou dědičnost a~bylo by velmi těžké vymyslet hiearchii dědičnosti bloků tak, abychom splnili požadavky pro všechny bloky a~zároveň si \uv{nesvázali ruce} pro nové bloky. S použitím komponent to bude snadné -- bloky, které danou funkcionalitu mají umět, budou mít danou komponentu a~budou s~ní moci pracovat.

Dalším problémem je interakce s~uživatelem. Abychom věděli, že hráč s~daným blokem chce interagovat, musíme vědět, že:
\begin{itemize}
	\item Je dostatečně blízko bloku
	\item Z pohledu hráče se dívá na daný blok 
	\item Vyjadřuje fakt, že chce interagovat (např. stiskem klávesy)
\end{itemize}


\subsection{Interakce a~označování}

Nejsnazší způsob, jak zjistit, na jaký herní objekt se hráč dívá, je použití RayTracingu (TODO link?, formát textu?). Díky němu můžeme \uv{z kamery} vyslat virtuální paprsek, který má stejný směr, jako je směr pohledu kamery. Pokud bude hráčův \HUD{} zobrazovat zaměřovací kříž či nějaký obdobný mechanismus a~náš paprsek bude z~pohledu kamery tímto zaměřovačem procházet, hráč může cíleně mířit na herní objekty a~my zároveň budeme mít správnou informaci o~objektu, na který hráč zaměřovačem míří. Tento způsob získávání informace o~objektech v~hráčově zaměřovači je ve hrách běžný a~použití RayTrace je (pokud je vhodně použito) i~dostatečně rychlé.

Nyní, když už víme, jak můžeme získávat informace o~tom, na který objekt hráč míří, tak tento mechanismus ještě rozšíříme o~další vlastnost. Je zapotřebí si uvědomit, že interakce s~blokem a~umisťování nového herního bloku (případně mazání) jsou prakticky jedna a~ta samá akce. Liší se pouze výsledkem -- reakcí na stisk nějaké klávesy či tlačítka myši. Ale ve všech případech musíme vědět, na jaký blok hráč míří zaměřovačem, u~umisťování navíc potřebujeme znát i~přesný polygon, na který hráč míří. Konkrétní polygon potřebujeme znát z~toho důvodu, že chceme, aby se přidávaný blok \uv{přilepil} k~bloku, na který míříme. Tedy chceme zachovat herní mechaniku, která je v~hrách z~kapitoly \ref{chap:uvod} běžná a~je natolik intuitivní a~rozšířená, že změna této mechaniky by nejspíše nedopadla dobře a~hráči by nebyla kladně přijata.

Všechny tyto požadavky lze splnit použitím metody \TT{LineTraceSingleByObjectType} (TODO ref?), které předáme správné parametry (především počátek a~konec paprsku a~typy objektů , které paprsek zaznamená) a~ta nám vrátí strukturu, popisující výsledek RT. Z něj se můžeme dozvědět, jestli byl nějaký blok v~cestě paprsku. A pokud ano, můžeme se ptát, zda měl komponentu interakce (potenciálně bychom mohli chtít bloky bez možnosti zaměření a~interakce, jakožto nesmazatelné objekty). Pokud bude i~tato podmínka splněna, můžeme se zajímat o~další vlastnosti kolize paprsku s~blokem a~na základě toho se nějak chovat.


\subsection{Umístění ve světě}

Smyslem této komponenty je oddělení implementace bloku jako takového a~implementace herního světa.


Ve výsledku tedy budeme chtít komponentu, která 

popis jednotlivých komponent dle předchozího, co všechno umí (např. přidání / odebrání hodnoty energie za použité zámku (není transakce))





\section{Definice bloků TODO}

Jak je řečeno v~analýze (..) potřebujeme tyhle vlastnosti: (TODO)

Určitě budeme schopni najít množinu vlastností, které mají všechny bloky společnou. Jak již bylo zmíněno v~analýze (TODO ref), mezi tyto vlastnosti určitě bude patřit  \textit{pozice} ve světě, \textit{rotace} bloku a~jeho \textit{velikost}.Tyto vlastnosti pak pro danou instanci můžeme nějakým způsobem serializovat do souboru a~tím budeme mít validní soubor pro uložení rozehrané hry. 

Protože jednotlivé bloky rozšiřují tyto základní informace o~další vlastnosti (například \textit{elektrická} či \textit{kyslíková} komponenta má také své vlastnosti), můžeme se zamyslet nad tím, že bychom pro jednotlivé části bázovou třídu základních vlastností rozšířili. Zde ale narážíme na problém, protože jednotlivé bloky mohou komponenty libovolně kombinovat. Ačkoliv \CPP{} standardně povoluje vícenásobnou dědičnost tříd, \UE{} toto nepovoluje (TODO link!) a~tudíž tento přístup není vhodný. Řešením bude rozdělit vlastnosti jednotlivých komponent do samostatných tříd a~bloky, které budou tyto komponenty obsahovat, si pak budou držet referenci na instanci dané třídy s~vlastnostmi pro danou komponentu. Jako bonus tím získáme typovou bezpečnost a~odpadá nám nutnost přetypovávání na vyšší typ (protože ve společném předkovi všech bloků

Z předchozí části víme, že budeme chtít takovou strukturu, abychom mohli v~\UE{} snadno modifikovat hodnoty jejích vlastností. Takže nepřipadá v~úvahu, abychom měli konstanty uložené ve zdrojovém kódu. Pak ale tyto konstanty budeme muset buď načítat z~nějakého souboru, nebo budou muset být uložené v~nějakém objektu z~\UE{}.

Při načítání ze souboru máme více možností -- můžeme popisovat bloky a~jejich chování například v~XML souborech. Obdobné řešení používá hra \ME{}. Tyto soubory jsou pak zpracovány herním enginem během načítání hry. 

\subsection{Textové soubory}
V této části se zamyslíme nad použitím definic v~textovém souboru. Může se jednat o~množinu samostatných souborů, přičemž budeme uvažovat pouze XML soubory, nebo můžeme použít třeba popis bloků v~nějakém tabulkovém formátu, třeba csv.

\subsubsection{Popis tabulkou}
Pokud bychom měli velice málo vlastností bloků, tento přístup by mohl být použitelný. Nicméně s~každým dalším nově přidaným blokem se do množiny všech vlastností mohou zanášet nové vlastnosti.  To by znamenalo, že popis ostatních bloků, které danou vlastnost nemají, by musel nutně v~tomto tabulkovém zápisu uvažovat nějakou (byť prázdnou) hodnotu. Zbytečně by nám tak rostl definiční soubor. Další nevýhodou je absence typové bezpečnosti. 

\subsubsection{Popis samostatným souborem -- XML}
Běžné soubory textového formátu není potřeba brát v~potaz. Výsledek je stejný jako při použití XML, ale nemůžeme zde použít definiční soubory pro automatickou kontrolu platnosti hodnot. Navíc bychom museli psát vlastní parser takového textového souboru, přičemž již hotové parsery XML jsou volně k~dispozici. 

Výhodou tohoto přístupu je také budoucí rozšiřovatelnost o~nové bloky

Soubor obsahuje popis vlastno

- externě editovatelné formáty (+ -- modding, -- těžší implementace, parsing, validace)
- binární formát

- xml


- interní formát
- specifické subclassy pro bloky včetně specifických vlastností přímo na 
- definiční struktura



%!TEX root = ../prace.tex

\section{Vlastnosti bloků}

- bloky mohou mít několik vlastností:

- mít možnost elektriky a zapojení do elektrické sítě

- mít možnost uchování kyslíku, v případě použitíí elektirky pak i generování

- bloky mohou být použitelné, tj. hráč s nimi může nějakým zůsobem interagovat

- bloky mohou být sebratelné, tedy hráč si je může dát do svého inventáře. vlastnosti jako třeba uchovaná hodnota kyslíku, pak zůstávají zachované

- bloy mohou být zákadem pro rozpoznávání tvarů (TODO )

\subsection{Elektrika}


\subsection{Kyslík}


\subsection{Označovatelnost}


\subsection{Možnost vzít do inventáře}

\subsection{Interakce}


