%!TEX root = ../../prace.tex



\section{Komponenty bloků}
\label{sec:komponents}

Abychom mohli snadno rozšiřovat vlastnosti a~chování bloků, použijeme systém \textit{komponent}, který nám \UE{} nabízí. Komponenta je programová část, která ovlivňuje chování vlastníka dané komponenty. Cílem je pak dosáhnout toho, že je možné za běhu hry jednu komponentu transparentně vyměnit za jinou (komponentu s~jinou implementací), a~vlastník komponenty se nemusí zajímat o~detaily implementace.

Z předchozí analýzy vyplývá, že budeme potřebovat řešit následující problémy:

\begin{itemize}
	\item Práce s~kyslíkem
	\item Práce s~elektrickou sítí a~energií
	\item Interakce s~uživatelem
	\item Umístění bloku v~herním světě
\end{itemize}


Tyto problémy jsou ideální kandidáti na použití komponent. Pokud bychom se někdy v~budoucnu rozhodli upravit chování některé funkcionality či jej z~libovolného důvodu měnit, komponentový systém pro nás bude výhodou. Navíc ne všechny herní bloky umí (z hlediska herního designu) kupříkladu s~kyslíkem či elektřinou pracovat. Jak jsme již zmínili dříve, \UE{} nepovoluje vícenásobnou dědičnost a~bylo by velmi těžké vymyslet hierarchii dědičnosti bloků tak, abychom splnili požadavky pro všechny bloky a~zároveň si \uv{nesvázali ruce} pro nové bloky. S použitím komponent to bude snadné -- bloky, které danou funkcionalitu mají umět, budou mít danou komponentu a~budou s~ní moci pracovat. Taktéž tím u~kyslíku a~energie získáme možnost přiřadit tyto komponenty hráčově postavě a~tím nebudeme duplikovat funkcionalitu.

\subsection{Komponenta kyslíku}

Cílem komponenty kyslíku je udržování informace o~aktuálním množství \linebreak \uv{vlastněného} kyslíku a~dále pak převody mezi jednotlivými komponentami. Vzhledem k~tomu, že kyslík je vyráběn z~elektrické energie, blok s~elektrickou a~kyslíkovou komponentou může měnit množství vlastněného kyslíku. 

\subsubsection{Převádění kyslíku}
Protože \UE{} umožňuje hrám pracovat s~více výpočetními vlákny, musíme zajistit konzistenci dat při převodech kyslíku. Můžeme pro to využít primitiva pro zamykání \TT{FCriticalSection}. Kritickou sekci pak budeme korektně zamykat a~odemykat (stejně jako u~klasického vícevláknového programování). Algoritmy pro vkládání a~získání kyslíku budou mít následující signaturu:

\begin{code}
    // dodej kyslík komponentě
    bool UOxygenComponent::PutAmount(float aviable,
                                     float& actuallyPutted)

    // získej kyslík z~komponenty                                     
    bool UOxygenComponent::ObtainAmount(float requested,
                                        float& actuallyObtained,
                                        bool requireExact)
\end{code}

Princip je prostý -- metody vrací \TT{bool} jakožto hodnotu, zda bylo možné operaci korektně provést. Parametry předávané \textit{referencí} pak v~případě úspěchu obsahují hodnotu skutečně vloženého či získaného kyslíku. Poslední parametr u~metody pro získání kyslíku značí, zda je vyžadované přesné množství. Pokud kyslíková komponenta obsahuje méně kyslíku, než je požadované množství a~je požadované přesně zadané množství, převod nebude úspěšný a~metoda vrátí \TT{false}. Pokud nebude požadované přesně dané množství, skutečně získané množství může být menší a~je na volajícím, aby se tomuto faktu přizpůsobil.

Obdobně jako u~bloků, kyslíková komponenta bude obsahovat reference na \textit{definiční} třídu a~\textit{instanční} třídu a~jejich chování bude vycházet z~principů chování definiční a~instanční třídy u~vlastností bloků. \textit{Definiční} třída definuje \textit{maximální objem drženého kyslíku} pro daný blok (a stejně jako u~výpočtu zdraví se pro danou instanci bloku použije algoritmus \ref{eq:alg}) a~dále definuje \textit{maximální přenesený objem kyslíku za 1~herní sekundu}. Z toho vyplývá, že můžeme řídit, jak dlouho bude probíhat přenos třeba z~bloku plničky do bloku kyslíkové bomby. Nebo z~bloku kyslíkové bomby k~hráči. Limitace přenosu pak přináší věrohodnější hráčský zážitek.

\subsection{Komponenta energie}

Komponenta energie je mnohém velmi podobná Kyslíkové komponentě. Jejím cílem je udržování informace o~\uv{vlastněné} energii a~taktéž obsahuje \textit{definiční} a~\textit{instanční} třídu. \textit{Definice} pro daný blok také obsahuje \textit{maximální objem} a~\textit{limit přenosu}. Signatury metod pro převod energie jsou identické s~kyslíkovými.

Navíc však komponenta energie umí pracovat s~elektrickou sítí. Aby byla komponenta v~nějaké elektrické síti, musí se blok vlastnící tuto komponentu dotýkat jiného bloku s~elektrickou komponentou. Protože vycházíme z~reálného světa, tento dotyk vlastně simuluje vodivý spoj. Proto musíme pro každý blok s~touto komponentou umět zadefinovat, jak takový vodivý spoj vypadá.

\subsubsection{Vodivé spojení bloků}

Definovat vodivý spoj můžeme několika způsoby. Mohli bychom použít (stejně jako u~stavění ve hře \TM{}) systém přípojných bodů. Nicméně to by znamenalo, že bychom museli řešit, jak tento systém upravit pro dynamicky škálovatelné bloky. Navíc, pro nové vizuální reprezentace bloků bychom museli speciálně řešit umístění těchto přípojných bodů, i~vzhledem k~ostatním možným přípojným bloků ve hře). Další zásadní nevýhodou je to, jak bychom napojovali například krychle o~maximálním a~minimálním rozměru. Abychom mohli využít přípojných bodů, minimální krychle by musela definovat $6$ takových bodů a~maximální krychle pak $2~400$ těchto bodů. S rostoucí velikostí bloku se počet kontrolovaných přípojných bodů zvětšuje.  Tento postup nám nevyhovuje.

Můžeme však zvolit obdobné řešení -- \uv{\textit{polygonální}}. Pro daný blok definujeme \textit{body}, které se budou v~rámci škálování korektně umisťovat vždy na stejné místo (například roh krychle). Dále pak definujeme \textit{polygony}, což je uspořádaný seznam bodů, které leží v~dané rovině. Tento seznam tvoří virtuální cyklus, takže mezi $k$-tým a~$(k+1)$-tým bodem je \textit{hrana} polygonu.

V rámci algoritmu spojování bloků skrz vodivé spojení pak budeme hledat pro daný blok sousedy právě dle těchto definic. Budeme počítat s~tím, že polygony jsou v~rámci jedné roviny v~prostoru, přičemž daná rovina je rovnoběžná s~některou z~rovin mřížky herního světa. V rámci této práce si algoritmus usnadníme takovým způsobem, že jednotlivé studované polygony rozšíříme na \textit{boxy}, tedy najdeme nejmenší možný kvádr, který je zarovnaný do mřížky definované herním světem (tedy nemá vůči ní žádnou rotaci) a~obsahuje všechny body z~daného polygonu\footnote{Abychom mohli použít standardních metod \TT{FBox::Intersect(...)}, musíme \textit{box} rozšířit o~něco netriviálního (např. $1~cm$) ve směru normály dané roviny.}. Za validní a~vodivé elektrické vedení pak budeme považovat takové dva polygony, které mají neprázdný průnik těchto boxů. Jsme si vědomi faktu, že tato úprava jde proti snahám hry o~vyšší stupeň realismu, ale v~této fázi nám to stačí a~nepotřebujeme matematicky správné řešení.


\subsection{Objekt elektrické sítě}

//TODO proč jsme to tak udělali?

Tento objekt popisuje jednu konkrétní elektrickou síť. Obsahuje seznam všech \textit{elektrických komponent} v~dané síti. Sama o~sobě nemusí mít žádnou složitou logiku. Síť by měla umět k~sobě přiřadit novou elektrickou komponentu či ji odebrat. 


// TODO
Ve hře pak bude jeden Actor, který se bude o~všechny sítě starat a~bude je hromadně aktualizovat -- protože sítě vznikají a~zanikají relativně rychle, je tam propagace do sítě, mnoho tikajících actorů -- bordel na update, proto jeden a~ten to vše vyřeší.



\subsection{Interakce a~označování}

Dalším problémem je interakce s~uživatelem. Abychom věděli, že hráč s~daným blokem chce interagovat, musíme vědět, že:
\begin{itemize}
	\item Je dostatečně blízko bloku
	\item Z pohledu hráče se dívá na daný blok 
	\item Vyjadřuje fakt, že chce interagovat (např. stiskem klávesy)
\end{itemize}



Nejsnazší způsob, jak zjistit, na jaký herní objekt se hráč dívá, je použití techniky sledování paprsku (\textit{ray tracing}). Díky němu můžeme \uv{z kamery} vyslat virtuální paprsek, který má stejný směr, jako je směr pohledu kamery. Pokud bude hráčův \HUD{} zobrazovat zaměřovací kříž (či použijeme nějaký obdobný mechanismus) a~náš paprsek bude z~pohledu kamery tímto zaměřovačem procházet, hráč může cíleně mířit na herní objekty a~my zároveň budeme mít správnou informaci o~objektu, na který hráč zaměřovačem míří. Tento způsob získávání informace o~objektech v~hráčově zaměřovači je ve hrách běžný a~jeho použití je (pokud je vhodně použito) i~dostatečně rychlé.

Nyní, když už víme, jak můžeme získávat informace o~tom, na který objekt hráč míří, tak tento mechanismus ještě rozšíříme o~další vlastnost. Je zapotřebí si uvědomit, že interakce s~blokem a~umisťování nového herního bloku (případně mazání) jsou prakticky stejné akce. Liší se pouze výsledkem -- reakcí na stisk nějaké klávesy či tlačítka myši. Ale ve všech případech musíme vědět, na jaký blok hráč míří zaměřovačem, u~umisťování navíc potřebujeme znát i~přesný polygon, na který hráč míří. Konkrétní polygon potřebujeme znát z~toho důvodu, že chceme, aby se přidávaný blok \uv{přilepil} k~bloku, na který míříme. Tedy chceme zachovat herní mechaniku, která je v~hrách z~kapitoly \ref{chap:uvod} běžná a~je natolik intuitivní a~rozšířená, že změna této mechaniky by nejspíše nedopadla dobře a~hráči by nebyla kladně přijata.

Pro implementaci nám poslouží metoda \TT{LineTraceSingleByObjectType},\linebreak které předáme správné parametry (především počátek a~konec paprsku a~typy objektů, které paprsek zaznamená) a~ta nám vrátí strukturu, popisující výsledek trasování. Z něj se můžeme dozvědět, jestli byl nějaký blok v~cestě paprsku. A pokud ano, můžeme se ptát, zda měl komponentu interakce (potenciálně bychom mohli chtít bloky bez možnosti zaměření a~interakce, jakožto nesmazatelné objekty). Pokud bude i~tato podmínka splněna, můžeme se zajímat o~další vlastnosti kolize paprsku s~blokem a~na základě toho se nějak chovat.

Abychom hráči hraní usnadnili a~zpříjemnili, budeme bloky zvýrazňovat za použití techniky \textit{obrysu} (implementace zvýrazňování bude vycházet z tutorialu \citep{ue_outline_tut}). Tato technika je také využívána v~různých hrách napříč různými herními žánry a~proto její implementací nic nezkazíme. Obrys zobrazíme v~případě, kdy hráč nic nestaví a~daný označený blok je použitelný (takže hráč ví, že blok může používat), nebo v~případě, kdy hráč staví (zvýrazníme hranice namířeného objektu). Barvu obrysu můžeme zvolit kupříkladu zelenou pro použitelný blok, žlutou při stavění a červenou při odebírání bloku. 


\subsubsection{Možnost vzít blok do inventáře}

- bloky mohou být sebratelné, tedy hráč si je může dát do svého inventáře. vlastnosti jako třeba uchovaná hodnota kyslíku, pak zůstávají zachované

\subsection{Umístění ve světě}

Smyslem této komponenty je oddělení implementace bloku jako takového a~implementace herního světa. Skrze tuto komponentu se blok bude moci dotazovat na ostatní bloky v~herním světě, především pak bloky ve svém okolí. To bude důležité například pro elektrickou komponentu, která na základě \uv{sousedství} bude jednotlivé bloky vázat k~sobě do elektrické sítě.






