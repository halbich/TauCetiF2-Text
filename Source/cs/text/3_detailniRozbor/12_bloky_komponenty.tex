%!TEX root = ../../prace.tex



\section{Komponenty bloků}

Abychom mohli snadno rozšiřovat vlastnosti a chování bloků, použijeme systém \textit{komponent}, který nám \UE{} nabízí. Komponenta je programová část, která ovlivňuje chování vlastníka dané komponenty. Cílem je pak dosáhnout toho, že je možné za běhu hry jednu komponentu transparentně vyměnit za jinou (komponentu s jinou implementací), a vlastník komponenty se nemusí zajímat o detaily implementace. V naší hře toto chování nejspíše nevyužijeme, ale použití komponent není na škodu a v případě dalšího vývoje budeme mít snazší práci. (TODO učesat)

Z předchozí analýzy vyplývá, že budeme potřebovat řešit následující problémy:

\begin{itemize}
	\item Práce s kyslíkem
	\item Práce s elektrickou sítí a energií
	\item Interakce s uživatelem
	\item Umístění bloku v herním světě
\end{itemize}


První dva body jsou ideální kandidáti na použití komponent. Pokud bychom se někdy v budoucnu rozhodli upravit chování této funkcionality či jej z libovolného důvodu měnit, komponentový systém pro nás bude výhodou. Navíc ne všechny herní bloky umí (z hlediska herního designu) s danými prvky pracovat. Jak jsme již zmínili dříve, \UE{} nepovoluje vícenásobnou dědičnost a bylo by velmi těžké vymyslet hiearchii dědičnosti bloků tak, abychom splnili požadavky pro všechny bloky a zároveň si \uv{nesvázali ruce} pro nové bloky. S použitím komponent to bude snadné -- bloky, které danou funkcionalitu mají umět, budou mít danou komponentu a budou s ní moci pracovat.

Dalším problémem je interakce s uživatelem. Abychom věděli, že hráč s daným blokem chce interagovat, musíme vědět, že:
\begin{itemize}
	\item Je dostatečně blízko bloku
	\item Z pohledu hráče se dívá na daný blok 
	\item Vyjadřuje fakt, že chce interagovat (např. stiskem klávesy)
\end{itemize}


\subsection{Interakce a označování}

Nejsnazší způsob, jak zjistit, na jaký herní objekt se hráč dívá, je použití RayTracingu (TODO link?, formát textu?). Díky němu můžeme \uv{z kamery} vyslat virtuální paprsek, který má stejný směr, jako je směr pohledu kamery. Pokud bude hráčův \HUD{} zobrazovat zaměřovací kříž či nějaký obdobný mechanismus a náš paprsek bude z pohledu kamery tímto zaměřovačem procházet, hráč může cíleně mířit na herní objekty a my zároveň budeme mít správnou informaci o objektu, na který hráč zaměřovačem míří. Tento způsob získávání informace o objektech v hráčově zaměřovači je ve hrách běžný a použití RayTrace je (pokud je vhodně použito) i dostatečně rychlé.

Nyní, když už víme, jak můžeme získávat informace o tom, na který objekt hráč míří, tak tento mechanismus ještě rozšíříme o další vlastnost. Je zapotřebí si uvědomit, že interakce s blokem a umisťování nového herního bloku (případně mazání) jsou prakticky jedna a ta samá akce. Liší se pouze výsledkem -- reakcí na stisk nějaké klávesy či tlačítka myši. Ale ve všech případech musíme vědět, na jaký blok hráč míří zaměřovačem, u umisťování navíc potřebujeme znát i přesný polygon, na který hráč míří. Konkrétní polygon potřebujeme znát z toho důvodu, že chceme, aby se přidávaný blok \uv{přilepil} k bloku, na který míříme. Tedy chceme zachovat herní mechaniku, která je v hrách z kapitoly \ref{chap:uvod} běžná a je natolik intuitivní a rozšířená, že změna této mechaniky by nejspíše nedopadla dobře a hráči by nebyla kladně přijata.

Všechny tyto požadavky lze splnit použitím metody \TT{LineTraceSingleByObjectType} (TODO ref?), které předáme správné parametry (především počátek a konec paprsku a typy objektů , které paprsek zaznamená) a ta nám vrátí strukturu, popisující výsledek RT. Z něj se můžeme dozvědět, jestli byl nějaký blok v cestě paprsku. A pokud ano, můžeme se ptát, zda měl komponentu interakce (potenciálně bychom mohli chtít bloky bez možnosti zaměření a interakce, jakožto nesmazatelné objekty). Pokud bude i tato podmínka splněna, můžeme se zajímat o další vlastnosti kolize paprsku s blokem a na základě toho se nějak chovat.


\subsection{Umístění ve světě}

Smyslem této komponenty je oddělení implementace bloku jako takového a implementace herního světa.


Ve výsledku tedy budeme chtít komponentu, která 

popis jednotlivých komponent dle předchozího, co všechno umí (např. přidání / odebrání hodnoty energie za použité zámku (není transakce))



