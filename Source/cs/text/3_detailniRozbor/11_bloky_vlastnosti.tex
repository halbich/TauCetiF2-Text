%!TEX root = ../../prace.tex

\section{Vlastnosti bloků}

Popis toho, co blok umí


\subsection{Energie}

- popis energie, co to umí (např. výkon)

\subsection{Energetická síť}

-způsob zapojení do sítě

\subsection{Kyslík}

- to je podobný jako energie

- mít možnost uchování kyslíku, v~případě použitíí elektirky pak i~generování


\subsection{Označovatelnost}

- hráč může avatarem zamířit na blok a~ten se označí červeně, žlutě zeleně



\subsection{Možnost vzít do inventáře}

- bloky mohou být sebratelné, tedy hráč si je může dát do svého inventáře. vlastnosti jako třeba uchovaná hodnota kyslíku, pak zůstávají zachované


\subsection{Interakce}

- vypínač, světla - vlastní UI


- bloky mohou být použitelné, tj. hráč s~nimi může nějakým zůsobem interagovat


\subsection{Zapojení do rozpoznávání tvarů}

- generátor bloků

jaké byly možnosti - abecné rozpoznávání (původní implementace, rozvést nutnost rozpadu tvarů na menší (slope) + doplnění kvádry

