%!TEX root = ../../prace.tex

\section{Modul Blocks (C++)}



\subsubsection{Převádění kyslíku}
Protože \UE{} umožňuje hrám pracovat s~více výpočetními vlákny, musíme zajistit konzistenci dat při převodech kyslíku. Můžeme pro to využít primitiva pro zamykání \TT{FCriticalSection}. Kritickou sekci pak budeme korektně zamykat a~odemykat (stejně jako u~klasického vícevláknového programování). Algoritmy pro vkládání a~získání kyslíku budou mít následující signaturu:

\begin{code}
    // dodej kyslík komponentě
    bool UOxygenComponent::PutAmount(float aviable,
                                     float& actuallyPutted)

    // získej kyslík z~komponenty                                     
    bool UOxygenComponent::ObtainAmount(float requested,
                                        float& actuallyObtained,
                                        bool requireExact)
\end{code}

TODO přesunout do programátoské, nechat jen obecný kecy co a~jak

Princip je prostý -- metody vrací \TT{bool} jakožto hodnotu, zda bylo možné operaci korektně provést. Parametry předávané \textit{referencí} pak v~případě úspěchu obsahují hodnotu skutečně vloženého či získaného kyslíku. Poslední parametr u~metody pro získání kyslíku značí, zda je vyžadované přesné množství. Pokud kyslíková komponenta obsahuje méně kyslíku, než je požadované množství a~je požadované přesně zadané množství, převod nebude úspěšný a~metoda vrátí \TT{false}. Pokud nebude požadované přesně dané množství, skutečně získané množství může být menší a~je na volajícím, aby se tomuto faktu přizpůsobil.




Modul bloků obsahuje podstatné informace o~tom, jak hra pracuje s~bloky, jak se tyto bloky skládají do herního světa, jaké jsou jejich komponenty apod. Také je v~tomto modulu možné nalézt specifické implementace jednotlivých bloků.

V dalším textu se budeme odkazovat na složky. Odkazujeme se tím do složek \TT{/Source/Blocks/Public} a~jejich \TT{Private} implementací. Strukturu bychom mohli shrnout následovně:

\begin{enumerate}
	\item Definice bloků (složka \TT{Definitions})
	\item Třídy s~popisem bloků (složka \TT{Info})
	\item Systém ukládání a~načítání bloků (složka \TT{Helpers})
	\item Rozhraní, které mohou bloky implementovat (složka \TT{Interfaces})
	\item Komponenty, kterými bloky rozšiřují svoji základní funkcionalitu (složka \TT{Components})
	\item Implementace jednotlivých bloků (složky \TT{BaseShapes}, \TT{Special})
	\item Stromové struktury herního světa (složka \TT{Tree})
\end{enumerate}
 

\subsection{Definice bloků}
V této složce se nachází všechny definiční soubory bloků. Definiční soubor obsahuje pouze popis datové struktury a~nějakou minimální funkcionalitu (kupříkladu získání korektního vektoru velikosti v~závislosti na tom, zda má definice daného bloku nastavenou vlastní velikost). Jednotlivé konkrétní instance s~daty jsou pak definovány na straně editoru. Konstanty (například minimální a~maximální škálování) je pak možné měnit v~editoru a~není vyžadována rekompilace projektu hry. 

Definiční soubor se skládá následujícím způsobem:

\begin{itemize}
	\item UBlockDefinition (\TT{BlockDefinition.h})
		\subitem -- FUsableBlockDefinition (\TT{UsableBlockDefinition.h})
		\subitem -- FBlockMeshStructureDefinition (\TT{BlockMeshStructureDefinition.h})
			\subsubitem -- FBlockMaterialDefinition (\TT{BlockMaterialDefinition.h})
		\subitem -- FBlockAdditionalFlags (\TT{BlockAdditionalFlags.h})
			\subsubitem -- FBlockFlagValue (\TT{BlockFlagValue.h})	
		\subitem -- FOxygenComponentDefinition (\TT{OxygenComponentDefinition.h})
		\subitem -- FElectricityComponentDefinition (\TT{ElectricityComponentDefinition.h})
			\subsubitem -- FElectricityBindableAreas (\TT{ElectricityBindableAreas.h})	
				
		

\end{itemize}

\subsection{Třídy s~popisem bloků}
Tyto třídy popisují už konkrétní instance bloků v~rámci hry. Jejich hodnoty jsou pak v~mezích definovaných v~definičních třídách. Tyto třídy jsou pak předmětem ukládání a~načítání. Dalším důležitým prvkem je BlockHolder, který slouží pro nalezení bloků. 

\subsection{Ukládání a~načítání bloků}

- ukládání -- máme něco jako block saving helpers


\subsection{Interfaces}
poskytují nástroje pro volání metod na instaních interfacu

popsat ideu za Implementation, Execute (BlueprintNativeEvent, BlueprintImplementableEvent)

\subsection{Komponenty bloků}
- pak máme komponenty bloků a~nějaké interfaces

\subsubsection{Elektrická komponenta}


\subsubsection{Elektrická síť}


\subsubsection{Kyslíková komponenta}


\subsubsection{Select target}


\subsubsection{World object}






\subsection{Implementace bloků}
 -- základ Block.h, zbytek v~jendotlivých podkategoriích (BaseShapes / Special
 
 TODO jak moc podrobné? vypsat všechny bloky a~co všechno implementují, nebo to stačí stručně zmínit? -- co implementují by si čtenář mohl uvědomit z~předchozího textu a~navíc je to jen nudný popis, jehož výžpovědní hodnota je ve zdrojácích a~není asi nutné to tu duplikovat

- popsat speciální bloky + nějaké speciality co umějí (showableWidget)


\subsection{Stromové struktury}
popsat stromové struktury, které tam mám


\subsubsection{MinMaxBox}

prapředek všeho



\subsubsection{KDTree}

dědí z~MMB, základ ve světě


\subsubsection{WeatherTargetsKDTree}

dědí z~MMB, slouží pro potřeby počasí