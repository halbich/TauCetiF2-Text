%!TEX root = ../../prace.tex

\section{Modul Commons (C++)}

Tento modul je základním modulem, který na jednom místě definuje všechny potřebné informace, které využívají ostatní moduly. Jedná se zejména o~definici herních konstant, či definice všech sdílených enumerátorů. Najdeme zde také prapředka použité herní instance. Tuto vlastní implementaci herní instanci využijeme pro ukládání nalezených bloků.

\subsection{Herní definice a~konstanty}



V souboru \TT{GameDefinitions.h}jsou definovány všechny herní konstanty. Například je zde definována velikost jednotkové krychle, velikost použitého světa, vztah mezi délkou dne herního světa a~počtem uplynulých reálných sekund, převody mezi energií, kyslíkem, zdraví a~jednotkou zásahu kyselého deště. Taktéž jsou zde definovány konstanty, které využívá technika obrysů objektu (viz \ref{subsec:interaction}). Dále jsou zde definovány konstanty ID implementovaných bloků, abychom s~nimi mohli pracovat i~v~kódu.

\subsection{Herní instance}

Herní instance \TT{TCF2GameInstance.h} se chová jako návrhový vzor \textit{Singleton} a~jako jediná zůstává vždy stejná po celou dobu běhu hry (jak uchovávat globální data~\citep{ue_gameInstance}). Proto se využívá například pro uchovávání dat při přechodu mezi jednotlivými Levely a~my toho také využijeme. Zároveň se tato třída dá využít pro implementaci delegátů, kterými je možné vyvolat nějakou událost a~libovolný prvek z~herního světa může tuto událost obsloužit. My toho využijeme u~reakce na denní cyklus u~bloku \textit{Přepínače}.

Dalším důležitým bodem pro nás bude, že tato třída si bude držet referenci na všechny nalezené bloky. Z předchozího textu již víme, že bloky je potenciálně možné rozšířit o~DLC (TODO budem to tu řešit?), takže je nutné, abychom si nalezené bloky a~jejich definice udrželi v~paměti i~při přechodu mezi levely. K tomu slouží proměnná \TT{BlockHolder}, která sice drží referenci na objekt definovaný v~modulu \textit{Blocks}, ale kvůli zpětným referencím mezi moduly (které nejsou povolené) musíme zde použít dostupného předka.

Kód tedy bude vypadat následovně:
\begin{code}
	UPROPERTY(Transient)
		UObject* BlockHolder;

	UFUNCTION(BlueprintCallable, Category = "TCF2 | GameInstance")
		void SetHolderInstance(UObject* holder);
\end{code}



 Parametr \TT{Transient} u~makra \TT{UPROPERTY} znamená, že daná proměnná bude vždy nastavena na svoji výchozí hodnotu. V tomto případě je to použito spíše z~důvodu zachování konzistence napříč projektu, ale zjednodušeně bychom důsledky mohli popsat následovně -- pokud bude nějaký Blueprint dědit z~nějaké \CPP{} třídy, tak vývojář může nastavit výchozí hodnoty properties. Tyto hodnoty jsou pak serializovány do \CDO{}, cože je \textit{Class Default Object} (otázka na Answers Unreal Engine~\citep{ue_cdo}). Během procesu vytváření nové instance objektu, který vychází z~daného Blueprintu pak budou tyto hodnoty naplněny během fáze inicializace properties (popis životního cyklu actorů~\citep{ue_actor_life}). V konečném důsledku by pak byla tato hodnota nějakým způsobem naplněna. Pokud chceme vynutit, aby tato property nebyla serializována do CDO, tak ji označíme jako \TT{Transient}.

V průběhu hry pak jednou naplníme tuto property pomocí metody\\ \TT{SetHolderInstance}, do které předáme referenci na korektně inicializovanou instanci třídy \TT{BlockHolder} z~modulu \textit{Blocks}. Pak si můžeme odkudkoliv získat aktuální herní instanci, přetypovat na \TT{TCF2GameInstance} a~získat si (přetypovanou) referenci na \TT{BlockHolder}. Z něho pak již můžeme získávat informace o~všech dostupných blocích.

\subsection{Enumerátory}

\TT{Enums.h}
Tento soubor slouží jako jednotné umístění pro všechny výčtové typy (enumerátory), které se používají napříč celým projektem. Neznamená to, že nutně obsahuje všechny -- některé třídy mohou využívat své specifické enumerátory, které ale nemusí být umístěny v~tomto globálně dostupném modulu.


\subsection{Helpery}

\TT{CommonHelpers.h} Tato třída poskytuje metody pro práci s~konfigurací. Statické metody umožňují načítat a~ukládat konfigurační položky typu \TT{float}, \TT{bool} a~\TT{string}.
Aby byla práce co nejjednodušší, metody přijímají enumerátor\linebreak[4]\TT{EGameUserSettingsVariable}. Třída pak sama na základě hodnoty tohoto enumerátoru použije správný klíč (který je textový) a~tak může ukládat či vracet hodnotu daného typu z~konfiguračního souboru.