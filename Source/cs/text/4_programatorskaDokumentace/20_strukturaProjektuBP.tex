%!TEX root = ../../prace.tex

\section{Struktura projektu v~Unreal Enginu}
\label{sec:ueStructure}

- ukázat jak se to dělí v~UE editoru, obrázky jak proudí informace, data flow

Struktura složek odpovídá /Content

\begin{itemize}
	\item Složka XY
\end{itemize}


TODO hodně obrázky, vztahy

\subsection{Mapy}

Základní částí projektu v~\UE{u} je \textit{herní mapa}. Ta obsahuje všechny důležité herní objekty.

popsat jednotlivé mapy, jaký je jejich cíl

- Game Loader

zavední hry

- MainMenu

menu, má sublevel MainMap (zobrazení bloků ve hře jako náhledu světa)

- Loading Screen

nahrávání, obsahuje mainmap jako streamovanej level

- Main map

\subsubsection{terén}

 (2 části), terén má speciální tag (označovatelnost, resp. získání cíle pro RayTracing)

\subsubsection{World Controller}

bloky ve světě, napojení na hlavní BP levelu

\subsubsection{GameElectricityPawn}

elektrika

\subsubsection{Game Weather Pawn}

počasí

\subsubsection{SkyBP}

denní cyklus

\subsubsection{AmbientSound}

řeší hudbu ve hře

\subsubsection{Tutorial}

řeší tutorial



\subsection{Mainmap -- průběh nahrávání}

popsat jak je to udělané

\subsection{Charakter}

výchozí spawn z~gamemodu, v~mainBP levelu Posses

\subsection{World Controller}

bloky ve světě, napojení na hlavní BP levelu


\subsection{bloky}

Kde a~jak definované, obrázek, jak se to definuje konstanty


\subsection{GameElectricityPawn}

elektrika, AI controller

\subsection{Game Weather Pawn}

počasí, AI controller, BT počasí + komponenty

\subsection{SkyBP}

denní cyklus -- popis

\subsection{AmbientSound}

řeší hudbu ve hře

\subsection{Tutorial}

řeší tutorial, používá widgety


\subsection{Lokalizacer}

nastínit lokalizaci



\subsection{UI}
nastínit UI -- základní widgety atd.



