%!TEX root = ../../prace.tex

\section{Modul Game Save (C++)}

Modul GameSave slouží k~ukládání a~načítání informací o~probíhající hře do binárního formátu. (TODO ref)  K~tomu používáme streamové operátory $<<$, které jsou v~tomto případě implementovány tak, že je možné je použít jak pro ukládání, tak pro načítání. // TODO link na tutorial

Díky tomuto přístupu tak můžeme definovat celou strukturu výsledného binárního souboru na jednom místě a~tedy rozšiřování uložené hry je triviální. Co si ovšem musíme pohlídat je to, abychom si drželi informaci o~verzi uloženého souboru. V~našem případě, pokud se bude lišit verze načteného souboru a~uložená konstanta v~programu, save prostě odmítneme (a dokonce smažeme). V~produkčním prostředí bychom si mazání nemohli dovolit, ale museli bychom save ignorovat a~uživateli zobrazit nějakou hlášku o~tom, že verze souboru není podporovaná. My jsme se však v~tomto případě rozhodli save mazat, protože jsme očekávali, že během vývoje hry se bude binární struktura savu často rozšiřovat. Po každé iteraci jsme si savy prostě vytvořili nové.

Zamysleme se nad tím, co by se stalo, kdybychom se snažili načíst save jiné verze. Celá hra by nejspíše byla ukončena s~chybou, protože by se pokoušela číst neplatná data a/nebo by očekávala nějaká data tam, kde žádná nejsou. Tím bychom četli z~neplatné lokace.




\subsection{GameSaveInterface}

Rozhraní pro ukládání a~načítání hry \TT{UGameSaveInterface.h} je možné implementovat a~používat v~Blueprintech (struktura vychází z~tutoriálu na Unreal Engine Wiki~\citep{ue_interfaces_tut}). Rozhraní definuje dvě veřejné metody, které musí actoři v~UE při implementaci tohoto rozhraní implementovat:

\begin{code}
UFUNCTION(BlueprintNativeEvent, BlueprintCallable, ... )
	bool SaveGame();

UFUNCTION(BlueprintNativeEvent, BlueprintCallable, ... )
	bool LoadGame();
\end{code}

Během načítání hry (TODO link) pak hlavní Blueprint hlavní mapy určité herní objekty přetypuje na toto rozhraní a~v tomto duchu s~nimi dále pracuje.

\subsection{FFileVisitor}
\label{subsec:ffv}
Pro získání všech herních savů z~nějaké zadané složky potřebujeme implementaci návrhového vzoru \textit{Visitor}, která je definovaná v~souboru \TT{FFileVisitor.h}. Implementace vychází z~internetové diskuse na téma procházení adresářů~\citep{ue_iterate_dir}.

\subsection{Helpers}
Smysl helperu \TT{SaveHelpers.h} je v~získání seznamu všech uložených her\linebreak (k~tomu využívá implementaci \textit{FileVisitoru}, kterou jsme zmínili v~předchozí části \ref{subsec:ffv}.

\subsection{Kontejner s~uloženou hrou}
Třída \TT{USaveGameCarrier} je v~zásadě přepravkou pro data s~možností ukládání a~načítání dat do binárního formátu. Navíc umožňuje během ukládání (při běhu v~Editoru) vypsat vlastnosti právě ukládané hry do konzole tak, abychom je mohli snadno zkopírovat a~tím snadno vytvořit pevně implementované hry. Tyto pevně implementované hry pak nebudou data vracet po přečtení nějakého binárního souboru, ale budou vracet pevně nastavená data. 

Přepravka obsahuje pouze holá data, takže se nevytvářejí žádné nové instance herních objektů. To je z~toho důvodu, že nemůžeme použít následující kód:

\begin{code}

// vlastnost kontejneru
TArray<UBlockInfo> UsedBlocks;

// během ukládání
void USaveGameCarrier::SaveLoadData(
	FArchive& Ar,
	USaveGameCarrier& carrier,
	TArray<FText>& errorList,
	bool bFullObject)
{
	Ar << carrier.UsedBlocks;
}
\end{code}
Museli bychom mít referenci na datový typ \TT{UBlockInfo}, který je ale definovaný v~modulu \textit{Blocks}, který musí nutně modul \textit{GameSave} referencovat. Tudíž zde použijeme následující konstrukci:

\begin{code}
// vlastnost kontejneru
TArray<FBlockInfo> usedBlocks;

// během ukládání
void USaveGameCarrier::SaveLoadData(
	FArchive& Ar,
	USaveGameCarrier& carrier,
	TArray<FText>& errorList,
	bool bFullObject)
{
	Ar << carrier.usedBlocks;
}
\end{code}

Třídu \TT{UBlockInfo} jsme nahradili strukturou \TT{FBlockInfo}, která sama slouží pouze jako přepravka na data. Vyšší vrstva, tedy modul \textit{Blocks} si při načítání hry těmito daty naplní své vlastní objekty, které následně využívá i~během hry. A~naopak, před samotným uložením se postará o~to, aby bylo toto pole korektně naplněno všemi daty určenými k~uložení. O~samotnou serializaci a~deserializaci dat do a~z~přepravky se stará pouze modul \textit{GameSave}. 

Tento systém je použit pro všechny rozšířené části uložené hry -- \textit{Bloky}, \textit{Inventář}, \textit{Počasí} -- a~všechny používají podobný způsob práce. Jedná se o~definici struktur, tedy samotných kontejnerů a~dále pak jeden soubor pojmenovaný \TT{*ArchiveHelpers.h}, kde je popsána vlastní struktura daných kontejnerů v~Archivu. Hvězdičku v~tomto případě bereme opravdu jako zástupný symbol. Navíc, některé objekty mohou řídit archivaci dle nějakých podmínek nastavených shora. Příkladem budiž definice serializace elektrické komponenty:

\begin{code}
// přetížení, které se nevolá vždy
FORCEINLINE FArchive& operator<<(
	FArchive &Ar,
	FPoweredBlockInfo& componentInfo)
{
	Ar << componentInfo.IsOn;
	Ar << componentInfo.AutoregulatePower;
	Ar << componentInfo.PowerConsumptionPercent;
	return Ar;
}

FORCEINLINE FArchive& operator<<(
	FArchive &Ar, 
	FElectricityComponentInfo& componentInfo)
{
	Ar << componentInfo.CurrentObjectEnergy;
	Ar << componentInfo.HasPoweredBlockInfo;

	// pokud máme navíc rozšiřující data, přidáme další data
	// tím efektivně zavoláme metodu uvedenou výše
	if (componentInfo.HasPoweredBlockInfo)
		Ar << componentInfo.PoweredBlockInfo;

	return Ar;
}
\end{code}

Z~kódu můžeme vidět, že se celý kód chová korektně jak při serializaci, tak i~při deserializaci. 

\subsection{NewGameSaveHolder}

Třída \TT{NewGameSaveHolder} je hlavní třídou, se kterou hra v~rámci nahrávání her pracuje. Definuje seznam napevno zabudovaných map a~také obsahuje jejich implementaci.









