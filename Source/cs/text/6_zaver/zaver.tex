%!TEX root = ../../prace.tex

\chapter{Závěr}

\section{Zhodnocení dotazníku}
\label{sec:quest}

Po dokončení implementace všech hlavních částí hry byl vytvořen a~zveřejněn dotazník. Kompletní zdrojová data je možné nalézt na (TODO). Bohužel musíme konstatovat, že je dotazník ovlivněn. Na základě průběžného vyhodnocování dotazníku byl zjištěn problém s~výkonem hry (TODO link do textu). Tento problém byl vyřešen a~hra byla optimalizována. Stále však hře chyběl tutoriál, který byl dokončen později. Prvotní implementace tutoriálu nebyla přijata kladně a~proto bylo přistoupeno k~úpravě tutoriálu do současné podoby. Od této úpravy se nám bohužel nepodařilo získat další odpovědi. Navíc musíme brát v~potaz fakt, že náš vzorek 24 respondentů nelze brát jako reprezentativní -- je prostě příliš malý. Navzdory tomu se pokusíme nějaké závěry vyvodit.

Podívejme se na celkové zhodnocení hry na obrázku \ref{fig:q7} (příloha \ref{sec:survey}). Z grafu je vidět, že hra byla hodnocena jako povedená (medián odpovědí je 7). Nicméně když se podíváme do odpovědí respondentů, kteří u~této otázky zvolili odpovědi 1 a~2, zjistíme, že si respondenti stěžovali na problémy s~výkonem. To bylo ještě před optimalizací hry. Věříme, že optimalizací hry by jejich odpovědi byly pozitivnější. 

Nyní se podívejme na obrázek \ref{fig:q6} -- hodnocení stavby z~různě velkých bloků. Vidíme, že respondenti tuto funkcionalitu vesměs hodnotili neutrálně. Odpovídá tomu i~medián odpovědí (5). Podívejme se podrobněji na odpovědi respondentů, kterým přišlo stavění jako nepříjemné a~odpověděli zde (2 -- 4). Vidíme, že se jedná o~zkušené hráče, většina jich má znalost stavitelských her. Stěžují si však na neoptimalizovanost hry, příliš malé bloky, absenci nastavení rychlosti myši a~neznalost hry (opět to jsou odpovědi před tutoriálem). Stejně jako v~předchozím odstavci tyto odpovědi chápeme jako reakci na nedostatky hry, z~nichž některé byly napraveny.

S výsledkem dotazníku jsme spokojeni. Ačkoliv jsme respondenty postavili do role \textit{testerů}, na jejich podněty jsme byli schopni reagovat a~celkově tak zlepšit hru. Díky negativním odpovědím jsme byli schopni identifikovat problémové části hry, což základní předpoklad pro rozvoj hry správným směrem.

\section{Zhodnocení práce}

Úspěšně se nám podařilo navrhnout a~implementovat herní mechaniky vzešlé z~cílů v~kapitolách \ref{chap:uvod} a~\ref{chap:analyza}. Ze zhodnocení dotazníku (kapitola \ref{sec:quest}) vyplývá, že myšlenka dynamicky škálovatelných bloků má svůj potenciál a~případná dokončená hra by si našla své publikum.

Ukázalo se, že implementace takto rozsáhlého projektu je pro jednoho člověka velice těžký oříšek. Vývojář musí mít dostatečně hluboké a~pokročilé znalosti \textit{herního designu}, \textit{2D} a~\textit{3D grafiky} a~samozřejmě \textit{programování}. Během vývoje jsme narazili na problémy, které jsou náplní práce specialistů na různých pozicích herního vývoje, ale nakonec se nám podařilo je nějak vyřešit.

Celkově hodnotíme práci jako povedenou a~její očekávaný přínos rozvoje stavitelských her byl naplněn.





\section{Budoucí práce}

AI a~další kraviny. TODO

\begin{itemize}
	\item dynamičtějí mřížka? 20cm je nejspíše dost málo a~vyžaduje to dost preciznosti // TODO zkusit pro test 25 či 30 cm a~patřičným způsobem upravit velikosti modelů? (nejspíše to musí zůstat hardcoded, ale zkusím se nad tím zamyslet, pokud bude čas)
	\item vlastní sortování v~seznamech

\end{itemize}

