%%% Hlavní soubor. Zde se definují základní parametry a odkazuje se na ostatní části. %%%

%% Verze pro jednostranný tisk:
% Okraje: levý 40mm, pravý 25mm, horní a dolní 25mm
% (ale pozor, LaTeX si sám přidává 1in)
\documentclass[12pt,a4paper]{report}
\setlength\textwidth{145mm}
\setlength\textheight{247mm}
\setlength\oddsidemargin{15mm}
\setlength\evensidemargin{15mm}
\setlength\topmargin{0mm}
\setlength\headsep{0mm}
\setlength\headheight{0mm}
% \openright zařídí, aby následující text začínal na pravé straně knihy
\let\openright=\clearpage

%\setcounter{secnumdepth}{5}

%% Pokud tiskneme oboustranně:
% \documentclass[12pt,a4paper,twoside,openright]{report}
% \setlength\textwidth{145mm}
% \setlength\textheight{247mm}
% \setlength\oddsidemargin{14.2mm}
% \setlength\evensidemargin{0mm}
% \setlength\topmargin{0mm}
% \setlength\headsep{0mm}
% \setlength\headheight{0mm}
% \let\openright=\cleardoublepage

%% Vytváříme PDF/A-2u
\usepackage[a-2u]{pdfx}

%% Přepneme na českou sazbu a fonty Latin Modern
\usepackage[czech]{babel}
\usepackage{lmodern}
\usepackage[T1]{fontenc}
\usepackage{textcomp}

%% Použité kódování znaků: obvykle latin2, cp1250 nebo utf8:
\usepackage[utf8]{inputenc}

%%% Další užitečné balíčky (jsou součástí běžných distribucí LaTeXu)
\usepackage{amsmath}        % rozšíření pro sazbu matematiky
\usepackage{amsfonts}       % matematické fonty
\usepackage{amsthm}         % sazba vět, definic apod.
\usepackage{bbding}         % balíček s nejrůznějšími symboly
			    % (čtverečky, hvězdičky, tužtičky, nůžtičky, ...)
\usepackage{bm}             % tučné symboly (příkaz \bm)
\usepackage{graphicx}       % vkládání obrázků
\usepackage{fancyvrb}       % vylepšené prostředí pro strojové písmo
\usepackage{indentfirst}    % zavede odsazení 1. odstavce kapitoly
\usepackage[numbers]{natbib}         % zajištuje možnost odkazovat na literaturu
			    % stylem AUTOR (ROK), resp. AUTOR [ČÍSLO]
\usepackage[nottoc]{tocbibind} % zajistí přidání seznamu literatury,
                            % obrázků a tabulek do obsahu
\usepackage{icomma}         % inteligetní čárka v matematickém módu
\usepackage{dcolumn}        % lepší zarovnání sloupců v tabulkách
\usepackage{booktabs}       % lepší vodorovné linky v tabulkách
\usepackage{paralist}       % lepší enumerate a itemize
\usepackage[usenames]{xcolor}  % barevná sazba

\usepackage[section]{placeins}	%sectioning



%% Vlastní makra
\newcommand{\MC}[1]{\textit{Minecraft#1}}
\newcommand{\SE}{\textit{Space Engineers}}
\newcommand{\ME}{\textit{Medieval Engineers}}
\newcommand{\TE}{\textit{Terraria}}
\newcommand{\TM}{\textit{Take on Mars}}
\newcommand{\NI}{\textit{Novus Inceptio}}
\newcommand{\PN}{\textit{Planet Nomads}}
\newcommand{\ARK}{\textit{ARK Survival Evolved}}
\newcommand{\NMS}{\textit{No man's sky}}

\newcommand{\XNA}{\textit{XNA}}
\newcommand{\MG}{\textit{Monogame}}
\newcommand{\OG}{\textit{Ogre}}
\newcommand{\UN}{\textit{Unity}}
\newcommand{\UE}[1][e]{\textit{Unreal Engin{#1}}}
\newcommand{\UEu}{\UE[u]{}}
\newcommand{\CRY}{\textit{CryEngine}}

\newcommand{\NPC}{\textit{NPC}}
\newcommand{\HUD}{\textit{HUD}}
\newcommand{\UBT}{\textit{UBT}}


\newcommand{\CS}{\texttt{C\#}}
\newcommand{\CPP}{\texttt{C++}}

\newcommand{\CDO}{\textit{CDO}}
\newcommand{\BP}[1]{\texttt{Blueprint#1}}


\newcommand{\TT}[1]{\texttt{#1}}


\newcommand{\CNF}{\textit{CNF}}


\DeclareUnicodeCharacter{2713}{\tick}
\DeclareRobustCommand\tick{%
  \unskip\nobreak\thinspace\textemtick\allowbreak\thinspace\ignorespaces}



%% Two counters to keep track of categories and 
%% subcategories
\newcounter{mycategorycounter}
\newcounter{subcategorycounter}

\newcommand{\mytablerow}{%%'
  & \stepcounter{subcategorycounter}%%'
    \Alph{mycategorycounter}\arabic{subcategorycounter}.}

\newcommand{\tableColumnTitles}[2]{%%'
  &\multicolumn{2}{l}{#1}#2 \\}
%% create the category names; rules are here to add open space
%% above and below entry.  I use `\rlap` to alow the category
%% name to apparently span multiply columns
\newcommand{\currentCategory}[1]
  {\rule{0pt}{3ex}%%'
   \rule[-1ex]{0pt}{1pt}%%'
   \setcounter{subcategorycounter}{0}%%'
   \stepcounter{mycategorycounter}%'
   \Alph{mycategorycounter} & \rlap{#1} & & & & & & }


% Load the package with the acronym option
%\usepackage[acronym, nonumberlist, nopostdot, toc]{glossaries}

% Generate the glossary
%\makeglossaries
%\makenoidxglossaries
%\newacronym{ue}{UE}{Unreal Engine}
\newacronym{ubt}{UBT}{Unreal Build Tool}

\newacronym{bt}{BT}{Behavior Tree}
\newacronym{hud}{HUD}{Head-up display}

\newacronym{npc}{NPC}{Non-playable character}

%%% Údaje o práci

% Název práce v jazyce práce (přesně podle zadání)
\def\NazevPrace{Tau Ceti f 2 -- budovatelská počítačová hra se strategickými prvky}

% Název práce v angličtině
\def\NazevPraceEN{Tau Ceti f 2 -- A Creative Computer Game with Strategic Elements}

% Jméno autora
\def\AutorPrace{Pavel Halbich}

% Rok odevzdání
\def\RokOdevzdani{2017}

% Název katedry nebo ústavu, kde byla práce oficiálně zadána
% (dle Organizační struktury MFF UK, případně plný název pracoviště mimo MFF)
\def\Katedra{Katedra distribuovaných a spolehlivých systémů}
\def\KatedraEN{Department of Distributed and Dependable Systems}

% Jedná se o katedru (department) nebo o ústav (institute)?
\def\TypPracoviste{Katedra}
\def\TypPracovisteEN{Department}

% Vedoucí práce: Jméno a příjmení s~tituly
\def\Vedouci{Mgr. Pavel Ježek, Ph.D.}

% Pracoviště vedoucího (opět dle Organizační struktury MFF)
\def\KatedraVedouciho{Katedra distribuovaných a spolehlivých systémů}
\def\KatedraVedoucihoEN{Department of Distributed and Dependable Systems}

% Studijní program a obor
\def\StudijniProgram{Informatika }
\def\StudijniObor{Programování a softwarové systémy}

% Nepovinné poděkování (vedoucímu práce, konzultantovi, tomu, kdo
% zapůjčil software, literaturu apod.)
\def\Podekovani{%
Děkuji mému vedoucímu Mgr. Pavlu Ježkovi, Ph.D., za pomoc s touto prací, mým rodičům za podporu a pevné nervy, mé přítelkyni Veronice taktéž za podporu a pomoc s 2D grafikou a Jiřímu Kurčíkovi za laskavé poskytnutí práv na použití jeho hudební tvorby v mé hře. 
}

% Abstrakt (doporučený rozsah cca 80-200 slov; nejedná se o zadání práce)
\def\Abstrakt{% 
Mnoho hráčů počítačových her má v~oblibě žánr stavitelských her. Mezi mnohými bychom mohli jmenovat hry \MC{} a~\SE{}. V~těchto hrách hráč staví budovy a~struktury z~bloků pevně dané velikosti. To shledáváme omezujícím a~proto se v~této práci zabýváme novým konceptem, které současné hry nenabízí -- \textit{stavěním} z~\textit{dynamicky škálovatelných} bloků. Cílem je zpříjemnit hráčův zážitek ze hry a~zrychlit stavění rozsáhlých staveb. Hráč však takto může vytvořit velké množství nových bloků, a proto se v~této práci také zabýváme \textit{automatizovanou správou inventáře} bloků, aby hráč zbytečně neztrácel čas hledáním bloků ke stavbě. Tyto herní mechaniky jsme implementovali do nově vzniklé hry \textit{TauCetiF2}. Pro vývoj naší hry jsme zvolili \UE{}, díky čemuž jsme mohli využít rychlosti \textit{C++} a~zároveň přívětivosti technologie \textit{Blueprintů}. Vhodnou kombinací těchto přístupů jsme dosáhli rychlého a~efektivního vývoje celé hry. Z~dotazníku, který byl vytvořen za účelem ověření pochopitelnosti a~zábavnosti těchto  mechanik vyplynulo, že se tyto mechaniky hráčům líbí. Očekávaný přínos této práce byl naplněn a~získali jsme nové poznatky, jak tyto mechaniky vylepšit. 
}
\def\AbstraktEN{%
TODO Many computer game players likes genre of building games. In these games player builds buildings and structures using blocks of a fixed size. We find it restrictive and therefore we are dealing with a new concept, unused in current games -- \textit{building} from \textit{dynamically scalable} blocks. The goal is to make player's experience more enjoyable and to speed up the construction of extensive buildings. Since player can create a lot of new blocks, we are also dealing with \textit{automated inventory management} for a blocks so the player does not waste time searching for blocks to build. These game mechanics have been implemented in the newly created game called TauCetiF2. From the questionnaire that was created to verify the understanding and fun of these mechanics, it turned out that the players like these mechanics. The expected benefit of this work has been fulfilled and we have gained new insights into how these mechanics could be improved.
}

% 3 až 5 klíčových slov (doporučeno), každé uzavřeno ve složených závorkách
\def\KlicovaSlova{%
{Stavitelská hra,} {dynamicky škálovatelné bloky,} {Unreal Engine}
}
\def\KlicovaSlovaEN{%
{Building game,} {dynamically scallable blocks,} {Unreal Engine}
}

%% Balíček hyperref, kterým jdou vyrábět klikací odkazy v PDF,
%% ale hlavně ho používáme k uložení metadat do PDF (včetně obsahu).
%% Většinu nastavítek přednastaví balíček pdfx.
\hypersetup{unicode}
\hypersetup{breaklinks=true}

%% Definice různých užitečných maker (viz popis uvnitř souboru)
\include{makra}

%% Titulní strana a různé povinné informační strany
\begin{document}
\include{titulka}



%%% Strana s automaticky generovaným obsahem bakalářské práce

\tableofcontents

%%% Jednotlivé kapitoly práce jsou pro přehlednost uloženy v samostatných souborech
%!TEX root = ../prace.tex

\chapter*{Úvod}
\addcontentsline{toc}{chapter}{Úvod}

V době vzniku této práce jsou velice populární hry s otevřeným světem a velkými možnostmi stavění. Autor této práce si je taktéž rád zahraje a rád by touto prací představil svoji vizi dalšího možného rozvoje tohoto žánru.

Když se podíváme na hry jako například Minecraft, Space Engineers (nebo její odnož Medieval Engineers) tak zjistíme, že hra od hráče vyžaduje nejen stavitelské a konstruktérské schopnosti, ale také taktické dovednosti, které hráči umožňují v daném světě přežít. Mnohdy hráč může využít nevšedních technik daných mechanikami dané hry. Příkladem budiž farma na golemy ve hře Minecraft, která využívá bloky lávy, které jsou drženy cedulemi\citep{minecraft_tut_farm}.

V současné době jsou velikosti bloků omezeny na konstantné velikost. Ve hře Minecraft je blok hranově omezen na 1m, hra Space Engineers bloky omezuje dle kategorií od 0.5\,\rm m do 2.5\,\rm m \citep{se_blocks_wiki}.
My bychom se v této práci chtěli zabývat myšlenkou proměnlivé velikosti stavitelných bloků. Hráči by tak mohli ovlivňovat detailnost svých výtvorů, aniž by to muselo mít nutně negativní vliv na dobu nutnou k postavení komplexního, ale zároveň detailního výtvoru. Tato myšlenka však přináší spoustu problémů k řešení, které se však v této práci snažíme vyřešit.

Aby hra byla nebyla pro hráče nudná, měla by také obsahovat nějakého nepřítele - tedy něco, co bude hráče nutit se zlepšovat, překonávat nástrahy a posouvat se dále. I tomuto elementu hry se v této práci budeme věnovat. Zároveň nám to nabízí nové možnosti, jak do hry dodat prvky strategických her. Aby to hráč neměl tak jednoduché, bude muset taktizovat, aby svého nepřítele porazil, nebo alespoň v dané chvíli neprohrál, byť za cenu nějakých ztrát.

\begin{itemize}
	\item V poslední době jsou kromě klasických her typu FPS také různé strategie a hry o přežití
	\item Typ: Minecraft, Space Engineers, Medieval Engineers, Take on Mars, ARK Survival Evolved, novější No man's sky
	\item autor tyto hry má též v oblibě a rád by představil svoji vizi budovatelské hry s prvky strategie a přežití		// TODO tohle by chtělo hodně učesat
\end{itemize}		

// TODO zmínit příběh, který jsem si vymyslel? (cesta za záchranou lidstva atd...)? \\

// možná bych to mohl dát do hry na úvod, poté zobrazit textový tutorial
%!TEX root = ../../prace.tex

\chapter{Analýza zadání}

V této části provedeme rozbor toho, jak různé hry v současné době přistupují k řešení jednotlivých součástí hry. Tím si připravíme prostor pro specifikaci toho, jak by naše hra mohla vypadat a co všechno by měla umět.

// TODO remove me:
Úkol zněl jasně: Cílem bakalářské práce je implementace budovatelské hry se strategickými prvky, hranou z pohledu třetí osoby. Hra se odehrává na nehostinné planetě, kde je hráčův úhlavní nepřítel nedostatek zdrojů a superkyselé deště. Hráč začíná v menší budově – zbytek přistávacího modulu kosmické lodi. Dochází mu elektrická energie i kyslík a je na hráči, aby takticky využíval dostupné zdroje, hledal nové možnosti výroby energie a přežil kyselé bouře. Cílem práce není vytvořit dohratelnou hru, spíše proof-of-concept, zda je tento typ hry s uvedenými mechanikami zábavný a má smysl v jejím vývoji pokračovat i nadále.

%!TEX root = ../../prace.tex


\section{Stávající implementace mechanismů}

Rozebereme si, jakým způsobem je v~současné době přistupováno k~blokům a~jak jsou tyto bloky umisťovány do herního světa. Dále nás budou zajímat ostatní herní mechaniky, jako například denní cyklus, nebo jakým způsobem je řešena herní postava, její inventář a~možnosti hráče interakce se světem. Poté se rozhodneme, jakým způsobem budeme uvedené mechaniky řešit my.

\subsection{Bloky}
\label{subsec:blocks}

Pod pojmem \uv{blok} budeme chápat objekt, který je umístěn v~nějaké ortogonální mřížce 3D prostoru a~beze zbytku tento prostor vyplňuje. Bloky spolu sdílí společnou množinu vlastností a~dále pak samy definují své vlastní vlastnosti, které vychází z~povahy bloku, tedy toho, co daný blok reprezentuje. V této kapitole rozebereme tyto vlastnosti a~následně definujeme, jaké problémy budeme muset v~této práci řešit.


\subsubsection{Základní vlastnosti}

Mezi základní vlastnosti bloku bychom měli zařadit \textit{vizuální reprezentaci}, \textit{pozici ve světě}, \textit{rotaci} a~\textit{velikost}. Tento výčet není kompletní, ale můžeme říct, že jsou pro hráče nejvíce viditelné. Podle typu hry, jejího stylu a~celkové koncepce bychom pak mohli do této množiny přidat třeba \textit{zdraví bloku}. V této části se zaměříme právě na základní vlastnosti a~rozšiřujícím vlastnostem se budeme věnovat v~následujících částech této kapitoly.

Vizuální reprezentace zřejmě vychází z~povahy bloku. Vždy ale platí, že se vždy musí vejít do nějakých hranic bloku, obvykle daných v~rámci pravidelné mřížky, ve které se blok nachází. Může v~sobě zahrnovat i~nějaké informace, které jsou pak hráčem vnímány pohledem. Kupříkladu mohou různými způsoby indikovat vnitřní stav objektu, podobně jako třeba svítivé diody indikují různé stavy elektronických zařízení v~reálném světě. Podrobněji se budeme vzhledem zabývat až budeme řešit konkrétní bloky v~naší hře.

Pozice bloku v~herním světě je u~her z~kapitoly \ref{chap:uvod} může být buď na jednoznačně definovaném místě v~herním světě, přičemž všechny bloky mají vůči sobě stejnou rotaci (\MC{}), nebo je blok součástí nějaké skupiny\footnote{Za skupinu budeme považovat shluk alespoň jednoho bloku a~všechny bloky ve skupině musí být ve stejné mřížce.} bloků s~jednoznačnou pozicí v~herním světě. Tato skupina pak může mít libovolnou rotaci vůči nějakému globálnímu souřadnicovému systému a~můžeme říct, že tato skupina tvoří lokální ortogonální systém. Různé skupiny mohou být vůči sobě různě natočeny. Toto chování je možné pozorovat třeba ve hře \ME{}, kde postavením prvního bloku určíme pozici a~rotaci této skupiny. Pokud budeme chtít přidat do světa další blok, do nějaké vzdálenosti od prvního bloku je tento nově stavěný blok stále přichytáván do mřížky definované prvním blokem a~rotace bloku jsou vždy o~90~stupňů. Obvykle platí, že první blok je umístěn vodorovně (tedy krychle bude i~na šikmém terénu umístěna tak, že její horní a~dolní strana je ve vodorovné poloze) a~je možné ho libovolně rotovat kolem vertikální osy.

Rotace bloků jsou u~naprosté většiny her řešeny v~zásadě podobně. Výjimkou je pouze \MC{}, který rotaci bloků neumožňuje. U většiny bloků to nevadí, protože jsou bloky umístěny po směru, ze kterého je blok umisťován. Můžeme však nalézt některé speciality, třeba blok \textit{kolejí} (oficiální popis bloku~\citep{mc_rail}). Koleje v~\MC{u} mohou být rovné (ve směru sever -- jih nebo východ -- západ), mohou být nakloněné a~překonávat výškový rozdíl mezi dvěma bloky, nebo mohou tvořit zatáčku. Díky absenci rotací tak hra používá několik pravidel, které určují výslednou podobu bloku. Podrobněji se jimi však zabývat nemusíme. U zbývajících her je situace, kterou jsme popsali v~úvodu této části -- první blok je natočen vodorovně, má libovolnou rotaci dle vertikální osy a~tímto určuje přichytávací mřížku pro další stavěné bloky.

Hry mají velikost bloků vždy stejně velkou. \MC{} má hranu bloku o~délce $1\,\rm m$ (popis bloku na oficiální Wiki stránce Minecraftu~\citep{mc_block}, oficiální popis jednotek použitých v~Minecraftu~\citep{mc_units}). \SE{} sice bloky hranově omezuje dle kategorií od $0,5\,\rm m$ do $2,5\,\rm m$ (oficiální popis bloků ve hře~\citep{se_blocks_wiki}), ale tyto kategorie mezi sebou nelze kombinovat a~je možné k~sobě vázat pouze bloky z~jedné a~té samé kategorie. U ostatních her je situace podobná, byť některé jsou v~raných fázích vývoje a~tudíž pro ně neexistuje žádný oficiální zdroj informací, takže velikosti bloků bychom mohli pouze odhadovat. 

Jako další základní vlastnosti bloku bychom mohli brát třeba \textit{zdraví}, \textit{cenu za postavení}, \textit{čas doby stavby}, \textit{délku trvání destrukce} (pokud vychází z~hráčovy akce), co hráč získá za zničení bloku apod. U doby konstrukce a~destrukce bloku můžeme říct, že doba trvání dané akce závisí na parametrech bloku a~použitém nástroji. Kupříkladu v~\MC{u} trvá kutání bloku kamene \textit{diamantovým} krumpáčem podstatně rychleji, než při použití krumpáče \textit{dřevěného}. Taktéž rychlost opotřebení nástrojů je různá. Nicméně když se ve hrách přepneme do tzv.~\textit{kreativního} módu, pak máme stavbu bloků \uv{zadarmo} a~ihned. Tedy nemusíme mít žádné komponenty ke stavbě, ani specializovaný nástroj (případ \SE{}, \ME{}), nebo nám při postavení bloku tyto bloky neubývají z~inventáře (\MC{}). 

\subsubsection{Součásti bloků}
Jako součásti bloků můžeme brát cokoliv, co rozšiřuje základní vlastnosti. Zde bychom mohli zmínit třeba \textit{Redstone} v~\MC{u}. Redstone je speciální typ bloku, chová se jako elektrický vodič a~v~kombinaci s~jinými bloky je možné vytvářet logická hradla. Je zřejmé, že pokud vhodným způsobem zkombinujeme určitá hradla, je možné v~\MC{u} vytvořit třeba bitovou sčítačku. Ačkoliv vytvoření jednoho hradla je snadné, složitější logické obvody (například kódový zámek) jsou pak velmi náročné na prostor. Nejen z~tohoto důvodu pro \MC{} vnikly různé módy, rozšiřující základní funkcionalitu hry o~nové elektrické komponenty. Pro zajímavost, mód \textit{RedPower} (blog autora~\citep{eloraam}) má jednotlivá hradla jako samostatné bloky (což šetří místo) a~navíc má možnost skládat různě barevné vodiče (což jsou pouze ekvivalenty Redstonových bloků) do sebe (až 16~linek signálu). Bez tohoto módu by hráč potřeboval takový prostor, aby položil 16~vedení Redstone tak, aby se nedotýkaly a~vzájemně neovlivňovaly (na rovině by tyto vodiče zabíraly šířku 31~bloků). 

Další možné součásti bloků, kromě vedení elektřiny, může být například práce s~kyslíkem či práce s~inventářem. Některé bloky hráči nabízí nějaký úložný prostor, kam může přesunovat objekty ze svého inventáře. Později pak hráč může k~bloku přijít a~objekty si opět navrátit do svého inventáře. Zajímavou specialitou je náhled inventáře. Kupříkladu u~hry \ME{} se jedná stůl a~jídlo na stole. Stůl má svůj inventář, do kterého je možné umisťovat jídlo. Ve hře je pak obsah inventáře stolu zobrazen tak, že jídlo má svůj náhledový model na talířích na stole, takže to vypadá, jako by bylo jídlo připraveno ke konzumaci. 


Dále můžeme zmínit interakci s~uživatelem, ať už přímou, nebo nepřímou. Jako přímou interakci budeme chápat takové použití bloku, kdy rovnou vidíme nějakou změnu. To může být například stisknutí tlačítka, nebo změna polohy nějaké páky. Výsledek této přímé interakce pak hráč vidí okamžitě a~vizuálně se blok nějakým způsobem změní. Jako nepřímou interakci bychom mohli uvést například otevření nějakého ovládacího rozhraní bloku, což je obvykle nějaká UI obrazovka. Obě interakce se mohou prolínat, takže výsledkem nepřímé interakce může být třeba změna barvy bloku.


\subsection{Komunikace bloků}
Bloky spolu mnohdy umí komunikovat. Dříve zmíněný Redstone z~\MC{u} by se dal taktéž považovat za jistou metodu komunikace mezi bloky. Například stiskem tlačítka lze změnit výstupní Redstonový signál z~tohoto tlačítka (třeba z~neaktivního na aktivní), který způsobí změnu polohy nějakého pístu. Ovšem komunikace může být i~méně viditelná -- například z~terminálu v~\SE{} je možné ovládat písty, otevírat a~zavírat dveře hangáru apod. bez explicitních vodičů signálu.

Ve hrách \MC{} a~\SE{} můžeme nalézt příklad dopravníkových systémů. Ty umožňují \uv{poslat} bloky či objekty na jiné místo, kupříkladu z~jednoho křídla budovy do druhého. V \MC{u} bez módů se této funkcionality dá vcelku snadno dosáhnout, nicméně je to zdlouhavé a~rychlost přesunu bloků je poměrně pomalá. Ale opět se můžeme obrátit na módy, třeba na \textit{BuildCraft} (oficiální stránky projektu~\citep{buildcraft}), který umožňuje rychle dopravovat bloky i~na velké vzdálenosti. Hra \SE{} má kvalitní dopravníkový systém už ve svém základu a~slouží například pro dopravu natěženého materiálu od bloku \textit{Vrtáku} do bloku \textit{Skladiště} (který má svůj inventář). Svým způsobem je pak přeprava v~rámci těchto systémů také druhem meziblokové komunikace - bloky si mezi sebou předávají konkrétní instance objektů.

\subsection{Skládání bloků do struktur}
Asi jediné příklady, který můžeme zmínit, jsou ve hře \MC{} -- postavení portálu do Netheru (obrázek \ref{fig:structs_nether_portal}), sněhuláka či golema. V momentě, kdy nějaká skupina bloků splňuje přesně definovaný tvar, tak se tyto bloky chápou jako jedna struktura a~tak se s~nimi i~zachází. Portál je možné pomocí \textit{křesadla} aktivovat, ale pokud je portál aktivní a~hráč odebere z~portálu některý blok, který jej tvoří, portál se uzavře. Bloky ve tvaru sněhuláka a~golema se po dokončení tvaru odeberou a~namísto nich je do světa umístěno \NPC{} sněhuláka či golema.

\begin{figure}[!ht]\centering
\includegraphics[ width=140mm]{../img/analysis/mc_portal}

\caption{Portál do Netheru. Zdroj: minecraftpocketedition.wikia.com~\citep{mc_nether_portal}}
\label{fig:structs_nether_portal}

\end{figure}

\FloatBarrier



\subsubsection{Speciality}
Ve hrách \ME{} a~\NMS{} můžeme nalézt zajímavou funkcionalitu \textit{multibloků}. Tato funkcionalita nabízí například nahrazení nějaké stěny nějakého bloku oknem či nějakým dalším vizuálním či funkčním elementem.


Také bychom měli zmínit propagaci kyslíku v~rámci postavených budov, kterou můžeme nalézt ve hrách \SE{} nebo \TM{}. Ačkoliv tento koncept je spíše vlastností herního prostředí, velice úzce souvisí s~bloky. Pokud bloky tvoří uzavřenou strukturu, tak je možné vnitřní prostory natlakovat, zaplnit kyslíkem a~sundat si helmu či dokonce celý skafandr. \SE{} využívá systém tzv.~\uv{místností} (oficiální dokumentace o~kyslíku~\citep{se_oxygen}). Místnosti lze zaplnit kyslíkem díky \textit{ventilaci}, což je blok, který umí do prostoru místnosti vhánět kyslík a~také ho z~něj odebírat. \TM{} dělá prakticky to samé, liší se pouze způsobem implementace rozpoznání místnosti, což souvisí se způsobem stavby bloků (jak jsme zmínili v~kapitole \ref{chap:uvod}).


\subsection{Herní svět}

V této části se zaměříme na zpracování herního světa a~s~tím souvisejících mechanik.

\subsubsection{Reprezentace}

Pokud shrneme poznatky z~kapitoly \ref{chap:uvod}, tak můžeme říct, že máme v~zásadě dvě možnosti. Buď můžeme po vzoru \MC{} mít celý svět tvořen bloky. Pak bychom kvůli optimalizaci výkonu také museli tyto bloky shlukovat do skupin (v \MC{u} se jim říká \textit{chunks} a~jsou velké $16*16*256$ bloků). Nebo můžeme mít po vzoru \SE{} svět více realistický -- a~třeba po vzoru této hry hráči připravit třeba celé planety (oficiální představení přidání planet do hry~\citep{se_planets}). Koncept planet se objevuje i~v~dalších hrách (\ME{}, \NMS{}, atd.).

\subsubsection{Bloky v~herním světě}

Jak jsme se zmínili v~části \ref{subsec:blocks}, bloky jsou v~herním světě umístěny vždy do nějaké mřížky. Různé hry však nabízí různé možnosti práce s~těmito mřížkami a~ta navíc vychází z~možností reprezentace herního světa.


\subsubsection{Denní / noční cyklus}

Všimněme si, že všechny hry implementují denní cyklus. To dodává hře na dynamičnosti -- ne všechny hráče by bavilo desítky či stovky hodin trávit v~identickém prostředí. Navíc, pro vývojáře je to možnost, jak hráči připravit nějaké nebezpečí spojené s~příchodem noci. Jeden příklad za všechny -- v~\MC{u} se mnozí nepřátelé začínají objevovat, až když je pro ně dostatečná tma. Což nemusí být nutně způsobeno začátkem noci, stačí třeba bouřkové počasí, málo osvětlená jeskyně ve skále či podzemní prostory. Ovšem denním cyklem je hráč nucen se nepřátelům bránit a~tím je pro něj zajištěn nějaký netriviální herní prvek, po jehož překonání má hráč obvykle radost a~pocit úspěchu.
Délka denního cyklu se různí podle hry. \MC{} má délku dne 20 minut, u~dalších her to je podobné.


\subsubsection{Počasí}
Zážitek ze hry umocňuje například \textit{proměnlivé počasí}. Stejně jako u~denního cyklu, změna počasí do hry přináší prvek náhody a~neopakovatelnosti hraní. Můžeme říci, že všechny námi zmíněné hry z~kapitoly \ref{chap:uvod} počasí řeší (v závislosti na jejich herním stylu). 

\subsubsection{(Ne)fyzikální chování}

Za zmínku stojí také přítomnost fyziky ve hře. Hry s~vyšším či úplným stupněm realismu ji implementují a~u~každé hry se více či méně přibližuje realitě. Vymyká se hra \MC{}, jejíž fyzikální model se může zdát na první pohled zvláštní, ale je zapotřebí si uvědomit, že tento model zapadá do celkové koncepce hry. Proto je v~této hře poměrně zajímavá mechanika kapalin, která je zcela nefyzikální a~umožňuje vytvoření nekonečných zdrojů vody. Ostatně sám blok zdroje vody je možné umístit z~boku k~nějakému sloupu a~voda bude neustále vytékat. Pokud navíc odebereme bloky ze sloupu, voda bude vytékat z~ničeho a~bude z~daného místa vytékat dokud dané místo nenahradíme nějakým pevným blokem, třeba hlínou či pískem. Stejně jako voda se ve hře chová i~láva, její chování se liší v~drobných detailech (například se šíří pomaleji než voda).

Další zajímavou vlastností je \uv{levitace} bloků. Trochu jsme na toto chování narazili v~předcházejícím odstavci, tak to rozveďme. Každý blok v~\MC{u} obývá právě jednu pozici v~herním světě. A u~většiny bloků platí, že mohou zůstat viset volně ve vzduchu. Ale například u~\textit{písku} se po přidání nebo odebrání některého z~okolních bloků provede \textit{aktualizace} bloku. A když blok po aktualizaci zjistí, že by neměl být jen tak ve vzduchu (pod ním nic není), tak začne padat dolů. A jak nastane situace, že hráč potká ve vzduchu visící písek? To souvisí s~algoritmem generování herního světa, který má více cyklů a~ve výsledku mohou být v~herním světě umístěny obdobné kuriozity. Nicméně stačí jeden přidaný či odebraný blok v~těsné blízkosti bloku podléhajícímu aktualizaci a~dle situace se bloky začnou kaskádovitě aktualizovat, takže v~případě \textit{písku} se začnou sypat dolů. Tam se bloky opět úhledně seskládají do sloupců (což u~\textit{písku} není standardní fyzikální chování --v reálném životě by se písek sesypal na hromadu).


\subsection{Inventář}
\label{subsec:inventory}

Obdobně jako u~fyziky ve hrách, chování inventáře je závislé na stupni realismu. \MC{} nabízí 27 slotů inventáře, 9 slotů rychlé nabídky a~4 sloty pro brnění. Sloty inventáře a~rychlé nabídky jsou plněny \textit{kupitelnými předměty}. To znamená, že podle typu bloku či předmětu může být v~daném slotu 1 až 64 prvků daného objektu. Takže například je možné mít v~jednom slotu 64 kusů \textit{hlíny}, ale jen 16 kusů \textit{vajec} a~pouze 1 \textit{meč}. Plnění rychlé nabídky je možné z~UI nabídky inventáře pomocí systému Drag and Drop. Navíc za použití kláves Ctrl a~Shift lze řídit počet přenášených kusů daného objektu. Při stavění se blok z~daného rychlého slotu odebere, zbraň či nástroj se opotřebí.

U \ME{} a~\SE{} jsou také pevné sloty, ale existuje zde koncept \textit{skupin slotů}. Přepínáním těchto skupin se mění všechny sloty rychlé nabídky, takže je možné si do různých skupin nastavit různé stavební prvky. Můžeme říct, že \MC{} má pouze jednu skupinu slotů. V momentě, kdy hráč staví z~více prvků než je velikost rychlé nabídky, se tato funkcionalita stává potřebnou, aby hráč nemusel neustále přehazovat stavěné bloky. Výhodou je, že u~těchto her jsou bloky spíše konstrukčního charakteru a~neodebírají se z~inventáře, takže prvky ve skupině slotů zůstávají zachovány. V \MC{u} by i~při této funkcionalitě bylo potřeba udržovat skupiny slotů, protože jak jsme již zmínili, stavba v~této hře odebírá položky z~inventáře. Ve hře jsou však objekty (třeba \textit{jídlo}), které jsou \textit{kupitelné} a~jejich chování je stejné jako u~\MC{u}, včetně řízení počtu přenášených položek za pomocí stisknutých kláves během přenosu do inventárního slotu.

Co se týče stavění, zbývající hry nepřináší nic nového či převratného. Pozorný čtenář si mohl všimnout, že v~\MC{u} je možné, aby hráč nesl $36*64$ kostek kamene, každou o~objemu $1\,\rm m^3$. Je zřejmé, že nosnost opět nevychází z~realistických předpokladů. Nicméně u~her s~vyšším stupněm realismu se u~nosnosti setkáváme, takže kupříkladu je omezena váha, kterou může hráčova postava nést. Navíc se u~některých her s~rostoucí zátěží zpomaluje pohyb. I toto je součástí herního designu, který větší či menší měrou odráží fyzikální realitu.

\subsection{Herní postava}
Mezi vlastnosti herní postavy bychom mohli zmínit \textit{zdraví}, \textit{výdrž}, \textit{hlad}, zbývající \textit{zásoba kyslíku} či \textit{energie} a~další podobné charakteristiky. Tyto vlastnosti jsou vlastně \textit{survival} prvky a~hráč se musí o~svoji herní postavu \uv{starat}, aby mu neumřela a~aby mohl hru hrát dál. Zde opět můžeme říct, že tyto herní mechaniky hry z~kapitoly \ref{chap:uvod} implementují podobným způsobem a~výsledný dojem z~nich není u~žádné z~her natolik zásadní a~inovativní, abychom se tím hlouběji zabývali..
Další vlastností, kterou jsme již zmínili, je nosnost v~rámci inventáře postavy. Podrobněji jsme tuto mechaniku rozebrali v~části \ref{subsec:inventory} a~proto se tím zde nebudeme podrobněji zabývat. 
Dále bychom mohli zmínit možnost přepínání pohledů (pohled z~1. osoby nebo z~3. osoby), což většina her také má. 





%!TEX root = ../../prace.tex



\section{Co bychom chtěli implementovat}

V následujících podkapitolách si rozebereme naše požadavky na hru



%!TEX root = ../../prace.tex

\subsection{Bloky}

TODO  Většina bloků je stejně velká a~má hranu o~délce 1 metru \citep{mc_block}, \citep{mc_units}. (TODO přesunout do detailní analýzy. TODO popisek odkazu)

V současné době jsou velikosti bloků omezeny na konstantné velikost. Ve hře Minecraft je blok hranově omezen na 1m, hra Space Engineers bloky omezuje dle kategorií od 0.5\,\rm m do 2.5\,\rm m \citep{se_blocks_wiki}.


různé druhy, velikosti, jejich vizuální reprezentace, rozšiřovatelnost, obecně co všechno by měly umět.

Název - Min - Max - Pitch - Roll - Type (kostka, zkosený, roh, vlastní)

Tam kde Min == Max -> Vlastní škálování

Typ ovlivňuje další chování

Třeba u~Světla by typ mohl být i~K a~hra by se chovala stejně, K = 1, Z = 0.5, R = 1/6, V = 1 (není v~potaz objem)

komponenty bloků a~nějaké další ptákoviny

\begin{tabular}{|rll*{5}{c}|}
	\hline
	\tableColumnTitles{Název}								{	&	Min		&	Max			&	P			&	R			&T	}		\hline
	\currentCategory{\textbf{Základní bloky}} 																					\\		\hline
		\mytablerow 				& Blok základny				& 1--1--4	& 20--20--4		& 				& 				&K	\\		\hline
		\mytablerow 				& Blok stavby				& 1--1--1	& 20--20--20	& \checkmark	& \checkmark	&K	\\		\hline
		\mytablerow 				& Blok polykarbonátu		& 1--1--1	& 20--20--20	& \checkmark	& \checkmark	&K	\\		\hline
		\mytablerow 				& Zkosený blok základny		& 1--1--4	& 20--20--4		& 				& 				&Z	\\		\hline
		\mytablerow 				& Zkosený blok stavby		& 1--1--1	& 20--20--20	& \checkmark	& \checkmark	&Z	\\		\hline
		\mytablerow 				& Roh bloku stavby			& 1--1--1	& 20--20--20	& \checkmark	& \checkmark	&R	\\		\hline
	\currentCategory{\textbf{Speciální bloky}} 									 												\\		\hline
		\mytablerow 				& Terminál			 		& 1--8--5 	& 1--8--5		& 				& 				&V	\\		\hline
		\mytablerow 				& Napájené okno				& 2--1--2	& 20--1--20		& \checkmark	& \checkmark	&K	\\		\hline
		\mytablerow 				& Dveře 					& 7--7--11	& 7--7--11		& 				& 				&V	\\		\hline
		\mytablerow 				& Světlo					& 1--1--1	& 1--1--1		& \checkmark	& \checkmark	&V	\\		\hline
		\mytablerow 				& Přepínač 					& 1--1--1	& 1--1--1		& \checkmark	& \checkmark	&V	\\		\hline
		\mytablerow 				& Generátor energie			& 3--3--2	& 20--20--2		& 				& 				&K	\\		\hline
		\mytablerow 				& Generátor objektů 		& 3--3--2	& 20--20--2		& 				& 				&K	\\		\hline
		\mytablerow 				& Akumulátor				& 3--3--3	& 3--3--3		& 				& 				&V	\\		\hline
		\mytablerow 				& Plnička kyslíkových bomb 	& 4--3--4	& 4--3--4		& 				& 				&V	\\		\hline
		\mytablerow 				& Kyslíková bomba			& 2--2--2	& 2--2--2		& 				& 				&V	\\		\hline
		
\end{tabular}


\subsection{Podrobný popis bloků}

Popis některých vlastností - má energetickou komponentu - > implikuje definici bindovacích bodů
má kyslíkovou komponentu - implikuje TotalObjectOxygen

Producer nebo Consumer implikuje Total object energy

Controllable implikuje IsController nebo IsControllable



\subsubsection{A1 - Blok základny}
- velikost v~ose Z omezena na 4 základní bloky

- má elektriku

Pokud bychom měli nerovný terén, tento blok by mohl zahrnovat podstavce pro vyrovnání terénu.

\subsubsection{A2 - Blok stavby}
- všechny velikosti

- má elektriku

Tento blok je základním stavebním blokem ve hře.

\subsubsection{A3 - Blok polykarbonátu}
- všechny velikosti
Tento blok je nejlevnější, není připojen do elektrické sítě. Ideou bloku je podpora průhledných stěn a~také možné pomocné stavební konstrukce pro výstavbu do výšky. Inspiraci můžeme vidět v~používání třeba bloku hlíny ve hře \MC{}, kdy hráč vyskočí a~pod sebe umístí nový blok a~tím se ve světě posune o~1 metr výš.

\subsubsection{A4 - Zkosený blok základny}
- velikost v~ose Z omezena na 4 základní bloky

- má elektriku

Stejné jako blok \textit{A1}, jen je zkosený. Může sloužit jako přístupová rampa.

\subsubsection{A5 - Zkosený blok stavby}
- všechny velikosti

- má elektriku
\subsubsection{A6 - Roh bloku stavby}
-všechny velikosti

- má elektriku

\subsubsection{B1 - Terminál}
- speciální, pevná velikost 1 x 8 x 5 bloků

- má elektriku, konzument, rychlé doplnění energie, ovládání rozhraní, komplexní přehled připojené elektrické sítě.
\subsubsection{B2 - Napájené okno}
- minimální velikost 2 x 1 x 2, maximální velikost 20 x 1 x 20 základních bloků

- má elektriku, konzument
\subsubsection{B3 - Dveře}
- speciální, pevná velikost 7 x 7 x 11 bloků

- má elektriku, otevírání
\subsubsection{B4 - Světlo}
-velikost omezena na 1 x 1 x 1 blok

- má elektriku, konzument, ovládání bez přepínače
\subsubsection{B5 - Přepínač}
-velikost omezena na 1 x 1 x 1 blok

- má elektriku, náhled stavu
\subsubsection{B6 - Generátor energie}
- omezená velikost v~ose Z na 2 bloky, jinak 3 x 3 až 20 x 20 v~ostatních osách

- má elektriku, producent
\subsubsection{B7 - Generátor objektů}
- omezená velikost v~ose Z na 2 bloky, jinak 3 x 3 až 20 x 20 v~ostatních osách

- má elektriku, konzument
\subsubsection{B8 - Akumulátor}
- speciální, pevná velikost 3 x 3 x 3 bloků

- má elektriku, producent, konzument, rychlý náhled naplnění
\subsubsection{B9 - Plnička kyslíkových bomb}
- speciální, pevná velikost 4 x 3 x 4 bloků

- má elektriku, kyslíkovou komponentu, konzument, UI, rychlé doplnění kyslíku

- využijeme ideu náhledu inventáře a~plnička bude zobrazovat blok B10, pokud bude nějaký takový blok plnit.

\subsubsection{B10 - Kyslíková bomba}
- speciální, pevná velikost 2 x 2 x 2 bloků

- má kyslíkovou komponentu, možnost sebrat, rychlý náhled naplnění, rychlé doplnění kyslíku







\subsection{Herní svět}

jaký chceme herní svět

\subsubsection{Reprezentace}

bude nám stačit nějaký tree, definovat rozměry, na chuncky kašlem

\subsubsection{Bloky v~herním světě}

do gridu

\subsubsection{Denní / noční cyklus}
dáme ho

\subsubsection{Herní překážky}

počasí, , atributy avataru

\subsubsection{(Ne)fyzikální chování}

nebudeme hrotit

\subsection{Inventář}

chceme volné sloty, rozšiřitelnost

\subsection{Avatar hráče}
avatar má nějaké vlastnosti, \HUD{}, 1St / 3rd person view, zdraví, O2, energie







\section{Herní nepřítel}
Protože samotné stavění bez nějakého cíle či překážky není úplně zábavné, musíme hráči připravit nějakou překážku, komplikaci, kterou musí překonávat. Zde neexistuje jednoznačné řešení --- to je závislé na celkovém prostředí hry, zamýšlené cílově skupině a mnoha dalších faktorech. Cílem našeho hráče bude přežít kyselé deště. Ty budou přicházet v náhodných intervalech a budou sloužit jako překážka v rozvoji hry. Zároveň to ale bude pro hráče nástroj, jak získávat prostředky pro ochranu před dalšími dešti a rozvoj svých staveb. 

\section{Backlog}


???



%!TEX root = ../prace.tex

\chapter{Detailní analýza}


%!TEX root = ../prace.tex

\section{Herní engine}
Nyní už víme, čeho bychom chtěli dosáhnout a je na čase vyřešit, jak toho dosáhneme. V první řadě bychom se měli zamyslet nad tím, jaký herní engine použijeme. Díky tomu budeme moct počítat s možnostmi a omezeními danými touto volbou. V zásadě máme několik možností:

\begin{itemize}
	\item Implementace vlastního enginu
	\item Použití existující nízkoúrovňový framework
	\item Použití existující herní engine
\end{itemize}

Každá volba má své pro a proti, což podrobně rozebereme v následujících odstavcích.

\subsection{Vlastní engine}
Hlavní výhodou, ale zároveň nevýhodou této volby je to, že bychom si všechny potřebné součásti enginu (třeba renderování) museli napsat sami. Tím bychom měli naprostou kontrolu nad celým produktem, ale zabralo by nám to netriviální množství času.
Vzhledem k rozsahu plánované funkcionality by tato volba byla nepraktická a tedy touto cestou se nevydáme.

\subsection{Nízkoúrovňový framework}
Máme na výběr z více druhů frameworků postavených na různých platformách. Mezi známějšími bychom mohli uvést například XNA (\texttt{C\#}) nebo Ogre (\texttt{C++}). Oba frameworky jsou k dispozici zdarma, nicméně jejich podpora stagnuje. Implementace hry s použitím některého z těchto frameworků by byla rychlejší než v předchozím případě, ale stále bychom museli spoustu funkcionality implementovat sami. 

\subsection{Existující herní engine}
V této kategorii máme nejvíce možností jak rychle implementovat celou práci. Zástupců je opět mnoho, nicméně mezi nejoblíbenější se řadí Unity (\texttt{C\#}) a \UE{}(\texttt{C++}), které jsou oba zdarma. Díky práci komunity pro oba enginy existuje i kvalitní dokumentace. Navíc jsou enginy obvykle multiplatformní a tedy existuje zde snadný postup distribuce na různé typy herních zařízení. Při rešerši jsme zjistili následující klady a zápory:

\subsubsection{Unity}
Výhoda (ale i nevýhoda) Unity je celkově v jednoduchosti. Engine nabízí dostatečné možnosti i pro tvorbu AAA herních titulů, ale za cenu toho, že si toho autoři musejí dost napsat sami (oproti \UE{}). 
\begin{itemize}
	\item Klady
		\subitem -- Pokud bychom chtěli modifikovatelný či nějakým způsobem dynamický terén, Unity implementuje podporu modifikace terénu.
	\item Zápory
		\subitem -- Při rešerši jsme řešili podporu dynamického navigačního meshe, která v Unity nebyla příliš dobrá. Při změně prostředí docházelo k lagům během přepočtu navmeshe. To by byl při umisťování bloků zásadní problém.
		\subitem -- Další nevýhodou je podpora materiálů, kdy bychom snadným způsobem nedosáhli vizuálně přitažlivého prostředí.

\end{itemize}

\subsubsection{Unreal Engine}
Oproti Unity je \UE{} podstatně komplexnější a pochopení všech vztahů a závislostí může být pro začínajícího herního programátora obtížné. Dalším zdrojem problémů může být i programování v \texttt{C++}, které je navíc díky technologii \UBT{} trochu jiné než klasické \texttt{C++} a je potřeba dodržovat standardy definované \UE{}.
\begin{itemize}
	\item Klady
		\subitem -- Komplexnější engine (\UE{} je primárně určen pro vývoj AAA titulů)
		\subitem -- Oproti Unity lepší podpora materiálů --- i negrafik může snadno vytvořit vizuálně přitažlivé povrchy objektů a nemusí se přitom zabývat psaním vlastních shaderů 
		\subitem -- Rychlý dynamický navmesh
	\item Zápory
		\subitem -- Komplexnější engine

\end{itemize}

Nakonec jsme zvolili poslední možnost - Unreal Engine. Autorovy znalosti především z oblasti \texttt{C\#} sice hovořily pro použití Unity, nicméně výhody použití \UE{} převážily nad nevýhodami i všemi výhodami Unity.	// TODO tohle chce vyladit

%!TEX root = ../../prace.tex

\section{Bloky}

Zde by měl být popis možností jak definovat a~následně implementovat bloky. jaké jsou výhody a~nevýhody jednotlivých implementací

- externě editovatelné formáty (+ - modding, - těžší implementace, parsing, validace)
- binární formát

- xml


- interní formát
- specifické subclassy pro bloky včetně specifických vlastností přímo na 
- definiční struktura



%!TEX root = ../prace.tex

\section{Vlastnosti bloků}

- bloky mohou mít několik vlastností:

- mít možnost elektriky a zapojení do elektrické sítě

- mít možnost uchování kyslíku, v případě použitíí elektirky pak i generování

- bloky mohou být použitelné, tj. hráč s nimi může nějakým zůsobem interagovat

- bloky mohou být sebratelné, tedy hráč si je může dát do svého inventáře. vlastnosti jako třeba uchovaná hodnota kyslíku, pak zůstávají zachované

- bloy mohou být zákadem pro rozpoznávání tvarů (TODO )

\subsection{Elektrika}


\subsection{Kyslík}


\subsection{Označovatelnost}


\subsection{Možnost vzít do inventáře}

\subsection{Interakce}


%!TEX root = ../../prace.tex



\section{Komponenty bloků}

Abychom mohli snadno rozšiřovat vlastnosti a~chování bloků, použijeme systém \textit{komponent}, který nám \UE{} nabízí. Komponenta je programová část, která ovlivňuje chování vlastníka dané komponenty. Cílem je pak dosáhnout toho, že je možné za běhu hry jednu komponentu transparentně vyměnit za jinou (komponentu s~jinou implementací), a~vlastník komponenty se nemusí zajímat o~detaily implementace. V naší hře toto chování nejspíše nevyužijeme, ale použití komponent není na škodu a~v~případě dalšího vývoje budeme mít snazší práci. (TODO učesat)

Z předchozí analýzy vyplývá, že budeme potřebovat řešit následující problémy:

\begin{itemize}
	\item Práce s~kyslíkem
	\item Práce s~elektrickou sítí a~energií
	\item Interakce s~uživatelem
	\item Umístění bloku v~herním světě
\end{itemize}


První dva body jsou ideální kandidáti na použití komponent. Pokud bychom se někdy v~budoucnu rozhodli upravit chování této funkcionality či jej z~libovolného důvodu měnit, komponentový systém pro nás bude výhodou. Navíc ne všechny herní bloky umí (z hlediska herního designu) s~danými prvky pracovat. Jak jsme již zmínili dříve, \UE{} nepovoluje vícenásobnou dědičnost a~bylo by velmi těžké vymyslet hiearchii dědičnosti bloků tak, abychom splnili požadavky pro všechny bloky a~zároveň si \uv{nesvázali ruce} pro nové bloky. S použitím komponent to bude snadné -- bloky, které danou funkcionalitu mají umět, budou mít danou komponentu a~budou s~ní moci pracovat.

Dalším problémem je interakce s~uživatelem. Abychom věděli, že hráč s~daným blokem chce interagovat, musíme vědět, že:
\begin{itemize}
	\item Je dostatečně blízko bloku
	\item Z pohledu hráče se dívá na daný blok 
	\item Vyjadřuje fakt, že chce interagovat (např. stiskem klávesy)
\end{itemize}


\subsection{Interakce a~označování}

Nejsnazší způsob, jak zjistit, na jaký herní objekt se hráč dívá, je použití RayTracingu (TODO link?, formát textu?). Díky němu můžeme \uv{z kamery} vyslat virtuální paprsek, který má stejný směr, jako je směr pohledu kamery. Pokud bude hráčův \HUD{} zobrazovat zaměřovací kříž či nějaký obdobný mechanismus a~náš paprsek bude z~pohledu kamery tímto zaměřovačem procházet, hráč může cíleně mířit na herní objekty a~my zároveň budeme mít správnou informaci o~objektu, na který hráč zaměřovačem míří. Tento způsob získávání informace o~objektech v~hráčově zaměřovači je ve hrách běžný a~použití RayTrace je (pokud je vhodně použito) i~dostatečně rychlé.

Nyní, když už víme, jak můžeme získávat informace o~tom, na který objekt hráč míří, tak tento mechanismus ještě rozšíříme o~další vlastnost. Je zapotřebí si uvědomit, že interakce s~blokem a~umisťování nového herního bloku (případně mazání) jsou prakticky jedna a~ta samá akce. Liší se pouze výsledkem -- reakcí na stisk nějaké klávesy či tlačítka myši. Ale ve všech případech musíme vědět, na jaký blok hráč míří zaměřovačem, u~umisťování navíc potřebujeme znát i~přesný polygon, na který hráč míří. Konkrétní polygon potřebujeme znát z~toho důvodu, že chceme, aby se přidávaný blok \uv{přilepil} k~bloku, na který míříme. Tedy chceme zachovat herní mechaniku, která je v~hrách z~kapitoly \ref{chap:uvod} běžná a~je natolik intuitivní a~rozšířená, že změna této mechaniky by nejspíše nedopadla dobře a~hráči by nebyla kladně přijata.

Všechny tyto požadavky lze splnit použitím metody \TT{LineTraceSingleByObjectType} (TODO ref?), které předáme správné parametry (především počátek a~konec paprsku a~typy objektů , které paprsek zaznamená) a~ta nám vrátí strukturu, popisující výsledek RT. Z něj se můžeme dozvědět, jestli byl nějaký blok v~cestě paprsku. A pokud ano, můžeme se ptát, zda měl komponentu interakce (potenciálně bychom mohli chtít bloky bez možnosti zaměření a~interakce, jakožto nesmazatelné objekty). Pokud bude i~tato podmínka splněna, můžeme se zajímat o~další vlastnosti kolize paprsku s~blokem a~na základě toho se nějak chovat.


\subsection{Umístění ve světě}

Smyslem této komponenty je oddělení implementace bloku jako takového a~implementace herního světa.


Ve výsledku tedy budeme chtít komponentu, která 

popis jednotlivých komponent dle předchozího, co všechno umí (např. přidání / odebrání hodnoty energie za použité zámku (není transakce))





%!TEX root = ../prace.tex

\section{Vlastnosti bloků}

- bloky mohou mít několik vlastností:

- mít možnost elektriky a zapojení do elektrické sítě

- mít možnost uchování kyslíku, v případě použitíí elektirky pak i generování

- bloky mohou být použitelné, tj. hráč s nimi může nějakým zůsobem interagovat

- bloky mohou být sebratelné, tedy hráč si je může dát do svého inventáře. vlastnosti jako třeba uchovaná hodnota kyslíku, pak zůstávají zachované

- bloy mohou být zákadem pro rozpoznávání tvarů (TODO )

\subsection{Elektrika}


\subsection{Kyslík}


\subsection{Označovatelnost}


\subsection{Možnost vzít do inventáře}

\subsection{Interakce}


%!TEX root = ../../prace.tex

\section{Bloky v herním světě}


- je více mmožností. Uchování pole 50000 x 50000 x 25000 // todo ověřit
je nesmysl. 

- nepotřebujeme otevřený svět bez mřížky (pozdější aktualizace ME, jinak SE), takže budeme hledat nějakou variantu stromové struktury

- nabízí se možnost clustorování budov a shlukování do skupin, s následnou optimalizací počtu hladin

- my jsme zvolili K-D strom kombinovatný s AABB. (proč? )

- náš strom má optimalizaci jedinného potomka, v případě potřeby se dogeneruje do úrovně níže, případně rozpadne na podčásti a rekurzivně se přidá.

- díky této variantě se můžeme snadno dotazovat na sousedy, což je hlavní cíl (proto)



%!TEX root = ../../prace.tex

\section{Počasí}

Abychom mohli popsat koncept počasí ve hře, musíme mít definovaný \textit{typ} počasí. Typ popisuje vlastnosti počasí (například \uv{je částečně zataženo}, \uv{hustě prší}) na základě nějakých parametrů. Základní koncept počasí ve hře pak můžeme popsat dvěma stavy -- typ počasí zůstává \textit{stejný}, nebo se \textit{mění}. Ve vlastnostech typu pak můžeme mít definováno, jak dlouho zůstává počasí v~daném typu a~jak dlouho probíhá změna na jiný typ (tyto konstanty mohou být dány intervalem, ze kterého se zvolí náhodné číslo a~tím dostaneme jistý stupeň \textit{proměnlivosti} počasí.

Přístupů, jak implementovat systém počasí, je opět více. Jednou možností je vytvořit herní objekt, který bude dědit z~třídy \TT{UActor}. V~aktualizační smyčce tohoto herního objektu pak můžeme mít implementovanou logiku, která bude aktualizovat počasí ve hře. Další možností je použít jednoduchý \textit{Behavior tree}, tedy strom chování. Použitím \textit{Behavior tree} získáváme možnost, jak měnit chování počasí v~Editoru bez nutnosti zásahů do zdrojových kódů hry. 

My jsme se rozhodli, že systém počasí implementujeme jako herní entitu s~umělou inteligencí, kterou bude tvořit právě \textit{Behavior tree}. Chceme, abychom mohli snadno spravovat chování počasí a~navíc tato funkcionalita není natolik výpočetně náročná, abychom ji nutně museli implementovat v~\CPP{}.

\textit{Typy} počasí pak budeme definovat v~Editoru, přičemž stejně jako u~definic bloků vytvoříme definiční třídu s~odpovídající strukturou v~\CPP{}. Tento přístup vyžadujeme z~toho důvodu, že budeme chtít ukládat stav počasí do souboru uložené hry a~pak tento stav při nahrávání opět obnovit. 

Dále požadujeme následující vlastnosti typu počasí: zda je daný typ \textit{bouřkový}, \textit{délku trvání}, \textit{čas změny} na jiný typ, koeficient \textit{rychlosti mraků} a~\textit{zataženosti oblohy}. Nechceme se zabývat grafickými detaily, takže jediným způsobem, jak můžeme hráči počasí vizualizovat, je zobrazením mraků. Podle koeficientu zataženosti pak budeme chtít měnit i~intenzitu osvětlení od Slunce. Protože je naše počasí velice jednoduché, budeme chtít, všechny vlastnosti byly zadány intervalem (vyjma informaci o~tom, zda jde typ bouřkový).

Bouřkový typ počasí budeme chápat speciální způsobem. V~průběhu tohoto typu počasí budeme poškozovat bloky a~tím tak budeme simulovat bouři kyselých dešťů. 

%!TEX root = ../../prace.tex

\section{Hráčova postava}
Hráčovu postavu nemusíme nijak dlouze rozebírat. Budeme nám stačit se držet základního konceptu ovládatelných postav v \UEu{}. Této ovládatelné postavě přidáme komponenty kyslíku, energie a inventáře. V kapitole (TODO ref) jsme zjistili, že budeme chtít umět označovat bloky, proto herní postavě navíc přidáme komponentu, která se bude starat o výpočty sledovacího paprsku. Tato komponenta bude mít k dispozici referenci na aktuální kameru, přiřazenou k ovládané postavě, což jí umožní tyto výpočty provádět.


%!TEX root = ../../prace.tex

\section{Inventář}

Inventář bude vhodné implementovat jako komponentu, kterou posléze můžeme přiřadit herní postavě. Opět se zde opíráme o~fakt, že bychom v~budoucnu mohli mít bloky či \NPC{}, kteří také mají svůj inventář. Z~kapitoly \ref{subsec:inventory} vyplývá, že inventář má několik základních vlastností:

\begin{itemize}
	\item Seznam postavitelných bloků
	\item Seznam umístitelných bloků
	\item Seznam inventárních skupin pro filtrování
\end{itemize}

V momentě, kdy budeme chtít hráči nabídnout bloky k~postavení či umístění, oba seznamy dofiltrujeme dle aktuálně zvolené inventární skupiny. Musíme se tedy zaměřit na to, jak budeme řešit inventární skupiny.

\subsection{Inventární skupiny}

Protože má každý blok své definované \textit{tagy}, musíme vymyslet takový systém, abychom snadno filtrovali podle těchto tagů. Nejsnazší způsob je takový, kdy každá inventární skupina definuje, jaké tagy musí blok mít, aby byl přiřazen. Takže pro blok, který má být zobrazen v~rámci inventární skupiny, platí, že blok definuje všechny tagy v~rámci dané inventární skupiny. Nicméně se může stát, že hráč bude chtít mít v~dané skupině takové bloky, které nemusí splňovat všechny tagy, ale splňují alespoň jeden z~definované skupiny. Z~této myšlenky se dostáváme k~jednoznačnému zápisu algoritmu filtrování tagů -- \CNF{}. \CNF{} je označení pro \textit{konjunktivně normální formu}, užívané ve výrokové logice.

Formálně bychom tuto myšlenku zapsali jako:
\begin{equation}\label{eq:cnf}
	( A_1 \lor A_2 \lor ... \lor A_n ) \land ( B_1 \lor B_2 \lor ... \lor B_n ) \land ... \land ( X_1 \lor X_2 \lor ... \lor X_n )
\end{equation}


Při pohledu na formuli \ref{eq:cnf} vidíme, že můžeme implementovat takový systém, kdy \textit{inventární skupina} definuje \textit{inventární skupiny tagů} (konjunktivní vyhodnocení), přičemž \textit{inventární skupiny tagů} definují \textit{skupiny tagů} (disjunktivní vyhodnocení). Při vyhodnocování, zda blok patří do dané \textit{inventární skupiny}, tedy budeme sledovat to, zda daný blok splňuje všechny tyto \textit{inventární skupiny tagů} a~pro každou z~nich budeme zjišťovat, zda blok definuje alespoň jeden tag ze \textit{skupiny tagů}. 

Abychom hráči ještě více zpříjemnili práci s~inventářem, nebudeme požadovat přesnou shodu tagů definovaných blokem a~\textit{skupinou tagů}. Bude nám totiž stačit pouze podmínka podřetězce, tedy podmínka, aby tag bloku obsahoval jako podřetězec tag ze \textit{skupiny tagů}. Tím dosáhneme toho, že hráč nebude muset v~inventáři vypisovat celé řetězce z~tagů, které si nadefinoval u~svých bloků. 





%!TEX root = ../../prace.tex

\section{Ukládání hry}

Možností, jak implementovat systém ukládání hry, je opět více. Stejně jako u~definic bloků v~kapitole \ref{sec:db} můžeme volit mezi textovým a~binárním formátem. Očekáváme, že soubor uložené hry bude obsahovat velké množství dat, a~proto bychom chtěli minimalizovat výslednou velikost uloženého souboru. Z~tohoto požadavku vyplývá, že nebudeme chtít ukládat hry v~textovém formátu.

Při rešerši možného řešení jsme objevili zajímavý tutoriál (wiki stránka tutoriálu~\citep{ue_save_system}) na ukládání hry v~binárním formátu. Pojďme se podívat, co nám dané řešení nabízí.

Výše zmíněný systém ukládání je postaven na faktu, že v~C++ je možné přetěžovat operátory, mimo jiné i~streamové operátory $<<$. Této vlastnosti je využito tak šikovně, že v~závislosti na volání funkce buď zapisuje do archivu (souboru uložené hry), nebo z~něj čte. Pořád se však jedná o~jediný zápis jedné funkce. To je výhodné, protože to předchází chybám, které by mohly vzniknout při použití 2 metod (jedné čtecí, jedné zapisovací). Také tím předcházíme chybám typu \textit{přehození} pořadí datových \textit{typů} (což by v~případě typů různých velikostí znamenalo následné špatné pochopení binárních dat, nebo rovnou pád aplikace), nebo kupříkladu prohození dvou vlastností stejného typu, což by vytvářelo těžko odhalitelné situace změn hodnot ve hře. Dále nám tento systém nabízí možnost \textit{komprese} dat. Pokud bychom měli ukládané soubory příliš velké, tuto vlastnost bychom mohli využít a~hráči tak šetřit místo na disku.

Shledali jsme, že je pro nás přínosné využít tento způsob ukládání dat.


%!TEX root = ../prace.tex

\section{Doplňující vlastnosti}



\subsection{Lokalizace}

- použití lokalizace 

\subsection{Hudba}

- atmosférický hudební doprovod




\chapter{Backlog}

\begin{itemize}
	\item Svět + jak vypadá + jak je reprezentován (K-D tree zde, nebo v programátorské?)
	\item Popsat bloky, velikosti
	\item Popsat komponenty bloků (že je něco jako komponenta elektriky, komponenta vzduchu
	\item Popsat hratelnou postavičku (že má taky možnost elektriky a kyslíku
	\item Popsat počasí - že má taky svoji blokovou reprezentaci a že je na pozadí Behavior Tree, který to celé řídí (ovlivňování konfigurace v programátorské části)
	\item Popsat implementaci elektriky // TODO dodělat ve hře
	\item Popsat implementaci rozpoznávání tvarů // TODO dodělat ve hře
	\item Popsat způsob ukládání a načítání hry (to je možná až do Progr. sekce?)
	\item Popsat, že bychom chtěli nějaké UI + nabídky menu
	\item Popsat, že bychom chtěli základní hudbu
	\item Popsat, že máme něco jako inventář s možností nějaké správy bloků
	\item Stejně tak pro builder + herní terminály  //TODO doimplementovat
\end{itemize}

%!TEX root = ../prace.tex

\chapter{Programátorská dokumentace}

%!TEX root = ../../prace.tex

\section{Počáteční inicializace projektu}

Pokud je cílem spustit projekt hry ze zdrojových kódů, je potřeba si stáhnout \UE{} ve verzi \TT{4.15.3}. Na hlavní stránce \UE{} je potřeba stáhnout \textit{Epic Games Launcher} (ke stažení zde~\citep{ue_download}). Po vytvoření uživatelského účtu a~následnému přihlášení do aplikace bude vidět obrazovka podobná obrázku \ref{fig:ueLauncher}:


\begin{figure}[!ht]\centering
\includegraphics[ width=140mm]{../img/program/ueLauncher}

\caption{Epic Games Launcher}
\label{fig:ueLauncher}

\end{figure}

\FloatBarrier

Pokud není vidět \UE{} dané verze v~seznamu verzí, je potřeba jej přidat kliknutím na tlačítko \textit{Add Versions} a~případném následném výběru správné verze. Přidaná verze se zobrazí stejným způsobem, jako je na obrázku \ref{fig:ueLauncher} zobrazen \UE{} verze \TT{4.16.2}. Posledním krokem je samotná instalace stisknutím tlačítka \textit{Install}. Použití novější verze \UEu{} je možné, ale běžnému uživateli to nedoporučujeme. Mezi verzemi se mohou projevit nekompatibility v~kódu, které je mnohdy nutné řešit zásahy přímo do zdrojových kódů hry.

Dále je potřeba mít k~dispozici zdrojové kódy ať už z~přiloženého DVD, nebo z~tohoto release na GitHubu (odkaz~\citep{gh_finalRelease}). Na DVD se soubory nachází ve složce \textit{Source}. Ve složce zdrojových kódů hry se nachází soubor \TT{TauCetiF2.uproject}, což je soubor projektu v~\UEu{}. Další postup závisí na tom, zda je cílem otevřít hru v~Editoru \UEu{}, nebo vytvořit projekt pro Visual Studio, z~něhož je pak možné Editor spustit. Oba přístupy jsou záměnné, protože vytvoření projektu pro Visual Studio po spuštění Editoru je také validní postup. Pro úspěšnou kompilaci a~spuštění hry je zapotřebí mít Visual Studio 2015 alespoň ve verzi \textit{Community} a~mít u~něj zapnutou podporu jazyka \CPP{} (to se řeší během instalace Visual Studia).
 
\subsubsection{Přímé vytvoření projektu pro Visual Studio}
Projekt pro Visual Studio je možné vytvořit kliknutím pravým tlačítkem myši na soubor \TT{TauCetiF2.uproject} a~následnou volbou \textit{Generate Visual Studio project files} (viz obrázek \ref{fig:generateProjectFiles}). 

\begin{figure}[!ht]\centering
\includegraphics[ width=80mm]{../img/program/generateProjectFiles}

\caption{Vytvoření projektu hry}
\label{fig:generateProjectFiles}

\end{figure}
\FloatBarrier

Tento postup je v~praxi běžně používaný. Celý projekt je totiž vytvořen na základě informací ze složky \textit{Source}. Mimo jiné má tento přístup tu výhodu, že při použití systému správy verzí (jako například GITu) se tyto vygenerované soubory nemusí verzovat -- každý vývojář si je vytvoří na svém počítači.

Pokud v~kontextové nabídce chybí možnosti pro \UE{}, je potřeba provést asociaci s~\TT{uproject} soubory. Tato asociace by měla být provedena automaticky, nejpozději po vytvoření libovolného \CPP{} projektu z~kterékoliv šablony. Ze zkušenosti autora se to mnohdy nemusí podařit. Pokud se nepodaří vygenerovat projekt hry, může pomoci druhý přístup otevření projektu, který je popsán v~následující části.

Předposledním krokem je v~případě úspěšného vytvoření projektu hry jeho otevření ve \textit{Visual Studiu}, nastavení \textit{TauCetiF2} jako hlavní spustitelnou část projektu (viz obrázek \ref{fig:vs}) a~následné spuštění editoru hry.


\begin{figure}[!ht]\centering
\includegraphics[ width=110mm]{../img/program/vs}

\caption{Vygenerování projektu hry}
\label{fig:vs}

\end{figure}

\FloatBarrier



\subsubsection{Přímé spuštění Editoru}
Druhý postup zahrnuje prosté spuštění Editoru dvojklikem na soubor projektu. Pokud soubor projektu nemá modrou ikonu jako na obrázku \ref{fig:generateProjectFiles}, není tento typ souborů asociovaný s~Unreal Editorem. Pak je nutné otevřít Editor přes \textit{Epic Games Launcher}, najít a~otevřít soubor projektu. Protože hra využívá systému modulů, při prvním spuštění se uživateli zobrazí hláška o~tom, že některé moduly je potřeba zkompilovat. Tuto akci necháme provést. Pokud by kompilace modulů selhala, nejspíše máme problém s~Visual Studiem. V~takovém případě doporučujeme postup níže v~rámci \textit{kroků poslední záchrany}.

Pokud je úspěšně otevřen a~načten Editor, zbývá poslední krok -- výběr obnovy projektu, jak je naznačeno na obrázku \ref{fig:ue_refresh}.

\begin{figure}[!ht]\centering
\includegraphics[ width=80mm]{../img/program/ue_refresh}

\caption{Vygenerovaný projekt hry}
\label{fig:ue_refresh}

\end{figure}

\FloatBarrier


\subsubsection{Kroky poslední záchrany}
Pokud všechny námi uvedené postupy selžou, je potřeba si ověřit, že je možné založit libovolný projekt (založený na \CPP{}) z~dodávaných šablon, zkompilovat jej a~vygenerovat projekt pro Visual Studio. Pokud se to povede s~projektem z~nějaké připravené šablony, mělo by to fungovat i~s~naší hrou. V~případě, že ani tento postup nefunguje, je potřeba zjistit příčinu z~kompilačních Logů. Řešení této nefunkčnosti jde mimo téma této práce, takže se jím nebudeme dále zabývat.







%!TEX root = ../prace.tex

\section{Struktura kódu}

Program se dělí do několika modulů. Jejich struktura je zachycena na obrázku \ref{fig:obrStruktura_DependencyDiag}.

\begin{figure}[h!]\centering
\includegraphics[ height=70mm]{../img/dependencyDiag.png}

\caption{Diagram závislostí modulů projektu.}
\label{fig:obrStruktura_DependencyDiag}

\end{figure}

Když si otevřeme zdrojovou sln projektu, tak uvidíme, že každý modul má několik podřízených věcí: 

- složku Private

- složku Public

- soubory .Build.cs, .h, .cpp

Každý modul pak má hlavičkové soubory ve složce Public, implementaci tříd pak ve složce Private. Poslední tři soubory jsou kvůli UBT a // TODO použitá zkratka 
použitím herních modulů v rámci UE



\subsection{Modul Commons}
%!TEX root = ../../prace.tex

\section{Modul Commons (C++)}

Tento modul je základním modulem, který na jednom místě definuje všechny potřebné informace, které využívají ostatní moduly. Jedná se zejména o~definici herních konstant, či definice všech sdílených enumerátorů. Najdeme zde také prapředka použité herní instance. Tuto vlastní implementaci herní instanci využijeme pro ukládání nalezených bloků.

\subsection{Herní definice a~konstanty}



V souboru \TT{GameDefinitions.h}jsou definovány všechny herní konstanty. Například je zde definována velikost jednotkové krychle, velikost použitého světa, vztah mezi délkou dne herního světa a~počtem uplynulých reálných sekund, převody mezi energií, kyslíkem, zdraví a~jednotkou zásahu kyselého deště. Taktéž jsou zde definovány konstanty, které využívá technika obrysů objektu (todo link na outline). Dále jsou zde definovány konstanty ID implementovaných bloků, abychom s~nimi mohli pracovat i~v~kódu.

\subsection{Herní instance}

Herní instance \TT{TCF2GameInstance.h} se chová jako návrhový vzor \textit{Singleton} a~jako jediná zůstává vždy stejná po celou dobu běhu hry (jak uchovávat globální data \citep{ue_gameInstance}). Proto se využívá například pro uchovávání dat při přechodu mezi jednotlivými Levely a~my toho také využijeme. Zároveň se tato třída dá využít pro implementaci delegátů, kterými je možné vyvolat nějakou událost a~libovolný prvek z~herního světa může tuto událost obsloužit. My toho využijeme u~reakce na denní cyklus u~bloku \textit{Přepínače}.

Dalším důležitým bodem pro nás bude, že tato třída si bude držet referenci na všechny nalezené bloky. Z předchozího textu již víme, že bloky je potenciálně možné rozšířit o~DLC (TODO budem to tu řešit?), takže je nutné, abychom si nalezené bloky a~jejich definice udrželi v~paměti i~při přechodu mezi levely. K tomu slouží proměnná \TT{BlockHolder} , která sice  drží referenci na objekt definovaný v~modulu \textit{Blocks}, ale kvůli zpětným referencím mezi moduly (které nejsou povolené) musíme zde použít dostupného předka.

Kód tedy bude vypadat následovně:
\begin{code}
	UPROPERTY(Transient)
		UObject* BlockHolder;

	UFUNCTION(BlueprintCallable, Category = "TCF2 | GameInstance")
		void SetHolderInstance(UObject* holder);
\end{code}



 Parametr \TT{Transient} u~makra \TT{UPROPERTY} znamená, že daná proměnná bude vždy nastavena na svoji výchozí hodnotu. V tomto případě je to použito spíše z~důvodu zachování konzistence napříč projektu, ale zjednodušeně bychom důsledky mohli popsat následovně -- pokud bude nějaký Blueprint dědit z~nějaké \CPP{} třídy, tak vývojář může nastavit výchozí hodnoty properties. Tyto hodnoty jsou pak serializovány do \CDO{}, cože je \textit{Class Default Object} (otázka na Answers Unreal Engine \citep{ue_cdo}). Během procesu vytváření nové instance objektu, který vychází z~daného Blueprintu pak budou tyto hodnoty naplněny během fáze inicializace properties (popis životního cyklu actorů \citep{ue_actor_life}). V konečném důsledku by pak byla tato hodnota nějakým způsobem naplněna. Pokud chceme vynutit, aby tato property nebyla serializována do CDO, tak ji označíme jako \TT{Transient}.

V průběhu hry pak jednou naplníme tuto property pomocí metody\\ \TT{SetHolderInstance}, do které předáme referenci na korektně inicializovanou instanci třídy \TT{BlockHolder} z~modulu \textit{Blocks}. Pak si můžeme odkudkoliv získat aktuální herní instanci, přetypovat na \TT{TCF2GameInstance} a~získat si (přetypovanou) referenci na \TT{BlockHolder}. Z něho pak již můžeme získávat informace o~všech dostupných blocích.

\subsection{Enumerátory}

\TT{Enums.h}
Tento soubor slouží jako jednotné umístění pro všechny výčtové typy (enumerátory), které se používají napříč celým projektem. Neznamená to, že nutně obsahuje všechny -- některé třídy mohou využívat své specifické enumerátory, které ale nemusí být umístěny v~tomto globálně dostupném modulu.


\subsection{Helpery}

\TT{CommonHelpers.h} Tato třída poskytuje metody pro práci s~konfigurací. Statické metody umožňují načítat a~ukládat konfigurační položky typu \TT{float}, \TT{bool} a~\TT{string}.
Aby byla práce co nejjednodušší, metody přijímají enumerátor\linebreak[4]\TT{EGameUserSettingsVariable}. Třída pak sama na základě hodnoty tohoto enumerátoru použije správný klíč (který je textový) a~tak může ukládat či vracet hodnotu daného typu z~konfiguračního souboru.


\subsection{Modul Game Save}
%!TEX root = ../prace.tex

Modul GameSave slouží  k ukládání a načítání informací o probíhající hře do binárního formátu. K tomu používáme streamové operátory $<<$, které jsou v tomto případě implementovány tak, že je možné je použít jak pro ukládání, tak pro načítání. // TODO link na tutorial

Díky tomuto přístupu tak můžeme definovat celou strukturu výsledného binárního souboru na jednom místě a tedy rozšiřování uložené hry je triviální. Co si ovšem musíme pohlídat je to, abychom si drželi informaci o verzi uloženého souboru. V našem případě, pokud se bude lišit verze načteného souboru a uložená konstanta v programu, save prostě odmítneme (a dokonce smažeme). V produkčním prostředí bychom si mazání nemohli dovolit, ale museli bychom save ignorovat a uživateli zobrazit nějakou hlášku o tom, že verze souboru není podporovaná. My jsme se však v tomto případě rozhodli save mazat, protože jsme očekávali, že během vývoje hry se bude binární struktura savu často rozšiřovat. Po každé iteraci jsme si savy prostě vytvořili nové.

Co by se stalo, kdybychom se snažili načíst save jiné verze? Celá hra by nejspíše byla ukončena s chybou, protože by se pokoušela číst neplatná data a/nebo by očekávala nějaká data tam, kde žádná nejsou. Tím bychom četli z neplatné lokace.


- popsat save game carrier ( + výsledný formát)

- zdůraznit, že se jedná o holá data, UObjekty si pak vytváří každý modul sám

- popsat NewSaveGameHolder, rozšiřování pevných savů 

- popsat *Archive helpers 


\subsection{Modul Blocks}
%!TEX root = ../../prace.tex

\section{Modul Blocks (C++)}



Modul bloků obsahuje podstatné informace o~tom, jak hra pracuje s~bloky, jak se tyto bloky skládají do herního světa, jaké jsou jejich komponenty apod. Také je v~tomto modulu možné nalézt specifické implementace jednotlivých bloků.

V dalším textu se budeme odkazovat na složky. Odkazujeme se tím do složek \TT{/Source/Blocks/Public} a~jejich \TT{Private} implementací. Strukturu bychom mohli shrnout následovně:

\begin{enumerate}
	\item Definice bloků (složka \TT{Definitions})
	\item Třídy s~popisem bloků (složka \TT{Info})
	\item Systém ukládání a~načítání bloků (složka \TT{Helpers})
	\item Rozhraní, které mohou bloky implementovat (složka \TT{Interfaces})
	\item Komponenty, kterými bloky rozšiřují svoji základní funkcionalitu (složka \TT{Components})
	\item Implementace jednotlivých bloků (složky \TT{BaseShapes}, \TT{Special})
	\item Stromové struktury herního světa (složka \TT{Tree})
\end{enumerate}
 

\subsection{Definice bloků}
V této složce se nachází všechny definiční soubory bloků. Definiční soubor obsahuje pouze popis datové struktury a~nějakou minimální funkcionalitu (kupříkladu získání korektního vektoru velikosti v~závislosti na tom, zda má definice daného bloku nastavenou vlastní velikost). Jednotlivé konkrétní instance s~daty jsou pak definovány na straně editoru (tuto funkcionalitu jsme již zmínili v~kapitole \ref{subsec:hb}). Konstanty (například minimální a~maximální škálování) je pak možné měnit v~editoru a~není vyžadována rekompilace projektu hry. 


\subsection{Třídy s~popisem bloků}
Tyto třídy popisují už konkrétní instance bloků v~rámci hry. Jejich hodnoty jsou pak v~mezích definovaných v~definičních třídách. Tyto třídy jsou pak předmětem ukládání a~načítání. Dalším důležitým prvkem je \textit{BlockHolder}, který slouží pro nalezení bloků. 


\subsection{Komponenty bloků}
Komponenty bloků vycházejí z~poznatků v~části \ref{sec:komponents}. Rádi bychom zde zmínili zajímavou část převodů kyslíku a~energie.


\subsubsection{Převádění kyslíku a~energie}
Protože \UE{} umožňuje hrám pracovat s~více výpočetními vlákny, musíme zajistit konzistenci dat při převodech kyslíku či energie. Můžeme pro to využít primitiva pro zamykání \TT{FCriticalSection}. Kritickou sekci pak budeme korektně zamykat a~odemykat (stejně jako u~klasického vícevláknového programování). Algoritmy pro vkládání a~získání kyslíku budou mít následující signaturu (pro energii bude signatura stejná):

\begin{code}
    // dodej kyslík komponentě
    bool UOxygenComponent::PutAmount(float aviable,
                                     float& actuallyPutted)

    // získej kyslík z~komponenty                                     
    bool UOxygenComponent::ObtainAmount(float requested,
                                        float& actuallyObtained,
                                        bool requireExact)
\end{code}


Princip je prostý -- metody vrací \TT{bool} jakožto hodnotu, zda bylo možné operaci korektně provést. Parametry předávané \textit{referencí} pak v~případě úspěchu obsahují hodnotu skutečně vloženého či získaného kyslíku. Poslední parametr u~metody pro získání kyslíku značí, zda je vyžadované přesné množství. Pokud kyslíková komponenta obsahuje méně kyslíku, než je požadované množství a~je požadované přesně zadané množství, převod nebude úspěšný a~metoda vrátí \TT{false}. Pokud nebude požadované přesně dané množství, skutečně získané množství může být menší a~je na volajícím, aby se tomuto faktu přizpůsobil.


\subsection{Implementace bloků}
\label{subsec:blImp}

Všechny herní bloky dědí ze základní třídy \TT{ABlock}. Dále jsme zavedli dělení na \TT{BaseShapes} a~\TT{Special}, přečemž jak struktura zdrojových kódů, tak struktura v~rámci \UEu{} tuto strukturu dodržuje. Toto dělení jsme již definovali v~části \ref{subsubsec:vlastnosti}.




\subsection{Modul Inventory}
%!TEX root = ../prace.tex



\section{Modul Inventory (C++)}

Modul inventáře byl vyčleněn do samostatné části. Je to hlavně jako ukázka možného členění do modulů. Navíc časem by se mohl tento modul rozšířovat jak by rostla kompelxicita správy inventáře.

Nejdůležitější inventory component


\subsection{Tag group}
nejnižší úroveň, odpovídá 'nebo'

\subsection{Inventory tag group}

celá skupina, odpovídá 'A zároveň'

\subsection{Inventory tags}

sdružuje všechny banky

\subsection{Inventory component}

celá komponenta, která je pak navázaná na hráčův charakter

definuje delegáty notifikující o změnách v akivní skupině, po filtrování apod.

na této úrovni se řeší aktualizace cache buildable i inventorybuildable při změnách, zároveň poskytuje možnost clear cache pro volání shora (BP)




\subsection{Modul TauCetiF2}
%!TEX root = ../../prace.tex

\section{Modul TauCetiF2 (C++)}
Primární herní modul \textit{TauCetiF2} obsahuje mimo jiné implementaci herního objektu \textit{WorldController}, který jsme popisovali v~části \ref{subs:wc}. Dále obsahuje bázové třídy pro většinu UI oken hry. Zde bychom rádi zmínili \TT{SynchronizeWidget}, který dědí z~\TT{UUserWidget}, tedy základního uživatelského widgetu (UI prvku, který je možné zobrazit). Tato třída pak implementuje metodu \TT{SynchronizeProperties}, díky níž získávají potomci této třídy možnost okamžité aktualizace po změně libovolné vlastnosti v~Editoru widgetů. Tato vlastnost běžně dostupná není, ale tímto \uv{trikem} lze snadno získat možnost ihned vidět změny v~rámci daného editovaného UI prvku.
 





%!TEX root = ../../prace.tex

\section{Struktura projektu v~Unreal Enginu}
\label{sec:ueStructure}

- ukázat jak se to dělí v~UE editoru, obrázky jak proudí informace, data flow

Struktura složek odpovídá /Content 

TODO  popsatco je BP co je CPP a~bude později 

\begin{itemize}
	\item Složka XY
\end{itemize}


TODO hodně obrázky, vztahy

\subsection{Mapy}

Základní částí projektu v~\UE{u} je \textit{herní mapa}. Ta obsahuje všechny důležité herní objekty.

popsat jednotlivé mapy, jaký je jejich cíl

- Game Loader

zavední hry

- MainMenu

menu, má sublevel MainMap (zobrazení bloků ve hře jako náhledu světa)

- Loading Screen

nahrávání, obsahuje mainmap jako streamovanej level

- Main map

\subsubsection{terén}

 (2 části), terén má speciální tag (označovatelnost, resp. získání cíle pro RayTracing)

\subsubsection{World Controller}

bloky ve světě, napojení na hlavní BP levelu

\subsubsection{GameElectricityPawn}

elektrika

\subsubsection{Game Weather Pawn}

počasí

\subsubsection{SkyBP}

denní cyklus

\subsubsection{AmbientSound}

řeší hudbu ve hře

\subsubsection{Tutorial}

řeší tutorial



\subsection{Mainmap -- průběh nahrávání}

popsat jak je to udělané

\subsection{Charakter}

výchozí spawn z~gamemodu, v~mainBP levelu Posses

\subsection{World Controller}

bloky ve světě, napojení na hlavní BP levelu


\subsection{bloky}

Kde a~jak definované, obrázek, jak se to definuje konstanty


\subsection{GameElectricityPawn}

elektrika, AI controller

\subsection{Game Weather Pawn}

počasí, AI controller, BT počasí + komponenty

\subsection{SkyBP}

denní cyklus -- popis

\subsection{AmbientSound}

řeší hudbu ve hře

\subsection{Tutorial}

řeší tutorial, používá widgety


\subsection{Lokalizacer}

nastínit lokalizaci



\subsection{UI}
nastínit UI -- základní widgety atd.





\subsection{Backlog}

Zde popsat jak jsem to celé implementoval a proč

Popsat jednotlivé moduly a nakreslit diagram vztahů mezi nimi

Popsat strukturu save gamu + důvod proč jsem to tak udělal
+popsat načíttání savů + systémových savů


Popsat jednotlivé C++ třídy a jejich odvozené Blueprintové deriváty + přidat případné obrázky z BL kódu (např. BlueprintImplementable event, který se zavolá jak na C++ tak i na BP) 

Udělat rozbor BT počasí + mechaniku počasí + denního cyklu
popsat řízení osvětlení dle počasí

Udělat rozbor bloků, škálování, konfigurace, datovou strukturu, implementaci dynamických textur, zvýraznění

Popsat mechaniku Selector - SelectTarget + napojení na Builder

Popsat mechaniku používání objektů + zvýraznění

Popsat mechaniku Inventáře

// TODO vymyslet vhodné pořadí, abych neskákal mezi prvky, toto pořadí dodržet i v předchozích kapitolách

-> Mám svět, ten má v sobě bloky, ty jsou v nějaké stromové struktuře, bloky mají komponenty, které přes tuto strukturu mohou na sebe vázat
Svět má také počasí se svojí vlastní strukturou, vuyžívající podobnosti s bloky (2D KD strom s Heapem na listech)

-> hráč může to a tamto, díky inventáři se dostane na bloky, a díky selectoru je pak můževložit do světa skrz World controller (zmíněno v předchozím)
->zároveň jsou všechny entity savovatelné 


-> Popsat struktury Widgetů, zmínit použití Synchronize Widgetu, implementaci mechaniky stackovatelných widgetů

-> popsat implementaci hudby

-> TODO otestovat možnost nového bloku v rámci DLC
->Zmínit zároveň, že s tímto by šlo tweakovat nastavení hry



// TODO obrázky s konfiguračními ukázkami do příloh (např. jak se definuje Blok z UE



%!TEX root = ../prace.tex

\chapter{Uživatelská dokumentace}


%!TEX root = ../prace.tex

\section{Hlavní menu}

Po načtení hry hráč vidí hlavní menu. Denní doba i počasí se zvolí náhodně.

\begin{figure}[!h]\centering
\includegraphics[ width=140mm]{../img/user/mainMenu/mmDay}

\caption{Obrazovka hlavního menu - den}
\label{fig:user_mainMenu_mmDay}

\end{figure}

\begin{figure}[!h]\centering
\includegraphics[ width=140mm]{../img/user/mainMenu/mmNoon}

\caption{Obrazovka hlavního menu - poledne, zataženo}
\label{fig:user_mainMenu_mmNoon}

\end{figure}

\begin{figure}[!h]\centering
\includegraphics[ width=140mm]{../img/user/mainMenu/mmNight}

\caption{Obrazovka hlavního menu - noc}
\label{fig:user_mainMenu_mmNight}

\end{figure}



\FloatBarrier

První volba, kterou je možné zvolit, je výběr nové hry. K dispozici je několik variant s různými obtížnostmi.

\begin{figure}[!h]\centering
\includegraphics[ width=140mm]{../img/user/mainMenu/mmBegin}

\caption{Obrazovka hlavního menu - Nová hra}
\label{fig:user_mainMenu_mmBegin}

\end{figure}
\FloatBarrier

Pokud hráč má nějaké uložené hry, může si je nahrát kliknutím na tlačítko \textbf{Nahrát hru}. V tomto případě žádné uložené hry k načtení k dispozici nejsou.

\begin{figure}[!h]\centering
\includegraphics[ width=140mm]{../img/user/mainMenu/mmLoad}

\caption{Obrazovka hlavního menu - Nahrát hru}
\label{fig:user_mainMenu_mmLoad}

\end{figure}
\FloatBarrier

Pod položkou \textbf{Nastavení} může uživatel měnit herní, zvuková a grafická nastavení hry. Nastavení se aplikují a ukládají okamžitě, výjimkou je pouze položka \textit{Animace generátoru energie}, která se projeví až po změně levelu. V případě konfigurace z hlavní nabídky se nastavení projeví okamžitě, v případě konfigurace během rozehrané hry je potřeba vyvolat opětovné načtení uložené hry, nebo znovu spustit novou hru.

\begin{figure}[!h]\centering
\includegraphics[ width=140mm]{../img/user/mainMenu/mmSettings}

\caption{Obrazovka hlavního menu - Nastavení}
\label{fig:user_mainMenu_mmSettings}

\end{figure}

\FloatBarrier
%!TEX root = ../prace.tex

\section{Herní menu - ukládání, nahrávání}



\begin{figure}[h!]\centering
\includegraphics[ width=140mm]{../img/user/save/0newSave}

\caption{Ukládání - Nový save}
\label{fig:user_save_0newSave}

\end{figure}

\begin{figure}[h!]\centering
\includegraphics[ width=140mm]{../img/user/save/1afterSave}

\caption{Ukládání - po uložení}
\label{fig:user_save_1afterSave}

\end{figure}

\begin{figure}[h!]\centering
\includegraphics[ width=140mm]{../img/user/save/2load}

\caption{Ukládání - nahrát hru}
\label{fig:user_save_2load}

\end{figure}


\begin{figure}[h!]\centering
\includegraphics[ width=140mm]{../img/user/save/3loadClicked}

\caption{Ukládání - potvrzení nahrání hry}
\label{fig:user_save_3loadClicked}

\end{figure}

%!TEX root = ../prace.tex

\section{Inventář}



\begin{figure}[h!]\centering
\includegraphics[ width=140mm]{../img/user/inventory/0AddTag}

\caption{Inventář - den}
\label{fig:user_inventory_0AddTag}

\end{figure}

\begin{figure}[h!]\centering
\includegraphics[ width=140mm]{../img/user/inventory/1setFilter}

\caption{Inventář - poledne, zataženo}
\label{fig:user_inventory_1setFilter}

\end{figure}

\begin{figure}[h!]\centering
\includegraphics[ width=140mm]{../img/user/inventory/2filterResult}

\caption{Inventář - noc}
\label{fig:user_inventory_2filterResult}

\end{figure}


\begin{figure}[h!]\centering
\includegraphics[ width=140mm]{../img/user/inventory/3additionalFilter}

\caption{Inventář - Nová hra}
\label{fig:user_inventory_3additionalFilter}

\end{figure}

\begin{figure}[h!]\centering
\includegraphics[ width=140mm]{../img/user/inventory/4addFilterCollapsed}

\caption{Inventář - Nahrát hru}
\label{fig:user_inventory_4addFilterCollapsed}

\end{figure}


\begin{figure}[h!]\centering
\includegraphics[ width=140mm]{../img/user/inventory/5deleteItem}

\caption{Inventář - Nastavení}
\label{fig:user_inventory_5deleteItem}

\end{figure}


\begin{figure}[h!]\centering
\includegraphics[ width=140mm]{../img/user/inventory/6itemWithParams}

\caption{Inventář - Nastavení}
\label{fig:user_inventory_6itemWithParams}

\end{figure}
%!TEX root = ../prace.tex

\section{Terminál}



\begin{figure}[h!]\centering
\includegraphics[ width=140mm]{../img/user/terminal/0terminalOverall}

\caption{Terminál - den}
\label{fig:user_terminal_0terminalOverall}

\end{figure}

\begin{figure}[h!]\centering
\includegraphics[ width=140mm]{../img/user/terminal/1terminalInfo}

\caption{Terminál - poledne, zataženo}
\label{fig:user_terminal_1terminalInfo}

\end{figure}

\begin{figure}[h!]\centering
\includegraphics[ width=140mm]{../img/user/terminal/2terminalHelp}

\caption{Terminál - noc}
\label{fig:user_terminal_2terminalHelp}

\end{figure}


\begin{figure}[h!]\centering
\includegraphics[ width=140mm]{../img/user/terminal/3terminalCtorSelector}

\caption{Terminál - Nová hra}
\label{fig:user_terminal_3terminalCtorSelector}

\end{figure}

\begin{figure}[h!]\centering
\includegraphics[ width=140mm]{../img/user/terminal/4builderSmall}

\caption{Terminál - Nahrát hru}
\label{fig:user_terminal_4builderSmall}

\end{figure}


\begin{figure}[h!]\centering
\includegraphics[ width=140mm]{../img/user/terminal/5builderSize}

\caption{Terminál - Nastavení}
\label{fig:user_terminal_5builderSize}

\end{figure}


\begin{figure}[h!]\centering
\includegraphics[ width=140mm]{../img/user/terminal/6builderParams}

\caption{Terminál - Nastavení}
\label{fig:user_terminal_6builderParams}

\end{figure}
%!TEX root = ../../prace.tex

\section{Stavební akce}

\textbf{Destruktor} umožňuje mazat bloky. Po jeho výběru je vidět červený outline vybraného bloku.

\begin{figure}[!ht]\centering
\includegraphics[ width=140mm]{../img/user/buildActions/delete}

\caption{Stavění - smazat}
\label{fig:user_buildActions_delete}

\end{figure}

\FloatBarrier

Pokud vybereme umístění nového bloku, vidíme žlutě hranice sousedního bloku, ke kterému blok přistavujeme. Stavěný / umisťovaný blok musí mít dostatečné místo pro svoje umístění. Dále je potřeba mít s~sebou dostatečně velkou zásobu energie (dle energetické náročnosti bloku).

\begin{figure}[!ht]\centering
\includegraphics[ width=140mm]{../img/user/buildActions/place}

\caption{Stavění - umístit}
\label{fig:user_buildActions_place}

\end{figure}

\FloatBarrier

Blok je též možný rotovat (Klávesy v~sekci Insert .. Page Down, případně jejich ekvivalenty 7,8,9,4,5,6).

\begin{figure}[!ht]\centering
\includegraphics[ width=140mm]{../img/user/buildActions/place_Rotate}

\caption{Stavění - rotace}
\label{fig:user_buildActions_place_Rotate}

\end{figure}

\FloatBarrier

Pokud bylo umístění v~pořádku, blok již není nadále průhledný a~hráči ubyla energie. Pokud je zapnut kreativní mód, energie samozřejmě neubývá.

\begin{figure}[!ht]\centering
\includegraphics[ width=140mm]{../img/user/buildActions/placeAfter}

\caption{Stavění - po umístění}
\label{fig:user_buildActions_placeAfter}

\end{figure}


\FloatBarrier
%!TEX root = ../../prace.tex

\section{Umístitelné předměty}

Blok \textbf{Kyslíková bomba} je možné umístit do světa a~pak si ho opět vzít do inventáře. Díky tomu je možné tyto bloky dále používat třeba pro plnění v~\textbf{Plničce kyslíkových bomb}.
Blok je možné rovnou použít (levým tlačítkem myši).
Tento blok zároveň rovnou ukazuje, kolik objemu je využito.

\begin{figure}[!ht]\centering
\includegraphics[ width=140mm]{../img/user/tank/0tankFull}

\caption{Umístitelné předměty -- plná kyslíková bomba}
\label{fig:user_tank_0tankFull}

\end{figure}

\FloatBarrier

Na dalším obrázku vidíme již částečně použitou kyslíkovou bombu. Pokud ji pravým tlačítkem myši sebereme a~otevřeme si inventář a~správnou skupinu (\textbf{Typ: Inventář} v~nastavení skupiny), uvidíme tento blok v~seznamu.

Požité tagy se při přidání do světa a~opětovném sebrání zachovávají.

\begin{figure}[!ht]\centering
\includegraphics[ width=140mm]{../img/user/tank/1tankAfterUse}

\caption{Umístitelné předměty -- použitá bomba}
\label{fig:user_tank_1tankAfterUse}

\end{figure}

\FloatBarrier

Zároveň v~inventáři můžeme vidět přesnou hodnotu naplnění bloku.

\begin{figure}[!ht]\centering
\includegraphics[ width=140mm]{../img/user/tank/2tankInventory}

\caption{Umístitelné předměty -- inventář}
\label{fig:user_tank_2tankInventory}

\end{figure}



\FloatBarrier
%!TEX root = ../../prace.tex

\section{Plnička kyslíkových bomb}

Kyslíkové bomby je potřeba opětovně naplnit, pokud byl jejich obsah spotřebován. Stejně jako u~Kyslíkové bomby, levým tlačítkem myši je možné rovnou doplnit zásoby kyslíku hráče. Pravým se pak otevře ovládací obrazovka.

\begin{figure}[!ht]\centering
\includegraphics[ width=140mm]{../../img/user/filler/0use}

\caption{Plnička kyslíkové bomby -- náhled}
\label{fig:user_filler_0use}

\end{figure}

\FloatBarrier

Ta má v~levé části přiřazený ovladač, případně je možné nejvýše jeden ovladač ze seznamu ovladačů přiřadit. Pokud je přiřazen ovladač, zapnutí bloku se řídí jeho nastavením. V pravé části je možné regulovat spotřebovávanou energii.

Uprostřed je možné vybrat bomby, které jsou v~inventáři hráče, k~naplnění.

\begin{figure}[!ht]\centering
\includegraphics[ width=140mm]{../../img/user/filler/1fill}

\caption{Plnička kyslíkové bomby -- ovládací obrazovka}
\label{fig:user_filler_1fill}

\end{figure}

\FloatBarrier

Pokud je plnička zapnuta, generuje kyslík a~ze své zásoby plní přiřazenou kyslíkovou bombu.

\begin{figure}[!ht]\centering
\includegraphics[ width=140mm]{../../img/user/filler/2filled}

\caption{Plnička kyslíkové bomby -- naplněno}
\label{fig:user_filler_2filled}

\end{figure}



\FloatBarrier
%!TEX root = ../prace.tex

\section{Přepínač}

Přepínač slouží jako ovládání pro světla a plničku

\begin{figure}[!ht]\centering
\includegraphics[ width=140mm]{../img/user/switcher/0switcherGeneral}

\caption{Přepínač - den}
\label{fig:user_switcher_0switcherGeneral}

\end{figure}

\FloatBarrier

Blok umožňuje reagovat na denní dobu - automatické přepínání na definovaný stav, pokud začne den, nebo začne noc.

V levé části le možné přiřazovat ovládané bloky

\begin{figure}[!ht]\centering
\includegraphics[ width=140mm]{../img/user/switcher/switcherControls}

\caption{Přepínač - poledne, zataženo}
\label{fig:user_switcher_switcherControls}

\end{figure}


\FloatBarrier
%!TEX root = ../prace.tex

\section{Světlo}

Světlo má podobné rozhraní jako Plnička. V levé části se přiřazuje ovladač, v pravé se edituje výkon bloku.

\begin{figure}[!h]\centering
\includegraphics[ width=140mm]{../img/user/light/0light}

\caption{Světlo - ovládací obrazovka}
\label{fig:user_light_0light}

\end{figure}

\FloatBarrier

%!TEX root = ../../prace.tex

\section{Kyselý déšť}

Pokud se blíží bouře kyselého deště, nebo právě jedna probíhá, hráč vidí v~levé horní části obrazovky zprávu s~odhadovaným časem a~intenzitou.

To umožňuje strategicky řídit chod svých budov a~případně limitovat spotřebovávané zdroje v~případě očekávaných dlouhotrvajících bouří. Odhadovaný čas je udán v~herním čase.

\begin{figure}[!ht]\centering
\includegraphics[ width=140mm]{../img/user/rain/0rainInfo}

\caption{Kyselý déšť -- info}
\label{fig:user_rain_0rainInfo}

\end{figure}

\FloatBarrier

Pokud hráč není ukrytý v~budově či pod nějakým blokem, dostává zásahy. Dokud má dostatek energie, je schopen odolávat účinkům bouře, v~momentě, kdy mu energie dojde, začne mu ubývat zdraví.


\begin{figure}[!ht]\centering
\includegraphics[ width=140mm]{../img/user/rain/1rainDamage}

\caption{Kyselý déšť -- zásahy}
\label{fig:user_rain_1rainDamage}

\end{figure}

\FloatBarrier

Pokud si hráč doplní energii, začne se mu zdraví obnovovat.

\begin{figure}[!ht]\centering
\includegraphics[ width=140mm]{../img/user/rain/2rainRefill}

\caption{Kyselý déšť -- obnova zdraví}
\label{fig:user_rain_2rainRefill}

\end{figure}


\begin{figure}[!ht]\centering
\includegraphics[ width=140mm]{../img/user/rain/3rainRefill1}

\caption{Kyselý déšť -- obnova zdraví}
\label{fig:user_rain_3rainRefill1}

\end{figure}

\FloatBarrier

Během bouře je též možné pozorovat animaci generátoru energie, kdy je po každém zásahu rozsvícen příslušný čtverec. Tuto animaci lze z~menu vypnout a~pokud má uživatel slabší stroj, tak to i~doporučujeme.

\begin{figure}[!ht]\centering
\includegraphics[ width=140mm]{../img/user/rain/4rainGeneratorAnim}

\caption{Kyselý déšť -- animace zásahů}
\label{fig:user_rain_4rainGeneratorAnim}

\end{figure}


\FloatBarrier

%!TEX root = ../prace.tex

\chapter{Závěr}


\section{Zhodnocení práce}


\section{Budoucí práce}


\begin{itemize}
	\item dynamičtějí mřížka? 20cm je nejspíše dost málo a vyžaduje to dost preciznosti // TODO zkusit pro test 25 či 30 cm a patřičným způsobem upravit velikosti modelů? (nejspíše to musí zůstat hardcoded, ale zkusím se nad tím zamyslet, pokud bude čas)
	\item vlastní sortování v seznamech

\end{itemize}

TODO dotazník?

%%% Seznam použité literatury
\include{literatura}

%%% Obrázky v bakalářské práci
%%% (pokud jich je malé množství, obvykle není třeba seznam uvádět)
%\listoffigures

%%% Tabulky v bakalářské práci (opět nemusí být nutné uvádět)
%%% U matematických prací může být lepší přemístit seznam tabulek na začátek práce.
%\listoftables

%%% Použité zkratky v bakalářské práci (opět nemusí být nutné uvádět)
%%% U matematických prací může být lepší přemístit seznam zkratek na začátek práce.

% Soubor se všemi zkratkami uvedenými v práci

%!TEX root = prace.tex


%\chapter*{Seznam použitých zkratek}
%\addcontentsline{toc}{chapter}{Seznam použitých zkratek}

%Print the glossary
%\printnoidxglossary[sort=use]
\printglossary[type=\acronymtype, title=Seznam použitých zkratek]

%%% Přílohy k bakalářské práci, existují-li. Každá příloha musí být alespoň jednou
%%% odkazována z vlastního textu práce. Přílohy se číslují.
%%%
%%% Do tištěné verze se spíše hodí přílohy, které lze číst a prohlížet (dodatečné
%%% tabulky a grafy, různé textové doplňky, ukázky výstupů z počítačových programů,
%%% apod.). Do elektronické verze se hodí přílohy, které budou spíše používány
%%% v elektronické podobě než čteny (zdrojové kódy programů, datové soubory,
%%% interaktivní grafy apod.). Elektronické přílohy se nahrávají do SISu a lze
%%% je také do práce vložit na CD/DVD. Povolené formáty souborů specifikuje
%%% opatření rektora č. 23/2016.



%!TEX root = ../prace.tex


\appendix
\chapwithtoc{Přílohy}

\renewcommand{\thesection}{\Alph{section}}




\section{Minimální HW konfigurace}

Doporučená minimální sestava (na ní byla hra vyvíjena): 

\begin{center}
	\begin{tabular} { | l | l |}
		\hline
		Procesor: 	&	Intel i7-2630QM @ 2.00GHz \\	\hline
		RAM:		&	12 GB	(8 GB by mělo také stačit) \\	\hline
		Grafika:	&	ATI Radeon HD 6700M \\	\hline
		OS:		&	Win 10 x64  (7 a vyšší by měly být v pohodě) \\
		\hline
	\end{tabular}
\end{center}

\newpage

\section{Požadavky na zprovoznění hry}


Pro spuštění zkompilované hry není potřeba nic speciálního. Je zapotřebí mít stroj s minimální uvedenou konfigurací. Dále je dobré mít nainstalované poslední verze ovladačů HW komponent (hlavně grafiky). 

Pokud je cílem spustit projekt hry ze zdrojových kódů, je potřeba si stáhnout Unreal Engine ve verzi 4.15 (TODO link!). Použití novější verze je možné, ale běžnému uživateli to nedoporučujeme. Mezi verzemi se mohou projevit nekompatibility v kódu, které je případně nutné řešit zásahy přímo do zdrojových kódů hry.
Dále je potřeba mít k dispozici zdrojové kódy ať už z DVD, nebo z tohoto release na GitHubu (TODO link na public repo, release commit).

Dále je zapotřebí vygenerovat solution pravým klikem na uproject file (TODO img!) Pokud tato možnost v kontextové nabídce není, je potřeba provést FIX . Ze zkušenosti autora - toto se mnohdy nemusí podařit. Pokud se nepodaří vygenerovat solution, může stačit otevřít projekt a dát zkompilovat chybějící binárky (TODO img!). Je zapotřebí mít VS 2015 alespoň ve verzi Community.

Pokud i toto selže, ověřte si, prosím, že je možné založit nějaký projekt založený na C++ (todo font), zkompilovat jej a taktéž vygenerovat solution. Pokud se to povede s template, mělo by to fungovat i s tímto projektem.


Dalším krokem je v případě úspěšného otevření ve VS (todo abrreviation) nastavení TCF2 jako výchozího projektu a následné spuštění. Dále by měl následovat krok spuštění Play in Editor v UE.


Pokud vše selže, je možné nalézt příčinu chyby v logu (TODO) v Saved/Logs

\openright
\end{document}
