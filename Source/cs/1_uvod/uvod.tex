%!TEX root = ../prace.tex

\chapter*{Úvod}
\addcontentsline{toc}{chapter}{Úvod}

V době vzniku této práce jsou velice populární hry s otevřeným světem a velkými možnostmi stavění. Autor této práce si je taktéž rád zahraje a rád by touto prací představil svoji vizi dalšího možného rozvoje tohoto žánru.

Když se podíváme na hry jako například Minecraft, Space Engineers (nebo její odnož Medieval Engineers) tak zjistíme, že hra od hráče vyžaduje nejen stavitelské a konstruktérské schopnosti, ale také taktické dovednosti, které hráči umožňují v daném světě přežít. Mnohdy hráč může využít nevšedních technik daných mechanikami dané hry. Příkladem budiž farma na golemy ve hře Minecraft, která využívá bloky lávy, které jsou drženy cedulemi\citep{minecraft_tut_farm}.

V současné době jsou velikosti bloků omezeny na konstantné velikost. Ve hře Minecraft je blok hranově omezen na 1m, hra Space Engineers bloky omezuje dle kategorií od 0.5\,\rm m do 2.5\,\rm m \citep{se_blocks_wiki}.
My bychom se v této práci chtěli zabývat myšlenkou proměnlivé velikosti stavitelných bloků. Hráči by tak mohli ovlivňovat detailnost svých výtvorů, aniž by to muselo mít nutně negativní vliv na dobu nutnou k postavení komplexního, ale zároveň detailního výtvoru. Tato myšlenka však přináší spoustu problémů k řešení, které se však v této práci snažíme vyřešit.

Aby hra byla nebyla pro hráče nudná, měla by také obsahovat nějakého nepřítele - tedy něco, co bude hráče nutit se zlepšovat, překonávat nástrahy a posouvat se dále. I tomuto elementu hry se v této práci budeme věnovat. Zároveň nám to nabízí nové možnosti, jak do hry dodat prvky strategických her. Aby to hráč neměl tak jednoduché, bude muset taktizovat, aby svého nepřítele porazil, nebo alespoň v dané chvíli neprohrál, byť za cenu nějakých ztrát.

\begin{itemize}
	\item V poslední době jsou kromě klasických her typu FPS také různé strategie a hry o přežití
	\item Typ: Minecraft, Space Engineers, Medieval Engineers, Take on Mars, ARK Survival Evolved, novější No man's sky
	\item autor tyto hry má též v oblibě a rád by představil svoji vizi budovatelské hry s prvky strategie a přežití		// TODO tohle by chtělo hodně učesat
\end{itemize}		

// TODO zmínit příběh, který jsem si vymyslel? (cesta za záchranou lidstva atd...)? \\

// možná bych to mohl dát do hry na úvod, poté zobrazit textový tutorial