%!TEX root = ../prace.tex

\chapter{Úvod}

V době vzniku této práce jsou velice populární budovatelské hry s otevřeným světem. Autor této práce si je taktéž rád zahraje a rád by touto prací představil svoji vizi dalšího možného rozvoje tohoto žánru.

Tyto hry se vyznačují tím, že hráč využívá stavebních bloků k tomu, aby rozšiřoval své herní možnosti v těchto světech. Dalším společným prvkem je přítomnost zdraví hráčova avataru (postavy, za kterou hráč hraje) a dalších survival atributů. Dokonce můžeme tvrdit, že je hráč nucen využívat alespoň určité minimum bloků, protože jinak by v herním světem nepřežil a tím hra obvykle končí. \\


Některé hry (kupříkladu \MC{} nebo \TE{}) využívají stavebních bloků jako základního konceptu a tyto bloky pak tvoří celý herní svět. Tím je dosažena jednotná stylizace herního světa. Jiné (\SE{}, \ME{}) využívají kombinaci herních bloků s voxelovou reprezentací světa a tím dosahují vyššího stupně realismu ve hře.
Můžeme však nalézt i další příklady her (// TODO \TM{}, \NI{}, \PN{}, \ARK{}, \NMS{}).

Obvykle je ve hře definován jeden základní rozměr bloku, který je neměnný (\SE{} definuje více velikostí -- ty však nelze vzájemně kombinovat). To však může být problémem, pokud se hráč rozhodne postavit v herním světě nějakou větší a komplexnější strukturu podle reálné či fiktivní předlohy. Pro příklad uveďme některé výtvory ze hry \MC{}:
\begin{itemize}
	\item King's landing z knih Píseň ledu a ohně
	\item Minas Tirith - hlavní město Gondoru z univerza J.R.R. Tolkiena
\end{itemize}

Autoři těchto výtvorů museli volit takové měřítko, aby byly výtvory dostatečně detailní, ale zároveň aby bylo možné výtvor postavit v nějakém rozumném čase. Obecně ale můžeme říct, že čím větších detailů chtěli autoři dosáhnout, tím větší musel celý výtvor být, ale za ceny toho, že velké plochy trvaly o to déle.

V této práci se chceme zaměřit na to, aby hráč mohl ovlivnit velikost stavěných bloků. Tedy aby mohl snadno stavět rozsáhlé struktury, ale zároveň aby se mohl piplat s detaily, pokud je to hráčovým cílem.

\section{Cíle práce}
Tato práce by měla naplnit následující cíle:
\begin{itemize}
	\item Bloky
	\item TODO
\end{itemize}



\begin{code}
  foo->bar( ).
\end{code}

{\tt tt text} a pokračuji díl.


Když se podíváme na hry jako například Minecraft, Space Engineers (nebo její odnož Medieval Engineers) tak zjistíme, že hra od hráče vyžaduje nejen stavitelské a konstruktérské schopnosti, ale také taktické dovednosti, které hráči umožňují v daném světě přežít. Mnohdy hráč může využít nevšedních technik daných mechanikami dané hry. Příkladem budiž farma na golemy ve hře Minecraft, která využívá bloky lávy, které jsou drženy cedulemi\citep{minecraft_tut_farm}.

V současné době jsou velikosti bloků omezeny na konstantné velikost. Ve hře Minecraft je blok hranově omezen na 1m, hra Space Engineers bloky omezuje dle kategorií od 0.5\,\rm m do 2.5\,\rm m \citep{se_blocks_wiki}.
My bychom se v této práci chtěli zabývat myšlenkou proměnlivé velikosti stavitelných bloků. Hráči by tak mohli ovlivňovat detailnost svých výtvorů, aniž by to muselo mít nutně negativní vliv na dobu nutnou k postavení komplexního, ale zároveň detailního výtvoru. Tato myšlenka však přináší spoustu problémů k řešení, které se však v této práci snažíme vyřešit.

Aby hra byla nebyla pro hráče nudná, měla by také obsahovat nějakého nepřítele - tedy něco, co bude hráče nutit se zlepšovat, překonávat nástrahy a posouvat se dále. I tomuto elementu hry se v této práci budeme věnovat. Zároveň nám to nabízí nové možnosti, jak do hry dodat prvky strategických her. Aby to hráč neměl tak jednoduché, bude muset taktizovat, aby svého nepřítele porazil, nebo alespoň v dané chvíli neprohrál, byť za cenu nějakých ztrát. xxd

\begin{itemize}
	\item V poslední době jsou kromě klasických her typu FPS také různé strategie a hry o přežití
	\item Typ: Minecraft, Space Engineers, Medieval Engineers, Take on Mars, ARK Survival Evolved, novější No man's sky
	\item autor tyto hry má též v oblibě a rád by představil svoji vizi budovatelské hry s prvky strategie a přežití		// TODO tohle by chtělo hodně učesat
\end{itemize}		

// TODO zmínit příběh, který jsem si vymyslel? (cesta za záchranou lidstva atd...)? \\

// možná bych to mohl dát do hry na úvod, poté zobrazit textový tutorial

