%!TEX root = ../prace.tex

\chapter{Programátorská dokumentace}

%!TEX root = ../../prace.tex

\section{Počáteční inicializace projektu}

Pokud je cílem spustit projekt hry ze zdrojových kódů, je potřeba si stáhnout \UE{} ve verzi \TT{4.15.3}. Na hlavní stránce \UE{} je potřeba stáhnout \textit{Epic Games Launcher} (ke stažení zde~\citep{ue_download}). Po vytvoření uživatelského účtu a~následnému přihlášení do aplikace bude vidět obrazovka podobná obrázku \ref{fig:ueLauncher}:


\begin{figure}[!ht]\centering
\includegraphics[ width=140mm]{../img/program/ueLauncher}

\caption{Epic Games Launcher}
\label{fig:ueLauncher}

\end{figure}

\FloatBarrier

Pokud není vidět \UE{} dané verze v~seznamu verzí, je potřeba jej přidat kliknutím na tlačítko \textit{Add Versions} a~případném následném výběru správné verze. Přidaná verze se zobrazí stejným způsobem, jako je na obrázku \ref{fig:ueLauncher} zobrazen \UE{} verze \TT{4.16.2}. Posledním krokem je samotná instalace stisknutím tlačítka \textit{Install}. Použití novější verze \UEu{} je možné, ale běžnému uživateli to nedoporučujeme. Mezi verzemi se mohou projevit nekompatibility v~kódu, které je mnohdy nutné řešit zásahy přímo do zdrojových kódů hry.

Dále je potřeba mít k~dispozici zdrojové kódy ať už z~přiloženého DVD, nebo z~tohoto release na GitHubu (odkaz~\citep{gh_finalRelease}). Na DVD se soubory nachází ve složce \textit{Source}. Ve složce zdrojových kódů hry se nachází soubor \TT{TauCetiF2.uproject}, což je soubor projektu v~\UEu{}. Další postup závisí na tom, zda je cílem otevřít hru v~Editoru \UEu{}, nebo vytvořit projekt pro Visual Studio, z~něhož je pak možné Editor spustit. Oba přístupy jsou záměnné, protože vytvoření projektu pro Visual Studio po spuštění Editoru je také validní postup. Pro úspěšnou kompilaci a~spuštění hry je zapotřebí mít Visual Studio 2015 alespoň ve verzi \textit{Community} a~mít u~něj zapnutou podporu jazyka \CPP{} (to se řeší během instalace Visual Studia).
 
\subsubsection{Přímé vytvoření projektu pro Visual Studio}
Projekt pro Visual Studio je možné vytvořit kliknutím pravým tlačítkem myši na soubor \TT{TauCetiF2.uproject} a~následnou volbou \textit{Generate Visual Studio project files} (viz obrázek \ref{fig:generateProjectFiles}). 

\begin{figure}[!ht]\centering
\includegraphics[ width=80mm]{../img/program/generateProjectFiles}

\caption{Vytvoření projektu hry}
\label{fig:generateProjectFiles}

\end{figure}
\FloatBarrier

Tento postup je v~praxi běžně používaný. Celý projekt je totiž vytvořen na základě informací ze složky \textit{Source}. Mimo jiné má tento přístup tu výhodu, že při použití systému správy verzí (jako například GITu) se tyto vygenerované soubory nemusí verzovat -- každý vývojář si je vytvoří na svém počítači.

Pokud v~kontextové nabídce chybí možnosti pro \UE{}, je potřeba provést asociaci s~\TT{uproject} soubory. Tato asociace by měla být provedena automaticky, nejpozději po vytvoření libovolného \CPP{} projektu z~kterékoliv šablony. Ze zkušenosti autora se to mnohdy nemusí podařit. Pokud se nepodaří vygenerovat projekt hry, může pomoci druhý přístup otevření projektu, který je popsán v~následující části.

Předposledním krokem je v~případě úspěšného vytvoření projektu hry jeho otevření ve \textit{Visual Studiu}, nastavení \textit{TauCetiF2} jako hlavní spustitelnou část projektu (viz obrázek \ref{fig:vs}) a~následné spuštění editoru hry.


\begin{figure}[!ht]\centering
\includegraphics[ width=110mm]{../img/program/vs}

\caption{Vygenerování projektu hry}
\label{fig:vs}

\end{figure}

\FloatBarrier



\subsubsection{Přímé spuštění Editoru}
Druhý postup zahrnuje prosté spuštění Editoru dvojklikem na soubor projektu. Pokud soubor projektu nemá modrou ikonu jako na obrázku \ref{fig:generateProjectFiles}, není tento typ souborů asociovaný s~Unreal Editorem. Pak je nutné otevřít Editor přes \textit{Epic Games Launcher}, najít a~otevřít soubor projektu. Protože hra využívá systému modulů, při prvním spuštění se uživateli zobrazí hláška o~tom, že některé moduly je potřeba zkompilovat. Tuto akci necháme provést. Pokud by kompilace modulů selhala, nejspíše máme problém s~Visual Studiem. V~takovém případě doporučujeme postup níže v~rámci \textit{kroků poslední záchrany}.

Pokud je úspěšně otevřen a~načten Editor, zbývá poslední krok -- výběr obnovy projektu, jak je naznačeno na obrázku \ref{fig:ue_refresh}.

\begin{figure}[!ht]\centering
\includegraphics[ width=80mm]{../img/program/ue_refresh}

\caption{Vygenerovaný projekt hry}
\label{fig:ue_refresh}

\end{figure}

\FloatBarrier


\subsubsection{Kroky poslední záchrany}
Pokud všechny námi uvedené postupy selžou, je potřeba si ověřit, že je možné založit libovolný projekt (založený na \CPP{}) z~dodávaných šablon, zkompilovat jej a~vygenerovat projekt pro Visual Studio. Pokud se to povede s~projektem z~nějaké připravené šablony, mělo by to fungovat i~s~naší hrou. V~případě, že ani tento postup nefunguje, je potřeba zjistit příčinu z~kompilačních Logů. Řešení této nefunkčnosti jde mimo téma této práce, takže se jím nebudeme dále zabývat.







%!TEX root = ../prace.tex

\section{Struktura kódu}

Program se dělí do několika modulů. Jejich struktura je zachycena na obrázku \ref{fig:obrStruktura_DependencyDiag}.

\begin{figure}[h!]\centering
\includegraphics[ height=70mm]{../img/dependencyDiag.png}

\caption{Diagram závislostí modulů projektu.}
\label{fig:obrStruktura_DependencyDiag}

\end{figure}

Když si otevřeme zdrojovou sln projektu, tak uvidíme, že každý modul má několik podřízených věcí: 

- složku Private

- složku Public

- soubory .Build.cs, .h, .cpp

Každý modul pak má hlavičkové soubory ve složce Public, implementaci tříd pak ve složce Private. Poslední tři soubory jsou kvůli UBT a // TODO použitá zkratka 
použitím herních modulů v rámci UE



\subsection{Modul Commons}
%!TEX root = ../../prace.tex

\section{Modul Commons (C++)}

Tento modul je základním modulem, který na jednom místě definuje všechny potřebné informace, které využívají ostatní moduly. Jedná se zejména o~definici herních konstant, či definice všech sdílených enumerátorů. Najdeme zde také prapředka použité herní instance. Tuto vlastní implementaci herní instanci využijeme pro ukládání nalezených bloků.

\subsection{Herní definice a~konstanty}



V souboru \TT{GameDefinitions.h}jsou definovány všechny herní konstanty. Například je zde definována velikost jednotkové krychle, velikost použitého světa, vztah mezi délkou dne herního světa a~počtem uplynulých reálných sekund, převody mezi energií, kyslíkem, zdraví a~jednotkou zásahu kyselého deště. Taktéž jsou zde definovány konstanty, které využívá technika obrysů objektu (todo link na outline). Dále jsou zde definovány konstanty ID implementovaných bloků, abychom s~nimi mohli pracovat i~v~kódu.

\subsection{Herní instance}

Herní instance \TT{TCF2GameInstance.h} se chová jako návrhový vzor \textit{Singleton} a~jako jediná zůstává vždy stejná po celou dobu běhu hry (jak uchovávat globální data \citep{ue_gameInstance}). Proto se využívá například pro uchovávání dat při přechodu mezi jednotlivými Levely a~my toho také využijeme. Zároveň se tato třída dá využít pro implementaci delegátů, kterými je možné vyvolat nějakou událost a~libovolný prvek z~herního světa může tuto událost obsloužit. My toho využijeme u~reakce na denní cyklus u~bloku \textit{Přepínače}.

Dalším důležitým bodem pro nás bude, že tato třída si bude držet referenci na všechny nalezené bloky. Z předchozího textu již víme, že bloky je potenciálně možné rozšířit o~DLC (TODO budem to tu řešit?), takže je nutné, abychom si nalezené bloky a~jejich definice udrželi v~paměti i~při přechodu mezi levely. K tomu slouží proměnná \TT{BlockHolder} , která sice  drží referenci na objekt definovaný v~modulu \textit{Blocks}, ale kvůli zpětným referencím mezi moduly (které nejsou povolené) musíme zde použít dostupného předka.

Kód tedy bude vypadat následovně:
\begin{code}
	UPROPERTY(Transient)
		UObject* BlockHolder;

	UFUNCTION(BlueprintCallable, Category = "TCF2 | GameInstance")
		void SetHolderInstance(UObject* holder);
\end{code}



 Parametr \TT{Transient} u~makra \TT{UPROPERTY} znamená, že daná proměnná bude vždy nastavena na svoji výchozí hodnotu. V tomto případě je to použito spíše z~důvodu zachování konzistence napříč projektu, ale zjednodušeně bychom důsledky mohli popsat následovně -- pokud bude nějaký Blueprint dědit z~nějaké \CPP{} třídy, tak vývojář může nastavit výchozí hodnoty properties. Tyto hodnoty jsou pak serializovány do \CDO{}, cože je \textit{Class Default Object} (otázka na Answers Unreal Engine \citep{ue_cdo}). Během procesu vytváření nové instance objektu, který vychází z~daného Blueprintu pak budou tyto hodnoty naplněny během fáze inicializace properties (popis životního cyklu actorů \citep{ue_actor_life}). V konečném důsledku by pak byla tato hodnota nějakým způsobem naplněna. Pokud chceme vynutit, aby tato property nebyla serializována do CDO, tak ji označíme jako \TT{Transient}.

V průběhu hry pak jednou naplníme tuto property pomocí metody\\ \TT{SetHolderInstance}, do které předáme referenci na korektně inicializovanou instanci třídy \TT{BlockHolder} z~modulu \textit{Blocks}. Pak si můžeme odkudkoliv získat aktuální herní instanci, přetypovat na \TT{TCF2GameInstance} a~získat si (přetypovanou) referenci na \TT{BlockHolder}. Z něho pak již můžeme získávat informace o~všech dostupných blocích.

\subsection{Enumerátory}

\TT{Enums.h}
Tento soubor slouží jako jednotné umístění pro všechny výčtové typy (enumerátory), které se používají napříč celým projektem. Neznamená to, že nutně obsahuje všechny -- některé třídy mohou využívat své specifické enumerátory, které ale nemusí být umístěny v~tomto globálně dostupném modulu.


\subsection{Helpery}

\TT{CommonHelpers.h} Tato třída poskytuje metody pro práci s~konfigurací. Statické metody umožňují načítat a~ukládat konfigurační položky typu \TT{float}, \TT{bool} a~\TT{string}.
Aby byla práce co nejjednodušší, metody přijímají enumerátor\linebreak[4]\TT{EGameUserSettingsVariable}. Třída pak sama na základě hodnoty tohoto enumerátoru použije správný klíč (který je textový) a~tak může ukládat či vracet hodnotu daného typu z~konfiguračního souboru.


\subsection{Modul Game Save}
%!TEX root = ../prace.tex

Modul GameSave slouží  k ukládání a načítání informací o probíhající hře do binárního formátu. K tomu používáme streamové operátory $<<$, které jsou v tomto případě implementovány tak, že je možné je použít jak pro ukládání, tak pro načítání. // TODO link na tutorial

Díky tomuto přístupu tak můžeme definovat celou strukturu výsledného binárního souboru na jednom místě a tedy rozšiřování uložené hry je triviální. Co si ovšem musíme pohlídat je to, abychom si drželi informaci o verzi uloženého souboru. V našem případě, pokud se bude lišit verze načteného souboru a uložená konstanta v programu, save prostě odmítneme (a dokonce smažeme). V produkčním prostředí bychom si mazání nemohli dovolit, ale museli bychom save ignorovat a uživateli zobrazit nějakou hlášku o tom, že verze souboru není podporovaná. My jsme se však v tomto případě rozhodli save mazat, protože jsme očekávali, že během vývoje hry se bude binární struktura savu často rozšiřovat. Po každé iteraci jsme si savy prostě vytvořili nové.

Co by se stalo, kdybychom se snažili načíst save jiné verze? Celá hra by nejspíše byla ukončena s chybou, protože by se pokoušela číst neplatná data a/nebo by očekávala nějaká data tam, kde žádná nejsou. Tím bychom četli z neplatné lokace.


- popsat save game carrier ( + výsledný formát)

- zdůraznit, že se jedná o holá data, UObjekty si pak vytváří každý modul sám

- popsat NewSaveGameHolder, rozšiřování pevných savů 

- popsat *Archive helpers 


\subsection{Modul Blocks}
%!TEX root = ../../prace.tex

\section{Modul Blocks (C++)}



Modul bloků obsahuje podstatné informace o~tom, jak hra pracuje s~bloky, jak se tyto bloky skládají do herního světa, jaké jsou jejich komponenty apod. Také je v~tomto modulu možné nalézt specifické implementace jednotlivých bloků.

V dalším textu se budeme odkazovat na složky. Odkazujeme se tím do složek \TT{/Source/Blocks/Public} a~jejich \TT{Private} implementací. Strukturu bychom mohli shrnout následovně:

\begin{enumerate}
	\item Definice bloků (složka \TT{Definitions})
	\item Třídy s~popisem bloků (složka \TT{Info})
	\item Systém ukládání a~načítání bloků (složka \TT{Helpers})
	\item Rozhraní, které mohou bloky implementovat (složka \TT{Interfaces})
	\item Komponenty, kterými bloky rozšiřují svoji základní funkcionalitu (složka \TT{Components})
	\item Implementace jednotlivých bloků (složky \TT{BaseShapes}, \TT{Special})
	\item Stromové struktury herního světa (složka \TT{Tree})
\end{enumerate}
 

\subsection{Definice bloků}
V této složce se nachází všechny definiční soubory bloků. Definiční soubor obsahuje pouze popis datové struktury a~nějakou minimální funkcionalitu (kupříkladu získání korektního vektoru velikosti v~závislosti na tom, zda má definice daného bloku nastavenou vlastní velikost). Jednotlivé konkrétní instance s~daty jsou pak definovány na straně editoru (tuto funkcionalitu jsme již zmínili v~kapitole \ref{subsec:hb}). Konstanty (například minimální a~maximální škálování) je pak možné měnit v~editoru a~není vyžadována rekompilace projektu hry. 


\subsection{Třídy s~popisem bloků}
Tyto třídy popisují už konkrétní instance bloků v~rámci hry. Jejich hodnoty jsou pak v~mezích definovaných v~definičních třídách. Tyto třídy jsou pak předmětem ukládání a~načítání. Dalším důležitým prvkem je \textit{BlockHolder}, který slouží pro nalezení bloků. 


\subsection{Komponenty bloků}
Komponenty bloků vycházejí z~poznatků v~části \ref{sec:komponents}. Rádi bychom zde zmínili zajímavou část převodů kyslíku a~energie.


\subsubsection{Převádění kyslíku a~energie}
Protože \UE{} umožňuje hrám pracovat s~více výpočetními vlákny, musíme zajistit konzistenci dat při převodech kyslíku či energie. Můžeme pro to využít primitiva pro zamykání \TT{FCriticalSection}. Kritickou sekci pak budeme korektně zamykat a~odemykat (stejně jako u~klasického vícevláknového programování). Algoritmy pro vkládání a~získání kyslíku budou mít následující signaturu (pro energii bude signatura stejná):

\begin{code}
    // dodej kyslík komponentě
    bool UOxygenComponent::PutAmount(float aviable,
                                     float& actuallyPutted)

    // získej kyslík z~komponenty                                     
    bool UOxygenComponent::ObtainAmount(float requested,
                                        float& actuallyObtained,
                                        bool requireExact)
\end{code}


Princip je prostý -- metody vrací \TT{bool} jakožto hodnotu, zda bylo možné operaci korektně provést. Parametry předávané \textit{referencí} pak v~případě úspěchu obsahují hodnotu skutečně vloženého či získaného kyslíku. Poslední parametr u~metody pro získání kyslíku značí, zda je vyžadované přesné množství. Pokud kyslíková komponenta obsahuje méně kyslíku, než je požadované množství a~je požadované přesně zadané množství, převod nebude úspěšný a~metoda vrátí \TT{false}. Pokud nebude požadované přesně dané množství, skutečně získané množství může být menší a~je na volajícím, aby se tomuto faktu přizpůsobil.


\subsection{Implementace bloků}
\label{subsec:blImp}

Všechny herní bloky dědí ze základní třídy \TT{ABlock}. Dále jsme zavedli dělení na \TT{BaseShapes} a~\TT{Special}, přečemž jak struktura zdrojových kódů, tak struktura v~rámci \UEu{} tuto strukturu dodržuje. Toto dělení jsme již definovali v~části \ref{subsubsec:vlastnosti}.




\subsection{Modul Inventory}
%!TEX root = ../prace.tex



\section{Modul Inventory (C++)}

Modul inventáře byl vyčleněn do samostatné části. Je to hlavně jako ukázka možného členění do modulů. Navíc časem by se mohl tento modul rozšířovat jak by rostla kompelxicita správy inventáře.

Nejdůležitější inventory component


\subsection{Tag group}
nejnižší úroveň, odpovídá 'nebo'

\subsection{Inventory tag group}

celá skupina, odpovídá 'A zároveň'

\subsection{Inventory tags}

sdružuje všechny banky

\subsection{Inventory component}

celá komponenta, která je pak navázaná na hráčův charakter

definuje delegáty notifikující o změnách v akivní skupině, po filtrování apod.

na této úrovni se řeší aktualizace cache buildable i inventorybuildable při změnách, zároveň poskytuje možnost clear cache pro volání shora (BP)




\subsection{Modul TauCetiF2}
%!TEX root = ../../prace.tex

\section{Modul TauCetiF2 (C++)}
Primární herní modul \textit{TauCetiF2} obsahuje mimo jiné implementaci herního objektu \textit{WorldController}, který jsme popisovali v~části \ref{subs:wc}. Dále obsahuje bázové třídy pro většinu UI oken hry. Zde bychom rádi zmínili \TT{SynchronizeWidget}, který dědí z~\TT{UUserWidget}, tedy základního uživatelského widgetu (UI prvku, který je možné zobrazit). Tato třída pak implementuje metodu \TT{SynchronizeProperties}, díky níž získávají potomci této třídy možnost okamžité aktualizace po změně libovolné vlastnosti v~Editoru widgetů. Tato vlastnost běžně dostupná není, ale tímto \uv{trikem} lze snadno získat možnost ihned vidět změny v~rámci daného editovaného UI prvku.
 





%!TEX root = ../../prace.tex

\section{Struktura projektu v~Unreal Enginu}
\label{sec:ueStructure}

- ukázat jak se to dělí v~UE editoru, obrázky jak proudí informace, data flow

Struktura složek odpovídá /Content 

TODO  popsatco je BP co je CPP a~bude později 

\begin{itemize}
	\item Složka XY
\end{itemize}


TODO hodně obrázky, vztahy

\subsection{Mapy}

Základní částí projektu v~\UE{u} je \textit{herní mapa}. Ta obsahuje všechny důležité herní objekty.

popsat jednotlivé mapy, jaký je jejich cíl

- Game Loader

zavední hry

- MainMenu

menu, má sublevel MainMap (zobrazení bloků ve hře jako náhledu světa)

- Loading Screen

nahrávání, obsahuje mainmap jako streamovanej level

- Main map

\subsubsection{terén}

 (2 části), terén má speciální tag (označovatelnost, resp. získání cíle pro RayTracing)

\subsubsection{World Controller}

bloky ve světě, napojení na hlavní BP levelu

\subsubsection{GameElectricityPawn}

elektrika

\subsubsection{Game Weather Pawn}

počasí

\subsubsection{SkyBP}

denní cyklus

\subsubsection{AmbientSound}

řeší hudbu ve hře

\subsubsection{Tutorial}

řeší tutorial



\subsection{Mainmap -- průběh nahrávání}

popsat jak je to udělané

\subsection{Charakter}

výchozí spawn z~gamemodu, v~mainBP levelu Posses

\subsection{World Controller}

bloky ve světě, napojení na hlavní BP levelu


\subsection{bloky}

Kde a~jak definované, obrázek, jak se to definuje konstanty


\subsection{GameElectricityPawn}

elektrika, AI controller

\subsection{Game Weather Pawn}

počasí, AI controller, BT počasí + komponenty

\subsection{SkyBP}

denní cyklus -- popis

\subsection{AmbientSound}

řeší hudbu ve hře

\subsection{Tutorial}

řeší tutorial, používá widgety


\subsection{Lokalizacer}

nastínit lokalizaci



\subsection{UI}
nastínit UI -- základní widgety atd.





\subsection{Backlog}

Zde popsat jak jsem to celé implementoval a proč

Popsat jednotlivé moduly a nakreslit diagram vztahů mezi nimi

Popsat strukturu save gamu + důvod proč jsem to tak udělal
+popsat načíttání savů + systémových savů


Popsat jednotlivé C++ třídy a jejich odvozené Blueprintové deriváty + přidat případné obrázky z BL kódu (např. BlueprintImplementable event, který se zavolá jak na C++ tak i na BP) 

Udělat rozbor BT počasí + mechaniku počasí + denního cyklu
popsat řízení osvětlení dle počasí

Udělat rozbor bloků, škálování, konfigurace, datovou strukturu, implementaci dynamických textur, zvýraznění

Popsat mechaniku Selector - SelectTarget + napojení na Builder

Popsat mechaniku používání objektů + zvýraznění

Popsat mechaniku Inventáře

// TODO vymyslet vhodné pořadí, abych neskákal mezi prvky, toto pořadí dodržet i v předchozích kapitolách

-> Mám svět, ten má v sobě bloky, ty jsou v nějaké stromové struktuře, bloky mají komponenty, které přes tuto strukturu mohou na sebe vázat
Svět má také počasí se svojí vlastní strukturou, vuyžívající podobnosti s bloky (2D KD strom s Heapem na listech)

-> hráč může to a tamto, díky inventáři se dostane na bloky, a díky selectoru je pak můževložit do světa skrz World controller (zmíněno v předchozím)
->zároveň jsou všechny entity savovatelné 


-> Popsat struktury Widgetů, zmínit použití Synchronize Widgetu, implementaci mechaniky stackovatelných widgetů

-> popsat implementaci hudby

-> TODO otestovat možnost nového bloku v rámci DLC
->Zmínit zároveň, že s tímto by šlo tweakovat nastavení hry



// TODO obrázky s konfiguračními ukázkami do příloh (např. jak se definuje Blok z UE


