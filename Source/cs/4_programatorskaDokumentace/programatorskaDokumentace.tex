%!TEX root = ../prace.tex

\chapter{Programátorská dokumentace}

Zde popsat jak jsem to celé implementoval a proč\\

Popsat jednotlivé moduly a nakreslit diagram vztahů mezi nimi\\

Popsat strukturu save gamu + důvod proč jsem to tak udělal
+popsat načíttání savů + systémových savů


Popsat jednotlivé C++ třídy a jejich odvozené Blueprintové deriváty + přidat případné obrázky z BL kódu (např. BlueprintImplementable event, který se zavolá jak na C++ tak i na BP) \\

Udělat rozbor BT počasí + mechaniku počasí + denního cyklu
popsat řízení osvětlení dle počasí

Udělat rozbor bloků, škálování, konfigurace, datovou strukturu, implementaci dynamických textur, zvýraznění

Popsat mechaniku Selector - SelectTarget + napojení na Builder

Popsat mechaniku používání objektů + zvýraznění

Popsat mechaniku Inventáře

// TODO vymyslet vhodné pořadí, abych neskákal mezi prvky, toto pořadí dodržet i v předchozích kapitolách

-> Mám svět, ten má v sobě bloky, ty jsou v nějaké stromové struktuře, bloky mají komponenty, které přes tuto strukturu mohou na sebe vázat
Svět má také počasí se svojí vlastní strukturou, vuyžívající podobnosti s bloky (2D KD strom s Heapem na listech)

-> hráč může to a tamto, díky inventáři se dostane na bloky, a díky selectoru je pak můževložit do světa skrz World controller (zmíněno v předchozím)
->zároveň jsou všechny entity savovatelné 


-> Popsat struktury Widgetů, zmínit použití Synchronize Widgetu, implementaci mechaniky stackovatelných widgetů

-> popsat implementaci hudby

-> TODO otestovat možnost nového bloku v rámci DLC
->Zmínit zároveň, že s tímto by šlo tweakovat nastavení hry



// TODO obrázky s konfiguračními ukázkami do příloh (např. jak se definuje Blok z UE


