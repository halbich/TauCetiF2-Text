%!TEX root = ../prace.tex

\section{Modul Game Save (C++)}

Modul GameSave slouží k ukládání a načítání informací o probíhající hře do binárního formátu. K tomu používáme streamové operátory $<<$, které jsou v tomto případě implementovány tak, že je možné je použít jak pro ukládání, tak pro načítání. // TODO link na tutorial

Díky tomuto přístupu tak můžeme definovat celou strukturu výsledného binárního souboru na jednom místě a tedy rozšiřování uložené hry je triviální. Co si ovšem musíme pohlídat je to, abychom si drželi informaci o verzi uloženého souboru. V našem případě, pokud se bude lišit verze načteného souboru a uložená konstanta v programu, save prostě odmítneme (a dokonce smažeme). V produkčním prostředí bychom si mazání nemohli dovolit, ale museli bychom save ignorovat a uživateli zobrazit nějakou hlášku o tom, že verze souboru není podporovaná. My jsme se však v tomto případě rozhodli save mazat, protože jsme očekávali, že během vývoje hry se bude binární struktura savu často rozšiřovat. Po každé iteraci jsme si savy prostě vytvořili nové.

Co by se stalo, kdybychom se snažili načíst save jiné verze? Celá hra by nejspíše byla ukončena s chybou, protože by se pokoušela číst neplatná data a/nebo by očekávala nějaká data tam, kde žádná nejsou. Tím bychom četli z neplatné lokace.




\subsection{GameSaveInterface}

\TT{UGameSaveInterface.h} 
Tento soubor definuje rozhraní, které je možné implementovat a používat v Blueprintech (struktura vychází z tutoriálu na Unreal Engine Wiki\citep{ue_interfaces_tut}). Rozhraní definuje dvě veřejné metody, které musí actoři v UE při implementaci tohoto rozhraní implementovat:

\begin{code}
UFUNCTION(BlueprintNativeEvent, BlueprintCallable, ... )
	bool SaveGame();

UFUNCTION(BlueprintNativeEvent, BlueprintCallable, ... )
	bool LoadGame();
\end{code}

Hlavní Blueprint levelu pak během svého běhu určité actory přetypuje na tento interface a bude s nimi dále pracovat. Podrobněji tato funkcionalita bude popsána v Blueprintové části.

\subsection{FFileVisitor}

\TT{FFileVisitor.h}

Tento soubor obsahuje implementaci návrhového vzoru Visitor pro získání všech herních savů z nějaké zadané složky. Implementace vychází z internetové diskuse na téma procházení adresářů \citep{ue_iterate_dir}.

\subsection{Helpers}

\TT{SaveHelpers.h}
Tento soubor řeší získání seznamu všech uložených her a využívá k tomu implementaci FileVisitoru z předchozí části.

\subsection{Kontejner s uloženou hrou}
\TT{SaveGameCarrier.h} Tato třída je v zásadě přepravkou pro data s možností ukládání a načítání dat do binárního formátu. Navíc umožňuje během ukládání v době vývoje vypsat vlastnosti právě ukládané hry do konzole tak, abychom mohli během vývoje snadno tento výstup zkopírovat a vytvořit pevně implementované hry. Tyto pevně implementované hry pak nebudou data vracet po přečtení nějakého binárního souboru, ale budou vracet pevně nastavená data. Vytváření těchto pevně vytvořených her pak bylo o dost jednodušší.

Přepravka obsahuje pouze holá data, tedy nevytvářejí se žádné nové instance herních objektů. To je z toho důvodu, že nemůžeme použít následující kód:

\begin{code}

// vlastnost kontejneru
TArray<UBlockInfo> UsedBlocks;

// během ukládání
void USaveGameCarrier::SaveLoadData(
	FArchive& Ar,
	USaveGameCarrier& carrier,
	TArray<FText>& errorList,
	bool bFullObject)
{
	Ar << carrier.UsedBlocks;
}
\end{code}
Museli bychom mít referenci na datový typ \TT{UBlockInfo}, který je ale definovaný v modulu Blocks, který musí nutně modul GameSave referencovat. Tudíž zde použijeme následující konstrukci:

\begin{code}
// vlastnost kontejneru
TArray<FBlockInfo> usedBlocks;

// během ukládání
void USaveGameCarrier::SaveLoadData(
	FArchive& Ar,
	USaveGameCarrier& carrier,
	TArray<FText>& errorList,
	bool bFullObject)
{
	Ar << carrier.usedBlocks;
}
\end{code}

Třídu dědící z UObjektu jsme nahradili strukturou \TT{FBlockInfo}, která sama slouží pouze jako přepravka na data. Vyšší vrstva, tedy modul Blocks si těmito daty naplní své vlastní objekty, které se pak dále využívají ve hře. A naopak, před samotným uložením se postará o to, aby bylo toto pole korektně naplněno všemi daty určenými k uložení. O samotnou serializaci a deserializaci dat do a z přepravky se stará pouze modul GameSave.

Celý systém vychází z tutoriálu \citep{ue_save_system} je postaven na tom, že v C++ je možné přetěžovat operátory, mimo jiné i \TT{<<}. Této vlastnosti je využito tak šikovně, že v závislosti na volání funkce buď zapisuje do archivu, nebo z něj čte, ale pořád se jedná o jediný zápis jedné funkce. To je výhodné, protože to předchází chybám, které by mohly vzniknout při použití 2 metod -- jedné čtecí, jedné zapisovací. Chybám typu přehození dvou datových typů (což by v případě typů různých velikostí znamenalo následné špatné pochopení binárních dat), nebo kupříkladu prohození dvou vlastností stejného typu, což by vytvářelo těžko odhalitelné situace změn hodnot ve hře.

Tento systém je použit pro všechny rozšířené části herního savu -- Bloky, Inventář, Počasí -- a všechny používají podobný způsob práce. Jedná se o definici struktur, tedy samotných kontejnerů a dále pak jeden soubor pojmenovaný \TT{*ArchiveHelpers.h}, kde je popsána vlastní struktura daných kontejnerů v Archivu. Hvězdičku v tomto případě bereme opravdu jako zástupný symbol. Navíc, některé objekty mohou řídit archivaci dle nějakých podmínek nastavených shora. Příkladem budiž definice serializace elektrické komponenty:

\begin{code}
// přetížení, které se nevolá vždy
FORCEINLINE FArchive& operator<<(
	FArchive &Ar,
	FPoweredBlockInfo& componentInfo)
{
	Ar << componentInfo.IsOn;
	Ar << componentInfo.AutoregulatePower;
	Ar << componentInfo.PowerConsumptionPercent;
	return Ar;
}

FORCEINLINE FArchive& operator<<(
	FArchive &Ar, 
	FElectricityComponentInfo& componentInfo)
{
	Ar << componentInfo.CurrentObjectEnergy;
	Ar << componentInfo.HasPoweredBlockInfo;

	// pokud máme navíc rozšiřující data, přidáme další data
	// tím efektivně zavoláme metodu uvedenou výše
	if (componentInfo.HasPoweredBlockInfo)
		Ar << componentInfo.PoweredBlockInfo;

	return Ar;
}
\end{code}

Jak vidíme z kódu, celý kód se chová korektně jak při serializaci, tak i při deserializaci. Je pouze potřeba vzít v úvahu, že vyšší struktury, které pak budou tyto data používat pro vlastní inicializaci, musí také brát v potaz podmínku. Pokud není splněna, tak svázaná proměnná (v tomto případě \TT{PoweredBlockInfo}) nebude obsahovat platná data.

\subsection{NewGameSaveHolder}

\TT{NewGameSaveHolder.h} Tato třída je hlavní třídou, se kterou hra před načítáním pracuje. Definuje seznam napevno zabudovaných levelů a obsahuje jejich implementaci.









