%!TEX root = ../prace.tex

Modul GameSave slouží  k ukládání a načítání informací o probíhající hře do binárního formátu. K tomu používáme streamové operátory $<<$, které jsou v tomto případě implementovány tak, že je možné je použít jak pro ukládání, tak pro načítání. // TODO link na tutorial

Díky tomuto přístupu tak můžeme definovat celou strukturu výsledného binárního souboru na jednom místě a tedy rozšiřování uložené hry je triviální. Co si ovšem musíme pohlídat je to, abychom si drželi informaci o verzi uloženého souboru. V našem případě, pokud se bude lišit verze načteného souboru a uložená konstanta v programu, save prostě odmítneme (a dokonce smažeme). V produkčním prostředí bychom si mazání nemohli dovolit, ale museli bychom save ignorovat a uživateli zobrazit nějakou hlášku o tom, že verze souboru není podporovaná. My jsme se však v tomto případě rozhodli save mazat, protože jsme očekávali, že během vývoje hry se bude binární struktura savu často rozšiřovat. Po každé iteraci jsme si savy prostě vytvořili nové.

Co by se stalo, kdybychom se snažili načíst save jiné verze? Celá hra by nejspíše byla ukončena s chybou, protože by se pokoušela číst neplatná data a/nebo by očekávala nějaká data tam, kde žádná nejsou. Tím bychom četli z neplatné lokace.


- popsat save game carrier ( + výsledný formát)

- zdůraznit, že se jedná o holá data, UObjekty si pak vytváří každý modul sám

- popsat NewSaveGameHolder, rozšiřování pevných savů 

- popsat *Archive helpers 