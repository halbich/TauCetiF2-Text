%!TEX root = ../prace.tex

Modul bloků obsahuje podstatné informace o tom, jak hra pracuje s bloky, jak se tyto bloky skládají do herního světa, jaké jsou jejich komponenty atd. 

- základní definice bloku je v (Block.h)

- block.h definuje hromadu společných věcí tak, aby nějaké základní bloky bylo možné implementovat třeba komplet v BP a neřešit vůbec kód. (To neplatí pokud blok má třeba Electricity component - díky odložené inicializaci by se správně nepropisovaly infa apod)

\subsubsection{Komponenty bloků}
- pak máme komponenty bloků a nějaké interfaces

\subsubsection{Definice bloků}
- popsat způsob definice bloků

\subsubsection{Nalezení bloků}
- popsat block holder a co všechno pro nás znamená


\subsubsection{Ukládání}

- ukládání - máme něco jako block saving helpers

\subsubsection{(Jednotlivé implementace)}

-popsat vstrvení )to se použíá i v BP
 - strom základní tvary / speci. impl (elektr, kyslík) apod

\subsubsection{Základní tvary}
- 3 varianty krychle

\subsubsection{Speciální}
- popsat speciální bloky + nějaké speciality co umějí (showableWidget)


\subsubsection{Hurá na stromy}
popsat stromové struktury, které tam mám