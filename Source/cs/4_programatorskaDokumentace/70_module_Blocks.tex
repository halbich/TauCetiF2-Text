%!TEX root = ../prace.tex

\section{Modul Blocks (C++)}

Modul bloků obsahuje podstatné informace o tom, jak hra pracuje s bloky, jak se tyto bloky skládají do herního světa, jaké jsou jejich komponenty atd. 

- základní definice bloku je v (Block.h)

- block.h definuje hromadu společných věcí tak, aby nějaké základní bloky bylo možné implementovat třeba komplet v BP a neřešit vůbec kód. (To neplatí pokud blok má třeba Electricity component - díky odložené inicializaci by se správně nepropisovaly infa apod)



\subsection{Definice bloků}
- popsat způsob definice bloků, co tam všechno je a proč (podklad pro deifnici v UE editoru

\subsection{Informace o bloku}
popisují stav bloku, jeho serializací a následnou deserializací ze savu lze blok obnovit a korektně umístit do světa


\subsection{Ukládání}

- ukládání - máme něco jako block saving helpers


\subsection{Nalezení bloků}
- popsat block holder a co všechno pro nás znamená, opětovně zmínit herní instanci


\subsection{Komponenty bloků}
- pak máme komponenty bloků a nějaké interfaces

\subsubsection{Elektrická komponenta}


\subsubsection{Elektrická síť}


\subsubsection{Kyslíková komponenta}


\subsubsection{Select target}


\subsubsection{World object}



\subsection{Interfaces}
poskytují nástroje pro volání metod na instaních interfacu

popsat ideu za Implementation, Execute (BlueprintNativeEvent, BlueprintImplementableEvent)



\subsection{Implementace bloků}
 - základ Block.h, zbytek v jendotlivých podkategoriích (BaseShapes / Special
 
 TODO jak moc podrobné? vypsat všechny bloky a co všechno implementují, nebo to stačí stručně zmínit? - co implementují by si čtenář mohl uvědomit z předchozího textu a navíc je to jen nudný popis, jehož výžpovědní hodnota je ve zdrojácích a není asi nutné to tu duplikovat

- popsat speciální bloky + nějaké speciality co umějí (showableWidget)


\subsection{Stromové struktury}
popsat stromové struktury, které tam mám


\subsubsection{MinMaxBox}

prapředek všeho



\subsubsection{KDTree}

dědí z MMB, základ ve světě


\subsubsection{WeatherTargetsKDTree}

dědí z MMB, slouží pro potřeby počasí