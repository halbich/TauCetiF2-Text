
% Abstrakt (doporučený rozsah cca 80-200 slov; nejedná se o zadání práce)
\def\Abstrakt{% 
Mnoho hráčů počítačových her má v~oblibě žánr stavitelských her. Mezi mnohými bychom mohli jmenovat hry \MC{} a~\SE{}. V~těchto hrách hráč staví budovy a~struktury z~bloků pevně dané velikosti. To shledáváme omezujícím a~proto se v~této práci zabýváme novým konceptem, které současné hry nenabízí -- \textit{stavěním} z~\textit{dynamicky škálovatelných} bloků. Cílem je zpříjemnit hráčův zážitek ze hry a~zrychlit stavění rozsáhlých staveb. Hráč však takto může vytvořit velké množství nových bloků, a proto se v~této práci také zabýváme \textit{automatizovanou správou inventáře} bloků, aby hráč zbytečně neztrácel čas hledáním bloků ke stavbě. Tyto herní mechaniky jsme implementovali do nově vzniklé hry \textit{TauCetiF2}. Pro vývoj naší hry jsme zvolili \UE{}, díky čemuž jsme mohli využít rychlosti \textit{C++} a~zároveň přívětivosti technologie \textit{Blueprintů}. Vhodnou kombinací těchto přístupů jsme dosáhli rychlého a~efektivního vývoje celé hry. Z~dotazníku, který byl vytvořen za účelem ověření pochopitelnosti a~zábavnosti těchto mechanik vyplynulo, že se tyto mechaniky hráčům líbí. Očekávaný přínos této práce byl naplněn a~získali jsme nové poznatky, jak tyto mechaniky vylepšit. 
}
\def\AbstraktEN{%
Many computer game players likes building games. Among many, we could mention games \MC{} and \SE{}. In these games, the player builds buildings and structures using blocks of a fixed size. We find it restrictive and therefore we are dealing with a new concept, unused in current games -- \textit{building} from \textit{dynamically scalable} blocks. The goal is to make player's experience more enjoyable and to speed up the construction of extensive buildings. Since player can create a lot of new blocks, we are also dealing with \textit{automated inventory management} for a blocks so the player does not waste time searching for a blocks to build. These game mechanics have been implemented in the newly created game called \textit{TauCetiF2}. To develop our game, we chose \UE{}, thus we could use speed of \textit{C++} and also friendliness of a \textit{Blueprint} technology. A good combination of these approaches has achieved a fast and effective development of the game. From the questionnaire that was created to verify the understanding and fun of these mechanics, it turned out that the players like these mechanics. The expected benefit of this work has been fulfilled and we have gained new insights into how these mechanics could be improved.
}
